\documentclass[a4paper,10pt]{article}

\usepackage{xcolor}
\usepackage{tcolorbox}
\usepackage{amsmath}
\usepackage{booktabs}
\usepackage{multirow}
\usepackage{tabularx}
\usepackage{dblfnote}
\usepackage{lipsum}
\usepackage{geometry}
\usepackage{graphicx}
\geometry{
    a4paper,
    left=1cm,
    right=1cm,
    top=1cm,
    bottom=2cm}
\usepackage{xepersian}
\settextfont{Vazirmatn-Regular.ttf}

\title{گزارش روش طراحی معماری سیستم‌‌های مرتبط با هوش مصنوعی برای پذیرش موفقیت‌آمیز در سازمان‌ها}
\author{محمد خورشیدی روزبهانی\\40215741002013 \and شارا شاهوردیان\\40215741002032 \and ملیکا محمدی گل\\40215741002066}
\date{}

\linespread{1.5}

\begin{document}

    \maketitle

    \vspace{0.5cm}

    \begin{abstract}

        در عصر کنونی، فناوری هوش مصنوعی به عنوان یکی از اولویت‌های اصلی سازمان‌ها برای افزایش بهره‌وری و کارآمدی شناخته می‌شود. با وجود پیشرفت‌های قابل توجه در این زمینه، بسیاری از سازمان‌ها همچنان در درک و اجرای فناوری‌های هوش مصنوعی با چالش‌های جدی مواجه هستند. این چالش‌ها عمدتاً ناشی از نبود دانش کافی در زمینه‌های فنی و نیازهای تخصصی مرتبط با هوش مصنوعی است. عدم توانایی در شناسایی و درک نیازهای فنی مانند داده‌ها و الگوریتم‌های مورد نیاز، یکی از موانع اصلی پذیرش موفقیت‌آمیز هوش مصنوعی در سازمان‌ها به شمار می‌آید.

        برای رفع این مشکل، این مقاله یک روش نوآورانه برای طراحی معماری سیستم هوش مصنوعی معرفی می‌کند که بر اساس ساختار IMO (ورودی-مدل هوش مصنوعی-خروجی) بنا شده است. این ساختار به سازمان‌ها کمک می‌کند تا نیازهای فنی خود را به صورت موثر شناسایی کرده و مدل‌های هوش مصنوعی واقعی و کاربردی توسعه دهند. در حالی که مطالعات پیشین به اهمیت و چالش‌های مرتبط با نیازهای فنی برای پذیرش هوش مصنوعی پرداخته‌اند، اما روش‌های سیستماتیکی برای تجسم این نیازها کمتر مورد بررسی قرار گرفته‌اند.

        روش پیشنهاد شده در این مقاله شامل سه مرحله اصلی است: تعریف مسئله، طراحی راه‌حل هوش مصنوعی، و شناسایی نیازهای فنی هوش مصنوعی. این رویکرد به سازمان‌ها اجازه می‌دهد تا به صورت سیستماتیک و جامع، نیازهای فناوری و فنی خود را شناسایی و برطرف کنند. کارایی و اثربخشی این روش از طریق یک مطالعه موردی، تحلیل‌های مقایسه‌ای با سایر مطالعات، و بررسی‌های کارشناسانه تأیید شده است. نتایج نشان می‌دهد که این روش می‌تواند به طور قابل توجهی به موفقیت سازمان‌ها در پذیرش و اجرای فناوری هوش مصنوعی کمک کند.

    \end{abstract}

    \section{مقدمه}

        با پیشرفت روزافزون فناوری هوش مصنوعی، پذیرش و ادغام موفقیت‌آمیز آن به یکی از اولویت‌های اساسی برای سازمان‌ها تبدیل شده است. پذیرش فناوری هوش مصنوعی می‌تواند منجر به ایجاد ارزش‌های جدید کسب‌وکار و بهبود بهره‌وری و کارآمدی در تصمیم‌گیری‌ها شود. در نتیجه، تلاش‌ها برای ترویج تحول و نوآوری سازمانی به سوی پذیرش فناوری‌های هوش مصنوعی در حال گسترش است و این امر نه تنها در شرکت‌های دیجیتالی مانند مایکروسافت\footnote{\hspace{2pt}Microsoft} و نتفلیکس\footnote{\hspace{2pt}Netflix}، بلکه در سازمان‌های سنتی نظیر صنایع دفاع و راه‌آهن نیز دیده می‌شود. برای مثال، در سال 2018، وزارت دفاع ایالات متحده، مرکز هوش مصنوعی مشترک\footnote{\hspace{2pt}(JAIC) Center AI Joint} را تأسیس کرد تا فناوری هوش مصنوعی را در تمام بخش‌های دفاعی به کار گیرد .

        با این حال، گزارش‌های معتبر از مؤسساتی مانند گارتنر و مک‌کینزی نشان می‌دهند که بسیاری از سازمان‌ها با مشکلاتی در پذیرش فناوری هوش مصنوعی روبرو هستند. از آنجا که هوش مصنوعی یک فناوری حیاتی برای افزایش قابلیت‌های سازمان‌های آینده به شمار می‌آید، انجام تحقیقات برای پذیرش مؤثر آن در سازمان‌ها ضروری است.

        پذیرش هوش مصنوعی در یک سازمان به معنای ادغام آن در سیستم‌ها یا فرآیندهای موجود به منظور انجام وظایف تخصصی حوزه سازمان است. هوش مصنوعی یک حوزه گسترده از فناوری است که از دهه 1950 میلادی مورد تحقیق قرار گرفته و شامل سیستم‌های مختلفی از جمله سیستم‌های متخصص و سیستم‌های مبتنی بر یادگیری ماشین و یادگیری عمیق می‌شود. سیستم‌های مبتنی بر یادگیری ماشین و یادگیری عمیق که امروزه بسیاری از سازمان‌ها به دنبال پذیرش آنها هستند، نیازمند مقادیر زیادی داده و الگوریتم‌های هوش مصنوعی برای استنتاج یا تصمیم‌گیری هوشمند هستن. بنابراین، برای پذیرش مؤثر هوش مصنوعی، سازمان‌ها باید فناوری‌های لازم برای دستیابی به اهداف خود را شناسایی کرده و نیازهای فنی مانند داده‌ها و الگوریتم‌های مورد نیاز را مشخص کنند.

        با این حال، بسیاری از سازمان‌ها هنوز در توضیح دقیق نیازهای خود در زمینه هوش مصنوعی دچار مشکل هستند و کارشناسان هوش مصنوعی نیز در درک نیازهای سازمان‌ها با چالش مواجه‌اند. به عنوان مثال، نیروی دریایی ایالات متحده، با وجود داشتن پیشرفته‌ترین سیستم‌های دفاعی، اذعان کرده که هنوز با وظایف مرتبط با هوش مصنوعی دست و پنجه نرم می‌کند، اگرچه اهمیت داده‌های بزرگ و هوش مصنوعی را درک می‌کند. همچنین، در یک نظرسنجی از کارکنان مایکروسافت\footnote{\hspace{2pt}Microsoft}، مشکلاتی به دلیل عدم توضیحات دقیق درباره مسائل و خواسته‌های نهادها در ساخت مدل‌های پیش‌بینی با استفاده از یادگیری ماشین گزارش شده است.

        این مشکلات ممکن است ناشی از شکاف بین فناوری جدید هوش مصنوعی و دانش حوزه‌ای سازمان‌ها باشد. بسیاری از سازمان‌ها در حوزه‌های خاصی مانند بهداشت، دفاع یا راه‌آهن فعالیت می‌کنند که با فناوری هوش مصنوعی تفاوت‌های قابل توجهی دارند. بنابراین، نیاز به روش‌های جدید برای همکاری نهادها و کارشناسان و تعریف سیستماتیک مسائل و نیازهای فنی برای پذیرش هوش مصنوعی وجود دارد.

        تحقیقات از دیدگاه چند رشته‌ای، مانند قابلیت‌های سازمانی و مهندسی نرم‌افزار، برای پذیرش موفق هوش مصنوعی در سازمان‌ها انجام شده است. مطالعات موجود عمدتاً بر عناصر لازم برای پیاده‌سازی فناوری هوش مصنوعی، مانند مقدار یا کیفیت داده‌ها و توسعه مدل‌های هوش مصنوعی متمرکز هستند. با این حال، با شناسایی اهمیت و چالش‌های این عناصر، تحقیقات کمی درباره چگونگی تجسم سیستماتیک آن‌ها توسط سازمان‌ها انجام شده است.

        برای پذیرش موفق هوش مصنوعی، مهم‌ترین نکته دستیابی به مدل‌های هوش مصنوعی است که بتوانند به نیازهای سازمان پاسخ دهند. مدل‌های هوش مصنوعی محصولات نهایی یادگیری ماشین هستند که وظایف هوش مصنوعی را در سیستم‌ها پیاده‌سازی می‌کنند.

        این مقاله یک روش‌شناسی برای طراحی معماری سیستم هوش مصنوعی به منظور پذیرش موفق هوش مصنوعی در سازمان‌ها پیشنهاد می‌کند. طراحی معماری به تعریف و توسعه مفاهیم، ساختارها و ارتباطات در طول دوره عمر سیستم به منظور اطمینان از موفقیت بهره‌وری اشاره دارد. بر اساس اصول مهندسی سیستم و استانداردهای بین‌المللی مانند \lr{ISO/IEC/IEEE 15288:2015} و \lr{ISO/IEC/IEEE 29148:2018}، طراحی معماری از طریق فرآیند تعریف مسئله و تعریف راه‌حل سیستم انجام می‌شود.

        به همین ترتیب، برای طراحی معماری یک سیستم هوش مصنوعی، فرآیندهای تعریف مسئله و تعریف راه‌حل سیستم هوش مصنوعی لازم است. با این حال، برای تعریف موفق راه‌حل سیستم هوش مصنوعی، مرحله جداگانه‌ای برای تعریف فناوری هوش مصنوعی مورد نیاز است. بنابراین، روش‌شناسی پیشنهاد شده شامل سه مرحله است: تعریف مسئله، راه‌حل سیستم هوش مصنوعی، و راه‌حل فنی هوش مصنوعی. در مرحله تعریف مسئله، فعالیت‌های عملیاتی مورد نیاز برای سازمان طراحی می‌شوند. در مرحله راه‌حل سیستم هوش مصنوعی، ساختار و جریان منابع برای پشتیبانی از فعالیت‌های عملیاتی طراحی می‌شود. نهایتاً، در مرحله راه‌حل فنی هوش مصنوعی، نیازهای فنی برای توسعه مدل‌های هوش مصنوعی مشخص می‌شوند.

        برای شناسایی نیازهای فنی لازم، مفهوم ساختار IMO (ورودی-مدل هوش مصنوعی-خروجی) در تمام مراحل فرآیند طراحی استفاده می‌شود. ساختار IMO به کمترین ساختار منطقی مورد نیاز برای اجرای عملکردهای هوش مصنوعی اشاره دارد.

        این روش‌شناسی به منظور پاسخ به سوالات کلیدی طراحی شده است: چگونه هوش مصنوعی می‌تواند مشکلات تخصصی سازمان را حل کند؟ (Q1) سیستم مورد نظر برای اجرای هوش مصنوعی چیست؟ (Q2) رفتارها و عملکردهای لازم هوش مصنوعی چیست؟ (Q3) و نیازمندی‌های فنی برای دستیابی به هوش مصنوعی چیست؟ (Q4) پاسخ دقیق به این سوالات می‌تواند احتمال موفقیت در پذیرش هوش مصنوعی را افزایش دهد. علاوه بر این، این سوالات به عنوان معیارهای ارزیابی برای روش‌شناسی در فصل چهارم استفاده می‌شوند.

    \section{کارهای مرتبط}

        \subsection{دیدگاه درباره قابلیت‌های سازمانی}

            برای موفقیت در پیاده‌سازی هوش مصنوعی در سازمان‌ها، مطالعات میان‌رشته‌ای مختلفی انجام شده است. سارکر به بررسی جامع انواع و طبقه‌بندی‌های هوش مصنوعی پرداخته و کاربردهای واقعی مانند اتوماسیون، هوش، و سیستم‌های هوشمند را که از فناوری‌های کلیدی در انقلاب صنعتی چهارم هستند، تبیین می‌کند. وی ادعا می‌کند که توسعه یک مدل هوش مصنوعی موثر به دلیل محیط عملیاتی پویا، داده‌ها و دیگر عوامل چالش برانگیز است و دیدگاه مدل‌سازی مبتنی بر هوش مصنوعی را به عنوان یک راهنمای مرجع برای دانشمندان، متخصصان صنعتی و تصمیم‌گیران ارائه می‌دهد.

            میکالف و همکاران کاربردهای هوش مصنوعی را بررسی کرده و آن را به عنوان منبعی برای ایجاد ارزش تجاری از دیدگاه سازمانی تعریف می‌کنند. آن‌ها قابلیت‌های هوش مصنوعی را شناسایی کرده و از دسته‌بندی‌های مشخصی که توسط گرنت ارائه شده استفاده می‌کنند. این دسته‌بندی‌ها شامل منابع ملموس مانند داده‌ها و فناوری، مهارت‌های انسانی نظیر مهارت‌های فنی و تجاری و منابع غیرمحسوس مانند هماهنگی بین‌بخشی است. تحقیق آن‌ها به بررسی میزان داده‌ها، قابلیت یکپارچه‌سازی آن‌ها و نیازمندی‌های فنی برای توسعه فناوری هوش مصنوعی می‌پردازد.

            دسوزا و همکاران نیز مسائل مربوط به اتخاذ هوش مصنوعی را از دیدگاه سازمانی بررسی می‌کنند. آن‌ها از طریق تجربه طراحی، توسعه و استقرار یک سیستم محاسبات شناختی (CCS) در بخش عمومی، چهار چالش کلیدی شامل داده، فناوری، سازمان و محیط را مطرح می‌کنند. هرچند روش پیشنهادی در این مقاله تمامی قابلیت‌ها یا چالش‌های سازمانی مطرح شده توسط آن‌ها را در بر نمی‌گیرد، اما به طور عمده بر روی زمینه‌های داده و فناوری تمرکز دارد.

            ناگبول و همکاران رویکردی برای پیاده‌سازی هوش مصنوعی غیرقابل تفسیر به شیوه‌ای مسئولانه و ایمن در سازمان‌ها ارائه می‌دهند. آن‌ها از یک روش طبقه‌بندی مفهومی برای محافظت از داده‌های آموزشی و ورودی/خروجی هوش مصنوعی استفاده کرده و روشی برای مدیریت تعادل بین تفسیرپذیری و عملکرد هوش مصنوعی در زمان استقرار آن در سازمان پیشنهاد می‌دهند. تحقیق آن‌ها به عواملی که در توسعه هوش مصنوعی از دیدگاه سازمانی باید مورد توجه قرار گیرد مانند داده‌های آموزشی و ورودی/خروجی پرداخته، اما به جنبه‌های سیستمی توجه کمتری دارد.

            در نهایت، اگرچه هر یک از این تحقیقات دیدگاه‌های مهمی را در مورد اتخاذ هوش مصنوعی ارائه می‌دهند، اما روش‌شناسی پیشنهاد شده در این مقاله عمدتاً به شناسایی و رفع نیازهای فنی و داده‌ای مورد نیاز برای توسعه مدل‌های هوش مصنوعی تمرکز دارد. این روش‌شناسی از ساختار IMO (ورودی-مدل هوش مصنوعی-خروجی) برای تعریف و طراحی سیستم‌های هوش مصنوعی استفاده می‌کند تا فرآیندهای پیاده‌سازی را به طور سیستماتیک و دقیق انجام دهد.

        \subsection{دیدگاه مهندسی سیستم / نرم‌افزار}

            آلوارز-رودریگز و همکاران چالش‌های مرتبط با ادغام چرخه عمر مدل‌های هوش مصنوعی با فرآیندهای مهندسی نرم‌افزار را بررسی کردند. این چالش‌ها شامل توصیف نیازها و قابلیت‌ها مانند داده، فناوری و سخت‌افزار است، که باید چرخه عمر هوش مصنوعی/یادگیری ماشین را در فرآیند مشخصات‌گذاری سیستم ادغام کنند. برای حل این چالش‌ها، آن‌ها یک معماری مفهومی پیشنهاد کردند. با این حال، در تحقیقات آن‌ها به روش‌هایی برای تجسم نیازمندی‌های فنی هوش مصنوعی پرداخته نشده است. در این مقاله، نیازمندی‌ها و سطح طراحی معماری را از معماری مفهومی پیشنهادی آلوارز-رودریگز و همکاران تجسم می‌کنیم.

            بلانی و همکاران نیز چالش‌های مرتبط با توسعه سیستم‌های پیچیده مبتنی بر هوش مصنوعی را از دیدگاه مهندسی نیازها شناسایی کردند. به عنوان بخشی از تحقیقات مهندسی نیازمندی برای هوش مصنوعی، آن‌ها چالش‌های مربوط به داده، مدل‌ها، سیستم‌ها و فعالیت‌های مهندسی نیازها (تجزیه و تحلیل، مشخصات‌گذاری، تأیید و غیره) را تجسم کردند. آن‌ها موجودیت‌های مرتبط با هوش مصنوعی لازم برای ساختن سیستم‌های پیچیده مبتنی بر هوش مصنوعی را از دیدگاه مهندسی نیازها به سه دسته داده، مدل (هوش مصنوعی) و سیستم (هوش مصنوعی) تقسیم‌بندی کردند. این دسته‌بندی‌ها مشابه عناصری هستند که در این تحقیق قصد داریم تجسم کنیم و نشان می‌دهد که روش ما می‌تواند از دیدگاه مهندسی نیازها مفید باشد. ما روش تجسم موجودیت‌های مرتبط با هوش مصنوعی پیشنهادی بلانی و همکاران را به کار گرفته‌ایم.

            احمد و همکاران بیان می‌کنند که نیاز به فناوری جدیدی برای دریافت نیازهای هوش مصنوعی به عنوان نتیجه ظهور این فناوری وجود دارد. آن‌ها شکافی را کشف کرده‌اند که نیاز به پل زدن بین مهندسان و متخصصان داده/هوش مصنوعی را برای دریافت نیازهای هوش مصنوعی و گسترش یا تکمیل زبان‌های مدل‌سازی دارد. در همین راستا، گردس یک رویکرد مشارکتی متمرکز بر داده برای طراحی اخلاقی هوش مصنوعی پیشنهاد داده است. آن‌ها بر اهمیت همکاری بین توسعه‌دهندگان یادگیری ماشین و متخصصان حوزه برای طراحی متمرکز بر داده تأکید دارند، زیرا عملکرد مدل‌های یادگیری ماشین توسط داده تعیین می‌شود.

            موچینی که معماری نرم‌افزار برای سیستم‌های مبتنی بر یادگیری ماشین را مورد مطالعه قرار داده است، ادعا می‌کند که سیستم‌های یادگیری ماشین مسائل جدیدی را معرفی می‌کنند که نمی‌توان آن‌ها را از طریق چارچوب‌های معماری نرم‌افزار استاندارد حل کرد. این موضوع نیاز به توسعه چارچوب‌های نرم‌افزار جدید را ایجاب می‌کند. با اینکه این تحقیق به موضوع توسعه زبان‌های مدل‌سازی مانند SysML نمی‌پردازد، با این نظر موافقیم که یک شکاف بین متخصصان حوزه و متخصصان هوش مصنوعی وجود دارد که باید پل زده شود. ما به روش‌های عملی برای پل زدن این شکاف بین مهندسان و متخصصان داده/هوش مصنوعی پیشنهادی احمد و همکاران و گردس پرداخته‌ایم.

        \subsection{دیدگاهی در مورد طراحی معماری}

            با اینکه بسیاری از سازمان‌ها در جامعه مدرن از سیستم‌های پیچیده تشکیل شده‌اند، تا حد دانش ما، تحقیقات در خصوص ادغام هوش مصنوعی از دیدگاه طراحی معماری سیستم محدود بوده است. دو مطالعه شناسایی شده است که از این دیدگاه انجام شده‌اند. تاکدا و همکاران یک روش توسعه معماری را ارائه دادند که با استفاده از ،SysML یک زبان مدل‌سازی سیستم، به عنوان مثال از یک ربات هوش مصنوعی، بر شفافیت و مسئولیت تأکید می‌کند. آن‌ها ادعا می‌کنند که توصیف کامل سیستم هوش مصنوعی به شفافیت هوش مصنوعی کمک می‌کند. روش آن‌ها بر روی نمایش سیستم از منظر کلی تمرکز دارد. روش پیشنهاد شده در این مقاله نیز نه تنها دیدگاه سیستم لازم برای توصیف سیستم را پوشش می‌دهد، بلکه دیدگاه‌های عملی و فناوری هوش مصنوعی را نیز شامل می‌شود. بنابراین، نظرات آن‌ها در مورد شفافیت هوش مصنوعی نشان می‌دهد که این تحقیق نیز می‌تواند به شفافیت هوش مصنوعی کمک کند.

            جولیان آی. جونز و همکاران معماری یک سیستم دفاع هوایی و موشکی\footnote{\hspace{2pt}AMD} را با استفاده از چارچوب معماری دفاع\footnote{\hspace{2pt}DoDAF} طراحی کردند. برای طراحی معماری، قسمتی از حلقه OODA\footnote{\hspace{2pt}مشاهده، جهت‌دهی، تصمیم‌گیری، عمل} و مدل‌های توضیحی OV\footnote{\hspace{2pt}نقطه‌نظر عملیاتی} و SV\footnote{\hspace{2pt}نقطه‌نظر سیستم} استفاده شد. تحقیق آن‌ها فرآیند یک سیستم دفاع هوایی و موشکی را از طریق حلقه OODA تجزیه می‌کند و معماری را برای هر مرحله تجسم می‌کند. تمرکز اصلی هوش مصنوعی در تحقیق آن‌ها بر روی اتوماسیون است. از طریق اتوماسیون با هوش مصنوعی، آن‌ها بررسی می‌کنند که چگونه می‌توان زمان‌بندی حلقه OODA را بهبود بخشید (مانند شناسایی سریع‌تر اهداف). تحقیق آن‌ها به شیوه‌ای تا حدودی مشابه با روش پیشنهادی این مقاله در شناسایی وظایف هوش مصنوعی مورد نیاز از دیدگاه عملیاتی و سیستمی عمل می‌کند. با این حال، وظایف هوش مصنوعی شناسایی شده از طریق معماری به سطح انتزاعی (مانند استدلال فضایی) محدود است و داده‌های لازم شناسایی نمی‌شود.

    \section{روش‌شناسی طراحی معماری سیستم هوش مصنوعی مبتنی بر ساختار IMO}

        \subsection{ساختار IMO}

            هدف این بخش از مقاله، روشن‌سازی انگیزه استفاده از ساختار IMO در روش طراحی معماری سیستم هوش مصنوعی پیشنهادی است. به این منظور، ساختار IMO تعریف و ضرورت آن توضیح داده می‌شود، و ملاحظات فنی که در فرآیند طراحی باید مدنظر قرار گیرند از دیدگاه‌های فناوری هوش مصنوعی و جنبه‌های سیستمی معرفی می‌شوند.

            \subsubsection{تعریف و ضرورت ساختار IMO}

                ساختار IMO، که مخفف ورودی-مدل هوش مصنوعی-خروجی است، به عنوان چارچوبی برای شناسایی و تجسم نیازمندی‌های فنی لازم برای توسعه مدل‌های هوش مصنوعی مورد استفاده قرار می‌گیرد. این ساختار به سازمان‌ها کمک می‌کند تا به وضوح ورودی‌های مورد نیاز، فرآیندهای مدل‌سازی هوش مصنوعی، و خروجی‌های مطلوب را تعریف و درک کنند. با استفاده از ساختار IMO، می‌توان به یک دید جامع و سیستماتیک از نیازهای فنی برای طراحی و پیاده‌سازی موفق مدل‌های هوش مصنوعی دست یافت.

            \subsubsection{ملاحظات فنی در فرآیند طراحی}

                در فرآیند طراحی معماری سیستم هوش مصنوعی، باید به دو جنبه اصلی توجه شود: فناوری هوش مصنوعی و جنبه‌های سیستمی.

                \begin{itemize}
                    
                    \item فناوری هوش مصنوعی:

                    \begin{itemize}
                        
                        \item داده‌ها:
                        
                        \begin{itemize}
                            
                            \item حجم و کیفیت داده‌ها: اطمینان از اینکه داده‌های کافی و با کیفیت برای آموزش و ارزیابی مدل‌های هوش مصنوعی وجود دارد.

                            \item منبع داده‌ها: شناسایی منابع داده‌های موجود و اطمینان از دسترسی به آن‌ها.

                            \item پردازش داده‌ها: فرآیندهای لازم برای پاکسازی، آماده‌سازی و پیش‌پردازش داده‌ها.

                        \end{itemize}

                        \item الگوریتم‌ها:

                        \begin{itemize}
                            
                            \item انتخاب الگوریتم‌های مناسب: انتخاب الگوریتم‌های یادگیری ماشین و یادگیری عمیق که بهترین عملکرد را برای مسائل خاص سازمان دارند.

                            \item بهینه‌سازی مدل‌ها: فرآیندهای بهینه‌سازی مدل‌ها برای دستیابی به بهترین عملکرد ممکن.

                        \end{itemize}

                        \item محاسبات:

                        \begin{itemize}
                            
                            \item منابع محاسباتی: شناسایی نیازهای محاسباتی برای آموزش و اجرای مدل‌ها.

                            \item زیرساخت‌های محاسباتی: استفاده از زیرساخت‌های مناسب مانند پردازش موازی و محاسبات ابری.

                        \end{itemize}

                    \end{itemize}

                    \item جنبه‌های سیستمی:

                    \begin{itemize}

                        \item یکپارچگی سیستم:
                        
                        \begin{itemize}
                            
                            \item تعاملات و وابستگی‌ها: شناسایی تعاملات و وابستگی‌های بین مدل‌های هوش مصنوعی و سایر اجزای سیستم.

                            \item یکپارچه‌سازی سیستم: برنامه‌ریزی برای یکپارچه‌سازی مدل‌های هوش مصنوعی با سیستم‌های موجود.

                        \end{itemize}

                        \item عملکرد سیستم:

                        \begin{itemize}
                            
                            \item کارایی و مقیاس‌پذیری: اطمینان از کارایی و مقیاس‌پذیری مدل‌های هوش مصنوعی در محیط عملیاتی.

                            \item نگهداری و پشتیبانی: برنامه‌ریزی برای نگهداری و پشتیبانی از مدل‌های هوش مصنوعی در طول دوره عمر آن‌ها.

                        \end{itemize}

                        \item امنیت و حریم خصوصی:

                        \begin{itemize}
                            
                            \item حفاظت از داده‌ها: اطمینان از اینکه داده‌ها به درستی حفاظت می‌شوند و مسائل مربوط به حریم خصوصی رعایت می‌شوند.

                            \item امنیت سیستم: اطمینان از اینکه مدل‌ها و سیستم‌های هوش مصنوعی در برابر تهدیدات امنیتی محافظت می‌شوند.

                        \end{itemize}

                    \end{itemize}

                \end{itemize}

                با استفاده از ساختار IMO، می‌توان به یک رویکرد منظم و سیستماتیک برای شناسایی و تجسم نیازهای فنی در طراحی معماری سیستم‌های هوش مصنوعی دست یافت. این رویکرد می‌تواند به سازمان‌ها کمک کند تا به صورت موثرتری از فناوری هوش مصنوعی بهره‌برداری کنند و چالش‌های مرتبط با آن را بهتر مدیریت کنند.

\end{document}