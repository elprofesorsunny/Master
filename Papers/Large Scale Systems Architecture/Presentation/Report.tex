\documentclass[a4paper,10pt]{article}

\usepackage{xcolor}
\usepackage{tcolorbox}
\usepackage{amsmath}
\usepackage{booktabs}
\usepackage{multirow}
\usepackage{tabularx}
\usepackage{geometry}
\usepackage{graphicx}
\geometry{
    a4paper,
    left=1cm,
    right=1cm,
    top=1cm,
    bottom=2cm}
\usepackage{xepersian}
\settextfont{Vazirmatn-Regular.ttf}

\title{گزارش روش طراحی معماری سیستم‌‌های مرتبط با هوش مصنوعی برای پذیرش موفقیت‌آمیز در سازمان‌ها}
\author{محمد خورشیدی روزبهانی\\40215741002013 \and شارا شاهوردیان\\40215741002032 \and ملیکا محمدی گل\\40215741002066}
\date{}

\linespread{1.5}

\begin{document}

    \maketitle

    \vspace{0.5cm}

    \begin{abstract}

        در عصر کنونی، فناوری هوش مصنوعی به عنوان یکی از اولویت‌های اصلی سازمان‌ها برای افزایش بهره‌وری و کارآمدی شناخته می‌شود. با وجود پیشرفت‌های قابل توجه در این زمینه، بسیاری از سازمان‌ها همچنان در درک و اجرای فناوری‌های هوش مصنوعی با چالش‌های جدی مواجه هستند. این چالش‌ها عمدتاً ناشی از نبود دانش کافی در زمینه‌های فنی و نیازهای تخصصی مرتبط با هوش مصنوعی است. عدم توانایی در شناسایی و درک نیازهای فنی مانند داده‌ها و الگوریتم‌های مورد نیاز، یکی از موانع اصلی پذیرش موفقیت‌آمیز هوش مصنوعی در سازمان‌ها به شمار می‌آید.

        برای رفع این مشکل، این مقاله یک روش نوآورانه برای طراحی معماری سیستم هوش مصنوعی معرفی می‌کند که بر اساس ساختار IMO (ورودی-مدل هوش مصنوعی-خروجی) بنا شده است. این ساختار به سازمان‌ها کمک می‌کند تا نیازهای فنی خود را به صورت موثر شناسایی کرده و مدل‌های هوش مصنوعی واقعی و کاربردی توسعه دهند. در حالی که مطالعات پیشین به اهمیت و چالش‌های مرتبط با نیازهای فنی برای پذیرش هوش مصنوعی پرداخته‌اند، اما روش‌های سیستماتیکی برای تجسم این نیازها کمتر مورد بررسی قرار گرفته‌اند.

        روش پیشنهاد شده در این مقاله شامل سه مرحله اصلی است: تعریف مسئله، طراحی راه‌حل هوش مصنوعی، و شناسایی نیازهای فنی هوش مصنوعی. این رویکرد به سازمان‌ها اجازه می‌دهد تا به صورت سیستماتیک و جامع، نیازهای فناوری و فنی خود را شناسایی و برطرف کنند. کارایی و اثربخشی این روش از طریق یک مطالعه موردی، تحلیل‌های مقایسه‌ای با سایر مطالعات، و بررسی‌های کارشناسانه تأیید شده است. نتایج نشان می‌دهد که این روش می‌تواند به طور قابل توجهی به موفقیت سازمان‌ها در پذیرش و اجرای فناوری هوش مصنوعی کمک کند.

    \end{abstract}

    \section{مقدمه}

        با پیشرفت روزافزون فناوری هوش مصنوعی، پذیرش و ادغام موفقیت‌آمیز آن به یکی از اولویت‌های اساسی برای سازمان‌ها تبدیل شده است. پذیرش فناوری هوش مصنوعی می‌تواند منجر به ایجاد ارزش‌های جدید کسب‌وکار و بهبود بهره‌وری و کارآمدی در تصمیم‌گیری‌ها شود. در نتیجه، تلاش‌ها برای ترویج تحول و نوآوری سازمانی به سوی پذیرش فناوری‌های هوش مصنوعی در حال گسترش است و این امر نه تنها در شرکت‌های دیجیتالی مانند مایکروسافت\footnote{\hspace{2pt}Microsoft} و نتفلیکس\footnote{\hspace{2pt}Netflix}، بلکه در سازمان‌های سنتی نظیر صنایع دفاع و راه‌آهن نیز دیده می‌شود. برای مثال، در سال 2018، وزارت دفاع ایالات متحده، مرکز هوش مصنوعی مشترک (JAIC) را تأسیس کرد تا فناوری هوش مصنوعی را در تمام بخش‌های دفاعی به کار گیرد .

        با این حال، گزارش‌های معتبر از مؤسساتی مانند گارتنر و مک‌کینزی نشان می‌دهند که بسیاری از سازمان‌ها با مشکلاتی در پذیرش فناوری هوش مصنوعی روبرو هستند. از آنجا که هوش مصنوعی یک فناوری حیاتی برای افزایش قابلیت‌های سازمان‌های آینده به شمار می‌آید، انجام تحقیقات برای پذیرش مؤثر آن در سازمان‌ها ضروری است.



\end{document}