\documentclass[a4paper,10pt]{article}

\usepackage{geometry}
\geometry{
    a4paper,
    left=1cm,
    right=1cm,
    top=1cm,
    bottom=2cm}
\usepackage{xepersian}
\settextfont{Vazirmatn-Regular.ttf}


\title{گزارش روش طراحی معماری سیستم هوش مصنوعی بر اساس ساختار IMO Model-Output) (Input-AI برای پذیرش موفقیت‌آمیز هوش مصنوعی در سازمان‌ها}
\author{محمد خورشیدی روزبهانی\\40215741002013 \and شارا شاهوردیان\\40215741002032 \and ملیکا محمدی گل\\40215741002066}
\date{}

\linespread{1.5}

\begin{document}

    \maketitle

    \vspace{0.5cm}

    \begin{abstract}
        
        با پیشرفت فناوری هوش مصنوعی، بهبود موفقیت فناوری هوش مصنوعی در سازمان‌ها اولویت اصلی جامعه مدرن شده است. با این حال، بسیاری از سازمان‌ها هنوز به بیان فناوری های هوش مصنوعی لازم و کارشناسان هوش مصنوعی دچار مشکلات در فهم مسائلی که این سازمان‌ها با آنها روبرو هستند، دچار مشکل هستند. این شکاف دانشی باعث می‌شود که برای سازمان‌ها امکان شناسایی نیازهای فنی، مانند داده‌ها و الگوریتم‌های لازم برای اتخاذ فناوری هوش مصنوعی، مشکل باشد. برای پاسخ به این مشکل، ما یک روش طراحی معماری سیستم هوش مصنوعی جدید بر اساس ساختار IMO (ورودی-مدل هوش مصنوعی-خروجی) پیشنهاد می‌دهیم. ساختار IMO امکان شناسایی موثر نیازهای فنی لازم برای توسعه مدل‌های واقعی هوش مصنوعی را فراهم می‌کند. در حالی که تحقیقات قبلی اهمیت و چالش‌های نیازهای فنی، مانند داده‌ها و الگوریتم‌های هوش مصنوعی، برای اتخاذ فناوری هوش مصنوعی را شناسایی کرده‌اند، تحقیقات کمی در زمینه روش‌شناسی برای تجسم آنها صورت نگرفته است. روش‌شناسی ما از سه مرحله تشکیل شده است: تعریف مسئله، راه‌حل AI سیستم، و راه‌حل فنی هوش مصنوعی برای طراحی فناوری و نیازهایی که سازمان‌ها به صورت سیستمی نیاز دارند استفاده می‌کند. اثربخشی روش ما از طریق یک مطالعه موردی، تحلیل مقایسه‌ای منطقی با دیگر مطالعات، و بررسی توسط کارشناسان، که نشان می‌دهد روش ما می‌تواند به موفقیت اتخاذ فناوری هوش مصنوعی در سازمان‌ها کمک کند، اثبات شده است.

    \end{abstract}

    \section{مقدمه}

        همچنان که فناوری هوش مصنوعی پیشرفت می‌کند، اتخاذ موفق فناوری هوش مصنوعی برای بسیاری از سازمان‌ها اولویت اصلی شده است. با اتخاذ فناوری هوش مصنوعی، سازمان‌ها می‌توانند ارزش کسب و کار جدیدی ایجاد و به بهره‌وری و کارآمدی در تصمیم‌گیری‌ها افزوده شود [51،52]. به عنوان یک نتیجه، تلاش‌ها برای ترویج تحول و نوآوری سازمانی به سوی اتخاذ فناوری‌های هوش مصنوعی، نه تنها در شرکت‌های دیجیتالی مانند مایکروسافت یا نت‌فلیکس، بلکه در سازمان‌های سنتی مانند صنایع دفاع و راه‌آهن نیز تقویت می‌شود [53]. به عنوان مثال، در سال ۲۰۱۸، وزارت دفاع ایالات متحده مرکز هوش مصنوعی مشترک (JAIC) را تأسیس کرد، یک دپارتمان مخصوص برای اتخاذ فناوری هوش مصنوعی در تمام زمینه‌های دفاعی [33]. با این حال، طبق تحقیقات معتبری مانند گارتنر و مک‌کینزی، بسیاری از سازمان‌ها با مشکلاتی در اتخاذ فناوری هوش مصنوعی روبه‌رو هستند [[1]، [2]، [3]، [34]، [35]، [36]، [37]، [38]، [39]، [40]]. از آنجایی که هوش مصنوعی یک فناوری اساسی برای افزایش قابلیت‌های سازمان‌های آینده است، تحقیقات برای اتخاذ موثر هوش مصنوعی در سازمان‌ها ضروری است.

        اتخاذ هوش مصنوعی در یک سازمان به معنای گنجاندن هوش مصنوعی در سیستم‌ها یا فرآیندهای موجود برای انجام وظایف تخصصی حوزه سازمان است. هوش مصنوعی یک حوزه گسترده از فناوری است که از دهه ۱۹۵۰ میلادی مورد تحقیق قرار گرفته است و شامل انواع سیستم‌هایی مانند سیستم‌های متخصص یا سیستم‌های مبتنی بر یادگیری ماشین/یادگیری عمیق می‌شود [55،64]. در میان آنها، سیستم‌های هوش مصنوعی مبتنی بر یادگیری ماشین/یادگیری عمیق که بسیاری از سازمان‌ها در حال تلاش برای اتخاذ آنها هستند، نیاز به مقدار زیادی داده و الگوریتم‌های هوش مصنوعی برای استنتاج یا تصمیم‌گیری هوشمند دارند [14،15]. بنابراین، برای سازمان‌ها برای اتخاذ موثر هوش مصنوعی، آنها باید فناوری‌های هوش مصنوعی لازم برای دستیابی به اهداف سازمان را شناسایی کنند و نیازمندی‌های فنی مانند داده و الگوریتم‌های هوش مصنوعی مورد نیاز برای تحقق آنها را شناسایی کنند. با این حال، بسیاری از سازمان‌ها هنوز دشواری در توضیح روشن هوش مصنوعی مورد نیاز خود دارند و کارشناسان هوش مصنوعی دشواری در فهم عملکردهای هوش مصنوعی مورد نیاز سازمان‌ها دارند. به عنوان مثال، نیروی دریایی ایالات متحده، که پیشرفته‌ترین سیستم‌های دفاعی را در سراسر جهان دارد، اقرار کرده است که هنوز با وظایفی که هوش مصنوعی نیاز دارد، دست و پنجه نرم می‌کنند، اگرچه اهمیت داده‌های بزرگ و هوش مصنوعی را درک می‌کنند [4]. علاوه بر این، طبق یک نظرسنجی از کارمندان مایکروسافت، اغلب هنگام ساختن مدل‌های پیش‌بینی با استفاده از یادگیری ماشین، آنها با مشکلاتی به دلیل عدم توضیحات روشن درباره مسائل و خواسته‌های نهادها برای دیدن اتفاقات جادویی از داده مواجه می‌شوند [19]. همانطور که در موارد مطرح شده، چالش‌ها اساساً برای سازمان‌ها حتی در حوزه‌های مختلف به یکسان است که در تعریف روشن هوش مصنوعی مورد نیاز شکست می‌خورند. این مشکلات ممکن است ناشی از شکاف بین فناوری جدید مانند هوش مصنوعی و دانش حوزه‌ای سازمان‌های موجود باشد [6،32]. بسیاری از سازمان‌های واقعی وظایف مخصوص حوزه‌ای را انجام می‌دهند که مناطق منحصر به فردی مانند بهداشت، دفاع یا راه‌آهن هستند، که کاملاً با فناوری هوش مصنوعی متفاوت هستند. بنابراین، برای حل این مشکلات، نیاز به روش‌های جدیدی برای همکاری نهادها و کارشناسان و تعریف مسائل به صورت سیستمی و مشخص کردن نیازمندی‌های فنی لازم برای اتخاذ هوش مصنوعی وجود دارد.

        تحقیقات از دیدگاه چند رشته‌ای، مانند قابلیت‌های سازمانی و مهندسی نرم‌افزار (SE)، برای اتخاذ موفق هوش مصنوعی در سازمان‌ها انجام می‌شود. مطالعات موجود نیز به طور اصلی بر عناصر لازم برای پیاده‌سازی فناوری هوش مصنوعی متمرکز هستند، مانند مقدار یا کیفیت داده و توسعه مدل‌های هوش مصنوعی (جهت کسب اطلاعات بیشتر به فصل ۲ مراجعه شود). با این حال، با شناسایی اهمیت و چالش‌های این عناصر ضروری برای پیاده‌سازی فناوری‌های هوش مصنوعی، هنوز کمبود تحقیقاتی در مورد اینکه چگونه سازمان‌ها می‌توانند آنها را به صورت سیستماتیک تجسم کنند، وجود دارد. برای اتخاذ موفق هوش مصنوعی، مهمترین چیز، به دست آوردن مدل‌های هوش مصنوعی لازم است که بتوانند به نیازهای سازمان پاسخ دهند. زیرا مدل‌های هوش مصنوعی، محصول نهایی یادگیری ماشین هستند و موضوعاتی هستند که وظایف هوش مصنوعی را در سیستم‌ها پیاده‌سازی می‌کنند.

        در این مطالعه، یک روش‌شناسی برای طراحی معماری سیستم هوش مصنوعی به منظور اتخاذ موفق هوش مصنوعی در سازمان‌ها پیشنهاد می‌دهیم. طراحی معماری به فعالیت تعریف و توسعه مفاهیم، ساختارها و ارتباطات در طول دوره عمر سیستم مورد علاقه به منظور اطمینان از موفقیت بهره‌وری اشاره دارد [7،8]. به طور کلی، طبق اصول اساسی مهندسی سیستم یا استانداردهای بین‌المللی معتبری مانند ISO/\lr{IEC/IEEE 15288:2015}، \lr{ISO/IEC/IEEE29148:2018}، طراحی معماری از طریق فرآیند تعریف مسئله و تعریف راه‌حل سیستم انجام می‌شود [7،8]. به همین ترتیب، برای طراحی معماری یک سیستم هوش مصنوعی، فرآیندهای تعریف مسئله مرتبط با هوش مصنوعی و تعریف راه‌حل سیستم هوش مصنوعی لازم است. با این حال، برای تعریف موفق راه‌حل سیستم هوش مصنوعی، مرحله جداگانه‌ای برای تعریف فناوری هوش مصنوعی مورد نیاز در سیستم هوش مصنوعی لازم است. بنابراین، در روش‌شناسی ما، طراحی معماری از طریق سه مرحله انجام می‌شود: تعریف مسئله، راه‌حل سیستم هوش مصنوعی، و راه‌حل فنی هوش مصنوعی. در مرحله تعریف مسئله، طراحی فعالیت‌های عملیاتی مورد نیاز برای سازمان در آینده نسبت به حال انجام می‌شود. در مرحله راه‌حل سیستم هوش مصنوعی، ساختار و جریان منابع سیستم برای پشتیبانی از فعالیت‌های عملیاتی طراحی می‌شود. در نهایت، در مرحله راه‌حل فنی هوش مصنوعی، نیازمندی‌های فنی مورد نیاز برای به دست آوردن مدل هوش مصنوعی مورد نیاز مشخص می‌شوند. به طور خاص، برای شناسایی نیازمندی‌های فنی لازم برای توسعه واقعی مدل‌های هوش مصنوعی، مفهوم ساختار IMO (ورودی-مدل هوش مصنوعی-خروجی) در تمام مراحل فرآیند طراحی استفاده می‌شود. ساختار IMO به کمترین ساختار منطقی مورد نیاز برای اجرای عملکردهای هوش مصنوعی اشاره دارد. نهایتاً، روش‌شناسی ما برای پاسخ به سوالات حداقل لازم برای موفقیت در اتخاذ سیستم هوش مصنوعی، ابتکار شد. هدف این سوالات عبارتند از: چگونه هوش مصنوعی می‌تواند مشکلات تخصص حوزه سازمان را حل کند؟ (Q1) سیستم مورد علاقه برای حقیقت اجرای هوش مصنوعی چیست؟ (Q2) رفتارها و عملکردهای لازم هوش مصنوعی چیست؟ (Q3) و در نهایت، نیازمندی‌های برای به دست آوردن هوش مصنوعی مورد نیاز چیست؟ (Q4). اگر بتوانیم به این سوالات به طور موشکاف پاسخ دهیم، احتمال موفقیت در اتخاذ هوش مصنوعی در سازمان‌ها افزایش خواهد یافت. علاوه بر این، این سوالات به عنوان موارد ارزیابی برای روش‌شناسی ما در فصل چهارم استفاده می‌شوند.

        قسمت باقی‌مانده این مقاله به شکل زیر سازمان‌دهی شده است. فصل ۲، تحلیلی از مطالعات مرتبط انجام شده تاکنون ارائه می‌دهد، و فصل ۳ روش پیشنهادی را به طور دقیق توضیح می‌دهد. در فصل ۴، موردهای نمونه را نشان می‌دهد و تحلیل می‌کند، فصل ۵ شامل بحث‌ها می‌شود، و در نهایت فصل ۶، نتیجه‌گیری‌ها و جهت‌های تحقیقات آینده را ارائه می‌دهد.

    \section{کارهای مرتبط و محدودیت‌هایشان}

        \subsection{کارهای مرتبط}

            \subsubsection{دیدگاه درباره قابلیت‌های سازمانی}

                متن

\end{document}