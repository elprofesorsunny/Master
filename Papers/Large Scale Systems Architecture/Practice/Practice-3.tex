\documentclass[a4paper,10pt]{article}

\usepackage{geometry}
\geometry{
    left=1cm,
    right=1cm,
    top=2cm,
    bottom=2cm
}

\usepackage{fancyhdr}
\pagestyle{fancy}
\fancyhf{}
\fancyhead[R]{معماری سیستم‌های مقیاس بزرگ}
\fancyhead[C]{تمرین 3}
\fancyhead[L]{\today}
\fancyfoot[R]{محمد خورشیدی روزبهانی - 40215741002013}
\fancyfoot[L]{\thepage}

\usepackage{xepersian}
\settextfont{Vazirmatn-Regular.ttf}

\linespread{1.5}

\begin{document}

    \paragraph{سوال:} چه روش‌هایی برای افزایش مقیاس پایگاه‌داده‌های رابطه‌ای Oracle به صورت افقی وجود دارد؟\\(منظور از مقیاس‌پذیزی افقی، Scalability Horizontal می‌باشد.)

    \paragraph{پاسخ:} مقیاس‌پذیری افقی یک مفهوم مهم در معماری سیستم است که به توانایی سیستم برای افزایش ظرفیت و عملکرد با افزایش تعداد منابع سخت‌افزاری یا نرم‌افزاری اشاره دارد. به عبارت دیگر، افزایش مقیاس پذیری افقی به معنای اضافه کردن سرورها یا نودهای جدید به سیستم است تا بتواند بار کاری بیشتری را پردازش کند.

    در مقیاس‌پذیری افقی، داده‌ها و بار کاری بین چندین سرور یا نود تقسیم می‌شوند. این به این معناست که هر سرور یا node مسئولیت بخش خاصی از کار را بر عهده دارد و این تقسیم‌بندی اجازه می‌دهد تا بار کاری به صورت موازی بین این سرورها یا node ها تقسیم شود. اگر یکی از سرورها یا node ها بیش از حد بار کاری را تحمل کند، سیستم می‌تواند از توزیع مجدد بار کاری یا افزودن سرورهای جدید به سیستم استفاده کند. اما یکی از مسائل حائز اهمیت، مزایای مقیاس‌پذیری افقی است که در زیر به چند مورد از این موارد اشاره خواهد شده است:

    \begin{itemize}
        
        \item افزایش ظرفیت: با اضافه کردن سرورها یا node های جدید، سیستم قادر است بیشترین بار کاری را پردازش کند و ظرفیت خود را افزایش دهد.

        \item افزایش قابلیت اطمینان: با توزیع بار کاری بین چندین سرور یا node، سیستم قابلیت اطمینان و دسترسی بالاتری را ارائه می‌دهد. اگر یکی از سرورها یا node ها خراب شود، سایرین می‌توانند کار را ادامه دهند.

        \item افزایش عملکرد: با استفاده از مقیاس‌پذیری افقی و تقسیم بار کاری، سیستم قادر است بهبود عملکرد و زمان پاسخ داده‌ها را ارائه دهد.

    \end{itemize}

    با این حال، برخی از معایب مقیاس‌پذیری افقی شامل پیچیدگی مدیریت، هزینه‌های بالای افزایش تعداد سرورها و node ها، و مشکلات مربوط به هماهنگی و هماهنگی داده‌ها است. برای حل این مشکلات، از تکنولوژی‌ها و راهکارهایی مانند اشتراک گذاری داده، پارتیشن‌بندی و استفاده از فن‌آوری‌های متمرکز بر مقیاس پذیری افقی مانند ابزارهای مدیریت مقیاس‌پذیری استفاده می‌شود.

    برای افزایش مقیاس پایگاه‌داده‌های رابطه‌ای Oracle به صورت افقی، چندین روش و راهکار وجود دارد. این روش‌ها عبارتند از:

    \begin{itemize}

        \item اشتراک‌گذاری داده Sharding) :(Data در این روش، داده‌های پایگاه داده به بخش‌های کوچک‌تر تقسیم می‌شوند و هر بخش را بر روی یک سرور مستقل نگهداری می‌کنند. این امکان را به سیستم می‌دهد که بتواند بیشترین بار ممکن را به طور موازی اجرا کند.

        \item پارتیشن‌بندی (Partitioning): با استفاده از این روش، داده‌ها بر اساس یک معیار مشخص مانند مقادیری از یک ستون معین به بخش‌های مجزا تقسیم می‌شوند. این کار اجازه می‌دهد تا کوئری‌ها فقط بر روی بخش‌های مشخصی از داده‌ها اعمال شود، که بهبود عملکرد را ایجاد می‌کند.

        \item توسعه سریع برنامه Development) Application :(Rapid با استفاده از این روش، برنامه‌هایی طراحی می‌شوند که به طور خودکار به سیستم افزایش پذیری افقی مقیاس پذیری Oracle متصل می‌شوند، بدون نیاز به تغییرات سرور یا دیتابیس.

        \item استفاده از پایگاه داده‌های هوشمند Databases) :(Smart این پایگاه داده‌ها خودکار می‌توانند به تغییرات در بار و ترافیک واکنش نشان دهند و به طور اتوماتیک منابع مورد نیاز را برای مقیاس پذیری افقی تخصیص دهند.

    \end{itemize}

    هر یک از این روش‌ها مزایا و معایب خود را دارند، بنابراین انتخاب بهترین روش برای معماری سیستم مقیاس بزرگ بستگی به نیازها و شرایط خاص سیستم دارد.

    \noindent\hrule

    \paragraph{سوال:} چه ابزارهایی وجود دارد برای این کار و از چه معماری‌هایی استفاده می‌شود؟

    \paragraph{پاسخ:} برای پیاده‌سازی مقیاس‌پذیری افقی در پایگاه‌داده‌های رابطه‌ای مانند Oracle، معمولاً از ابزارها و معماری‌هایی استفاده می‌شود که امکان تقسیم بار کاری و مدیریت چندین سرور یا node را فراهم می‌کنند. برخی از این ابزارها و معماری‌ها عبارتند از:

    \begin{itemize}

        \item Clusters Application Real Oracle :(RAC) RAC یک معماری برای پایگاه‌داده‌های Oracle است که امکان تجمیع چندین سرور به عنوان یک سیستم بزرگ واحد را فراهم می‌کند. این معماری به اطمینان از دسترسی پیوسته و افزایش عملکرد پایگاه داده کمک می‌کند.

        \item Guard :Data Guard Data Oracle یک راهکار برای مدیریت و کنترل تطابق داده و برنامه‌ریزی برای فروشنده/پذیرنده (Primary/Standby) در معماری RAC یا بر روی سرورهای مستقل است. این امکان را فراهم می‌کند تا در صورت خرابی یک سرور، سرور دیگری به طور خودکار به عنوان فروشنده (Primary) فعال شود.

        \item Partitioning :Oracle این ویژگی به اطمینان از این که داده‌ها به صورت موزون و منظم در سرورها یا node ها پخش شده و بهینه‌سازی برای اجرای کوئری‌ها ارائه می‌دهد.

        \item :Sharding Sharding Oracle یک راهکار برای تقسیم‌بندی داده‌ها به صورت افقی و جدا کردن آن‌ها بین چندین سرور یا node است. این امکان را فراهم می‌کند تا بتوان داده‌ها را به صورت موازی پردازش کرد و بار کاری را بین سرورها تقسیم کرد.
        
        \item Management Storage Automatic Oracle :(ASM) ASM یک راهکار برای مدیریت فضای ذخیره‌سازی در پایگاه‌داده‌های Oracle است که به صورت خودکار می‌تواند داده‌ها را بین چندین دیسک جدا کرده و بهینه‌سازی کند.

    \end{itemize}

    این ابزارها و معماری‌ها توسط Oracle برای ایجاد محیط‌های مقیاس‌پذیری افقی در پایگاه‌های داده Oracle ارائه شده‌اند و به شرکت‌ها کمک می‌کنند تا بتوانند نیازهای مختلف عملیاتی خود را برآورده کنند و بهبود عملکرد و قابلیت اطمینان را ارتقا دهند.

\end{document}