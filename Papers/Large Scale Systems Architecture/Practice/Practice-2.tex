\documentclass[a4paper,10pt]{article}

\usepackage{xcolor}
\usepackage{tcolorbox}

\usepackage{geometry}
\geometry{
    left=1cm,
    right=1cm,
    top=2cm,
    bottom=2cm
}

\usepackage{booktabs}
\usepackage{tabularx}

\usepackage{fancyhdr}
\pagestyle{fancy}
\fancyhf{}
\fancyhead[R]{معماری سیستم‌های مقیاس بزرگ}
\fancyhead[C]{تمرین 2}
\fancyhead[L]{\today}
\fancyfoot[R]{محمد خورشیدی روزبهانی - 40215741002013}
\fancyfoot[L]{\thepage}

\usepackage{xepersian}
\settextfont{Vazirmatn-Regular.ttf}

\newtcolorbox{conclusion}[1][]{
    colback=black!5,
    colframe=black!30,
    title=جمع‌بندی
}

\linespread{1.5}

\begin{document}

    \paragraph{سوال:} سرویس‌دهنده وب Nginx چه وظایفی را برای اپلیکیشن‌های وب انجام می‌دهد؟ (به عبارت دیگر چه task Main هایی دارد؟)

    \paragraph{پاسخ:} Nginx یک سرور وب قدرتمند است که برای پردازش و ارسال درخواست‌های وب استفاده می‌شود. این سرور به صورت موازی و با کارایی بالا، درخواست‌های HTTP و HTTPS را پردازش می‌کند. یکی از ویژگی‌های برجسته آن توانایی مدیریت توزیع بار است که به انتقال بهترین درخواست‌ها به سرورهای پشتیبان کمک می‌کند، این باعث بهبود کارایی و اعتبار سایت می‌شود. همچنین، Nginx قابلیت ارائه خدمات امنیتی مانند فشرده‌سازی محتوا و پشتیبانی از SSL را نیز دارد. با این ویژگی‌ها، Nginx یکی از سرورهای وب محبوب برای برنامه‌های وب و سرویس‌های آنلاین است.
    
    Nginx به عنوان یک سرور وب، وظایف متعددی انجام می‌دهد از جمله:

    \begin{enumerate}
        
        \item سرویس دهی به درخواست‌های HTTP و HTTPS از مشتریان.

        \item ارائه خدمات مانند توزیع بار Balancing) (Load بین سرورها.

        \item ارائه عملکردهای مانند فشرده‌سازی محتوا Compression) (Content برای بهبود کارایی.

        \item اجرای Proxy Reverse برای پنهان کردن و ایجاد واسط بین کلاینت و سرورهای پشتیبان.

        \item مدیریت و کنترل دسترسی به منابع مانند فایل‌ها، پوشه‌ها و دسترسی به API‌ها.

        \item پشتیبانی از TLS/SSL برای ارتباطات امن بین کلاینت و سرور.

        \item مدیریت قابلیت انعطاف‌پذیری برای پیکربندی و سفارشی‌سازی تنظیمات سرور.

    \end{enumerate}

    \noindent\hrulefill

    \paragraph{سوال:} تفاوت سرورهای Nginx و Server Apache چه مواردی است؟

    \paragraph{پاسخ:} سرورهای Nginx و Server Apache از ابعاد مختلفی می‌توانند بررسی شوند و به وجوه تمایز آن‌ها از یکدیگر پرداخت. تفاوت‌های اصلی بین Nginx و Server Apache عبارتند از:


    \begin{enumerate}
        
        \item معماری: Nginx بر مبنای رویداد (event-driven) عمل می‌کند که به طور موازی و غیربلاکی (non-blocking) درخواست‌ها را پردازش می‌کند. در مقابل، Server Apache بر مبنای نخی (thread-based) عمل می‌کند که هر نخ یک درخواست را پردازش می‌کند.

        \item مصرف حافظه و منابع: Nginx به طور کلی حافظه کمتری نیاز دارد و با تعداد کمتری منبع می‌تواند تعداد بیشتری درخواست را پردازش کند. این امر به کارایی و عملکرد بهتر Nginx کمک می‌کند.

        \item امنیت: به طور کلی، Nginx به عنوان یک سرور با امنیت بالا شناخته می‌شود. از آنجایی که درخواست‌ها به صورت non-blocking پردازش می‌شوند، کمترین امکان برای حملات از نوع Service of Denial (DoS) وجود دارد. همچنین، Nginx با امکانات امنیتی مانند Rate Limiting و Module امنیتی ModSecurity همراه است.

        \item قابلیت‌های توسعه: Server Apache به عنوان یک سرور وب با تاریخچه‌ی بلند، از اکوسیستمی گسترده از ماژول‌ها و افزونه‌ها برخوردار است که قابلیت گسترش و سفارشی‌سازی بیشتری را فراهم می‌کند. از سوی دیگر، Nginx همچنین افزونه‌هایی دارد اما کمتر از .Apache

        \item کارایی: Nginx به طور کلی بهترین کارایی را در بارگذاری و پردازش درخواست‌های استاتیک و با تعداد بالا ارائه می‌دهد. در حالی که Apache بیشترین کارایی را در پردازش درخواست‌های پویا (مانند PHP و (CGI دارد.

    \end{enumerate}

    \newpage

    \paragraph{سوال:} چطور Nginx درخواست‌ها (requests) را پردازش می‌کند؟

    \paragraph{پاسخ:} Nginx به صورت موازی و غیربلاکی (non-blocking) درخواست‌ها را پردازش می‌کند. این به این معنی است که به‌جای ایجاد یک نخ جدید برای هر درخواست، از رویدادها (events) استفاده می‌کند تا به طور همزمان درخواست‌ها را پردازش کند. زمانی که یک درخواست دریافت می‌شود، Nginx آن را به یک process worker (پردازش‌کننده کارگر) منتقل می‌کند که در مدل event-driven عمل می‌کند. سپس در این process ،worker به‌طور موازی و همزمان، درخواست‌ها پردازش می‌شوند و نتیجه به مشتری ارسال می‌شود. به این ترتیب، Nginx با بهره‌گیری از مدل event-driven و non-blocking، توانایی پردازش بالا و کارایی بهتری را نسبت به سرورهای وب سنتی دارد.

    \vspace{15pt}

    \begin{conclusion}

        به طور خلاصه، Nginx یک سرور وب و پروکسی معکوس با عملکرد بالا، کارایی عالی، و امنیت قوی است. از ویژگی‌های برجسته آن می‌توان به معماری موازی و ،non-blocking توانایی مدیریت توزیع بار، فشرده‌سازی محتوا، امنیت ارتباطات و امکانات گسترده برای سفارشی‌سازی اشاره کرد. با این ویژگی‌ها، Nginx به عنوان یکی از سرورهای وب محبوب و پرکاربرد در صنعت اینترنت و تکنولوژی اطلاعات شناخته می‌شود.

    \end{conclusion}
    
\end{document}