\documentclass[a4paper,10pt]{article}

\usepackage{geometry}
\geometry{
    left=1cm,
    right=1cm,
    top=2cm,
    bottom=2cm
}


\usepackage{fancyhdr}
\pagestyle{fancy}
\fancyhf{}
\fancyhead[R]{معماری سیستم‌های مقیاس بزرگ}
\fancyhead[C]{تمرین 1}
\fancyhead[L]{\today}
\fancyfoot[R]{محمد خورشیدی روزبهانی - 40215741002013}
\fancyfoot[L]{\thepage}


\usepackage{xepersian}
\settextfont{Vazirmatn-Regular.ttf}

\linespread{1.5}

\begin{document}

    \paragraph{سوال:} چارچوب gRPC به چه صورت است و چه مزایایی نسبت به API Rest دارد؟

    \paragraph{پاسخ:} gRPC و Rest API هر دو روش‌هایی برای ارتباط بین سرویس‌ها در برنامه‌نویسی وب هستند، اما با رویکردهای مختلفی به این کار می‌پردازند.

    gRPC یک فرمت مبتنی بر RPC Call) Procedure (Remote است که برای تبادل داده بین سرویس‌ها استفاده می‌شود. این فرمت از protobuf Buffers) (Protocol به عنوان فرمت پیش‌فرض برای توصیف داده‌ها استفاده می‌کند که به صورت باینری و کم حجم است. gRPC از HTTP/2 برای انتقال داده استفاده می‌کند که باعث بهبود کارایی و کاربردی‌تر شدن ارتباطات میان سرویس‌ها می‌شود. همچنین gRPC از ویژگی‌هایی مانند توزیع کد و آپشن‌های امنیتی پیشرفته پشتیبانی می‌کند.
    
    API Rest به عنوان یک روش معمول برای تبادل داده در اینترنت مورد استفاده قرار می‌گیرد. در این روش، منابع (معمولاً با استفاده از (URL توسط عملیات HTTP مختلفی مانند ،GET ،POST PUT و DELETE مورد دسترسی قرار می‌گیرند. در مقابل، API Rest از فرمت‌هایی مانند JSON یا XML برای نمایش داده‌ها استفاده می‌کند که برای برنامه‌نویسان خوانایی بیشتری دارند.
    
    از آنجا که gRPC و API Rest هر دو روش‌های معتبر برای ارتباط بین سرویس‌ها هستند، انتخاب بین آن‌ها بسته به نیازهای پروژه و محیط برنامه‌نویسی مختلف است. در مواردی که نیاز به کارایی بالا و کاربردی‌ترین روش ارتباط بین سرویس‌ها وجود دارد، gRPC ممکن است انتخاب مناسبی باشد. اما در موارد دیگر که نیاز به سادگی و قابلیت استفاده از هر زبان برنامه‌نویسی و فرمت داده‌ای معمولی‌تر است، API Rest می‌تواند گزینه مناسب‌تری باشد.

    \noindent\hrulefill

    \paragraph{سوال:} در gRPC از چه پروتوکلی برای انتقال داده استفاده می‌شود؟

    \paragraph{پاسخ:} پروتکل HTTP/2، یک پروتکل ارتباطی است که برای ارتباط بین کلاینت و سرور در gRPC استفاده می‌شود. این پروتکل در اساس یک نسخه بهبود یافته از HTTP/1.1 است که طراحی شده است تا کارایی و کاربردی بودن ارتباطات وب را افزایش دهد. در ،gRPC از HTTP/2 به عنوان پروتکل انتقال داده استفاده می‌شود به دلیل ویژگی‌های زیر:

    \begin{itemize}
        
        \item موازی و دوطرفه بودن: HTTP/2 امکان ارسال و دریافت داده‌ها به صورت موازی و دوطرفه را فراهم می‌کند، که بهبود کارایی و پاسخگویی سریع‌تر را در مقایسه با HTTP/1.1 ایجاد می‌کند.

        \item فشرده‌سازی هدرها: HTTP/2 از فشرده‌سازی هدرها استفاده می‌کند که حجم داده‌های ارسالی را کاهش می‌دهد و پهنای باند را بهبود می‌بخشد.

        \item مدیریت جریان: HTTP/2 امکان مدیریت بهتر جریان داده‌ها را فراهم می‌کند که بهبود کارایی ارتباطات را ایجاد می‌کند و از تبادل داده‌های بی‌اعتبار جلوگیری می‌کند.

        \item توانایی ارسال همزمان چندین درخواست و پاسخ: HTTP/2 امکان ارسال همزمان چندین درخواست و پاسخ را بر روی یک اتصال TCP ایجاد می‌کند، که بهبود عملکرد و کارایی ارتباطات را فراهم می‌کند.

    \end{itemize}

    به طور کلی، استفاده از HTTP/2 در gRPC بهبود کارایی، پایداری و امنیت ارتباطات را بهبود می‌بخشد و تجربه استفاده از خدمات وب را بهبود می‌دهد.

    \noindent\hrulefill

    \paragraph{سوال:} تفاوت پروتوکل gRPC و فرمت JSON در چه مواردی است؟

    \paragraph{پاسخ:} gRPC و JSON هر دو ابزاری هستند که برای ارتباطات بین سیستم‌های مختلف در برنامه‌نویسی استفاده می‌شوند، اما دارای ویژگی‌ها و موارد استفاده متفاوتی هستند. در ادامه تفاوت‌های اصلی بین gRPC و JSON را بررسی می‌کنیم:

    

\end{document}