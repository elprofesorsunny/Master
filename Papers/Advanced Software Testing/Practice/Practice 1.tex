\documentclass[a4paper,10pt]{article}

\usepackage{geometry}
\geometry{top=2cm, left=1.5cm, right=1.5cm, bottom=2cm}

\usepackage{fancyhdr}
\pagestyle{fancy}
\fancyhf{}
\fancyhead[R]{آزمون نرم‌افزار پیشرفته}
\fancyhead[C]{تمرین 1}
\fancyhead[L]{\today}
\fancyfoot[R]{محمد خورشیدی روزبهانی - 40215741002013}
\fancyfoot[L]{\thepage}

\usepackage{xepersian}
\settextfont{Vazirmatn-Regular.ttf}

\linespread{1.5}

\begin{document}

    \paragraph{سوال:} تفاوت بین Failure و Fault را در حوزه آزمون نرم‌افزار بیان کنید.

    \paragraph{پاسخ:} در حوزه آزمون نرم‌افزار، Fault و Failure دو مفهوم کلیدی هستند که با وجود ارتباط نزدیکی که با یکدیگر دارند، تفاوت‌های مهمی دارند.

    \begin{itemize}
        
        \item تعریف Fault (خطا یا نقص)
        
        Fault که به آن Bug یا Defect نیز گفته می‌شود، به هرگونه اشتباه یا خطایی گفته می‌شود که در کد منبع نرم‌افزار وجود دارد. این اشتباه می‌تواند در نتیجه‌ اشتباهات برنامه‌نویسی، طراحی، یا حتی تحلیل نیازمندی‌ها به وجود آمده باشد. به بیان ساده، Fault به آن بخشی از نرم‌افزار اشاره دارد که از حالت صحیح انحراف داشته و ممکن است باعث بروز مشکلات در عملکرد نرم‌افزار شود.

        مثال: فرض کنید که در یک برنامه‌ بانکی، یک خطا در کدنویسی باعث شود که محاسبات مربوط به سود بانکی به درستی انجام نشود. این خطا (Fault) ممکن است بلافاصله به‌ عنوان یک Failure قابل مشاهده نباشد اما وجود دارد و ممکن است در آینده مشکل ایجاد کند.

        \item تعریف Failure (شکست)
        
        Failure به شرایطی گفته می‌شود که نرم‌افزار نمی‌تواند عملکرد صحیح خود را انجام دهد و خروجی نادرست یا رفتاری غیرمنتظره از خود نشان می‌دهد. زمانی که یک Fault در زمان اجرای نرم‌افزار باعث شود که نرم‌افزار عملکرد خود را به‌درستی انجام ندهد، این حالت را Failure می‌نامیم. بنابراین، Failure نتیجه‌ نهایی وجود یک یا چند Fault در نرم‌افزار است که به صورت مشکل در کارکرد سیستم ظاهر می‌شود.

        مثال: همان خطای بانکی که در بخش قبلی مطرح شد، اگر هنگام محاسبه‌ سود به کاربر نتیجه‌ی نادرست نمایش دهد، این رفتار نادرست نرم‌افزار را Failure می‌نامیم. بنابراین، وقتی کاربر متوجه خروجی اشتباه می‌شود، در واقع Failure رخ داده است.

        \item ارتباط بین Fault و Failure

        ارتباط بین این دو مفهوم به این شکل است که هر Fault لزوماً منجر به Failure نمی‌شود. برخی از Fault ها ممکن است هرگز به کاربر نشان داده نشوند یا در شرایط خاصی رخ دهند که تشخیص داده نشوند. همچنین، ممکن است Fault هایی وجود داشته باشند که تنها تحت شرایط خاصی (مانند یک ورودی غیرمنتظره یا یک تعامل نادر با سیستم) به یک Failure تبدیل شوند. این تفاوت می‌تواند به دلیل پیچیدگی سیستم‌های نرم‌افزاری و حالت‌های مختلفی که سیستم ممکن است در آن اجرا شود باشد.

        \item مفهوم Latent Fault (نقص نهفته)
        
        گاهی اوقات، Fault ها در نرم‌افزار پنهان می‌مانند و بلافاصله باعث Failure نمی‌شوند. این نوع از Fault ها را Latent Fault می‌نامند. این نقص‌ها ممکن است تنها تحت شرایط خاصی آشکار شوند که ممکن است در مراحل ابتدایی تست یا حتی تا مدت‌ها پس از استفاده از نرم‌افزار نیز قابل شناسایی نباشند.

        \item مفهوم Error (خطا)

        برای درک بهتر این موضوع، باید به مفهوم دیگری به نام Error نیز اشاره کنیم. Error معمولاً به معنای اشتباهی است که توسط توسعه‌دهنده یا تحلیل‌گر در طول فرآیند توسعه ایجاد شده و منجر به ایجاد یک Fault می‌شود. بنابراین، Error علت است و Fault نتیجه.

        \item مثالی برای درک تفاوت Fault و Failure

        فرض کنید شما در حال استفاده از یک نرم‌افزار حسابداری هستید. اگر کد برنامه به دلیل یک اشتباه برنامه‌نویسی طوری نوشته شده باشد که به جای جمع دو عدد، آن‌ها را ضرب کند، این یک Fault است. اما اگر این قسمت از کد هرگز توسط کاربر استفاده نشود یا ورودی‌هایی که منجر به بروز خطا می‌شوند هرگز ارائه نشوند، هیچ Failure ای رخ نمی‌دهد. در عوض، اگر کاربر ورودی‌های مربوطه را وارد کند و نرم‌افزار خروجی نادرست تولید کند، یک Failure اتفاق افتاده است.

        \item پیشگیری و شناسایی

        تیم‌های توسعه‌ی نرم‌افزار باید به‌طور مداوم به دنبال یافتن و اصلاح Fault ها باشند تا از وقوع Failure ها جلوگیری کنند. آزمون‌های نرم‌افزاری به این منظور طراحی می‌شوند که Fault ها را شناسایی کنند تا پیش از آنکه به یک Failure منجر شوند، برطرف شوند. با این حال، در پروژه‌های بزرگ و پیچیده، امکان شناسایی تمامی Fault ها بسیار سخت است.

    \end{itemize}

\end{document}