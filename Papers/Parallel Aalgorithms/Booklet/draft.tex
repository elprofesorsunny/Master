\documentclass[a4paper,10pt]{article}

\usepackage{geometry}
\geometry{top=2cm,right=1.5cm,bottom=2cm,left=1.5cm}

\usepackage{xepersian}
\settextfont{Vazirmatn-Regular.ttf}

\linespread{1.5}

\begin{document}

    برای اجرای دستورات روی پردازنده‌های موازی نیازمند الگوریتم‌های موازی هستیم.

    چند مدل به عنوان دسته بندی برای الگوریتم‌های موازی و موازی‌سازی داریم که به شرح زیر است:

    \begin{itemize}
        
        \item یک دستورالعمل، یک داده (SISD): 
        \item چند دستورالعمل، یک داده (MISD):
        \item یک دستورالعمل، چند داده (SIMD): در این حالت زمان اجرا سنکرون و همگام است و علت این کار این است که چون یک دستورالعمل داریم پس همه پردازش‌ها با همدیگر تمام می‌شوند.
        
        \begin{itemize}
            
            \item shared-memory: اگر در این حال پردازنده A بخواهد داده‌ای را به پردازنده B ارسال کند، آن داده را در یک فضای اشتراکی می‌گذارند. بر اساس نحوه قرارگیری داده در حافظه، 4 دسته‌بندی مطرح می‌شود:
            
            \begin{itemize}
                
                \item EREW
                \item CREW
                \item ERCW
                \item CRCW: در این مورد چون خواندن و نوشتن به صورت همزمان در حال انجام است، پس تضاد یا Conflict ایجاد می‌شود و برای این موضوع سیاست‌هایی در نظر گرفته شده است که در مقالات متفاوتی به آن‌ها پرداخته شده است.

            \end{itemize}

            \item interconnection network
            
            \begin{itemize}
                
                \item Dynamic
                \item Static 

            \end{itemize}

        \end{itemize}
        

        \item چند دستورالعمل، چند داده (MIMD):
        
    \end{itemize}



\end{document}