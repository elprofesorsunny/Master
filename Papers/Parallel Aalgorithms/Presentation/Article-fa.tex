\documentclass[a4paper,10pt]{article}

\usepackage{dblfnote}
\usepackage{lipsum}

\usepackage{geometry}
\geometry{top=2cm, bottom=2cm, left=1.5cm, right=1.5cm}

\usepackage{xepersian}
\settextfont{Vazirmatn-Regular.ttf}

\title{\huge بهینه‌سازی کلونی مورچه‌ها به‌صورت موازی بر روی پردازنده‌های چند هسته‌ای SIMD}
\author{محمد خورشیدی روزبهانی\\40215741002013 \and شارا شاهوردیان\\40215741002032}
\date{}

\linespread{1.5}

\begin{document}

    \maketitle

    % \vspace{5pt}

    \begin{abstract}
        
        بهینه‌سازی کلونی مورچه‌ها\footnote{\hspace{2pt}Optimization Colony Ant - ACO} یک فراابتکاری\footnote{\hspace{2pt}Metaheuristics} مبتنی بر جمعیت برای حل مسائل سخت بهینه‌سازی ترکیبی است. مطالعات بسیاری به تسریع بهینه‌سازی کلونی مورچه‌ها با استفاده از سخت‌افزارهای موازی، به‌ویژه واحدهای پردازش گرافیکی\footnote{\hspace{2pt}Unit Processing Graphics - GPU}، اختصاص یافته است. با این حال، به دلیل الگوی نامنظم (تصادفی) در دسترسی به داده‌ها و جریان کنترل در الگوریتم‌های بهینه‌سازی کلونی مورچه‌ها، عملکرد روش‌های مبتنی بر واحدهای پردازش گرافیکی به دلیل محدودیت‌های سخت‌افزاری محدود می‌شود. پردازش مبتنی بر SIMD در پردازنده‌های برای بهینه‌سازی کلونی مورچه‌ها در ادبیات گذشته به ندرت مورد بررسی قرار گرفته و همچنان ناشناخته است که الگوریتم‌های بهینه‌سازی کلونی مورچه‌ها مبتنی بر پردازنده‌های چند هسته‌ای SIMD تا چه حد می‌توانند عملکرد داشته باشند. در این مقاله، ما مدلی از بهینه‌سازی کلونی مورچه‌ها موازی برداری برای معماری پردازنده‌های چند هسته‌ای SIMD ارائه و ارزیابی می‌کنیم. در مدل پیشنهادی، هر مورچه با یک هسته پردازنده مرتبط شده و ساخت مسیر هر مورچه با استفاده از دستورات برداری تسریع می‌شود. علاوه بر این، بر اساس مدل پیشنهادی، یک روش جدید انتخاب متناسب با تناسب به نام چرخ رولت مبتنی بر بردار\footnote{\hspace{2pt}Wheel Roulette Vector-based - VRW} در مرحله ساخت مسیر معرفی می‌شود. در این روش، مقادیر تناسب در خطوط SIMD گروه‌بندی شده و مجموع پیشوند به‌صورت موازی برداری محاسبه می‌شود. الگوریتم پیشنهادی بر روی نمونه‌های استاندارد مسئله فروشنده دوره‌گرد\footnote{\hspace{2pt}Problem Salesman Travelling The - TSP} با ابعاد 198 تا 4461 شهر آزمایش شده و نشان‌دهنده سرعت‌افزایی تا 57.8 برابر نسبت به نسخه تک‌ریسمانی پردازنده است. به‌طور قابل توجهی، ما رویکرد خود را با بهینه‌سازی‌های کلونی مورچه‌ها مبتنی بر واحدهای پردازش گرافیکی با کارایی بالا مقایسه کرده‌ایم و نتایج پتانسیل قوی بهینه‌سازی‌های کلونی مورچه‌ها موازی مبتنی بر پردازنده را نشان می‌دهد.

    \end{abstract}

    % \vspace{15pt}

    \section{مقدمه}

        حل مسائل بهینه‌سازی در دنیای واقعی پیچیده و زمان‌بر است، به‌ویژه برای پردازنده‌های مرکزی در مسائل بزرگ‌مقیاس. مدل‌سازی این مسائل نیز وابسته به نوع مسئله است. فراابتکارها روش‌های کارآمدی هستند که امکان دستیابی به یک حل رضایت‌بخش (یک بهینه تقریبی) را در زمانی معقول فراهم می‌کنند و چارچوبی الگوریتمی عمومی ارائه می‌دهند که می‌توان آن را با تغییرات اندک در مسائل مختلف به کار برد. بهینه‌سازی کلونی مورچه‌ها یک فراابتکار مبتنی بر جمعیت است که از رفتار اجتماعی مورچه‌ها الهام گرفته است. مورچه‌های مصنوعی به‌طور مستقل راه‌حل‌هایی را می‌سازند و از طریق مکانیزم استیگمرژی\footnote{\hspace{2pt}Stigmergy} با یکدیگر ارتباط برقرار می‌کنند. این فرآیند به‌صورت تکراری تا زمانی که معیار توقف حاصل شود، اجرا می‌شود. مهم‌ترین ویژگی بهینه‌سازی کلونی مورچه‌ها بازخورد مثبت است که از طریق فرومون‌های به‌جا مانده توسط مورچه‌ها، می‌تواند فرآیند ساخت راه‌حل را هدایت کند. این فراابتکار برای حل مسائل NP-سخت مانند مسئله فروشنده دوره‌گرد و مسئله تخصیص درجه دوم\footnote{\hspace{2pt}Problem Assignment Quadratic - QAP} مورد استفاده قرار گرفته است.

        در سال‌های اخیر، انواع جدیدی از سخت‌افزار که قدرت پردازشی موازی گسترده‌ای را ارائه می‌دهند، مانند پردازنده سلولی Cell/BE و FPGA واحدهای پردازش گرافیکی، در دسترس قرار گرفته‌اند. در میان این سخت‌افزارها، پردازنده‌های گرافیکی یکی از چشمگیرترین شتاب‌دهنده‌ها هستند که می‌توانند قدرت محاسباتی بالایی را با هزینه کم ارائه دهند. الگوریتم‌های بهینه‌سازی کلونی مورچه‌ها موازی شتاب‌گرفته توسط پردازنده‌های گرافیکی به موضوعی بسیار مورد توجه تبدیل شده‌اند. کارایی محاسباتی الگوریتم‌های بهینه‌سازی کلونی مورچه‌ها به‌طور قابل‌توجهی با استفاده از پردازنده‌های گرافیکی بهبود یافته است.

        با این حال، بهینه‌سازی کلونی مورچه‌ها به‌صورت موازی مبتنی بر پردازنده‌های گرافیکی با حداقل دو مشکل مواجه است. اولاً، بهینه‌سازی کلونی مورچه‌ها مبتنی بر پردازنده‌های گرافیکی در مقایسه با بهینه‌سازی کلونی مورچه‌ها ترتیبی مبتنی بر پردازنده مرکزی تنها بهبودهای سرعتی متوسطی را نشان می‌دهد، در حالی که عملکرد نظری اوج پردازنده‌های گرافیکی به‌طور قابل‌توجهی بالاتر از یک ریسمان واحد پردازنده مرکزی است. ثانیاً، بهینه‌سازی کلونی مورچه‌ها مبتنی بر پردازنده‌های گرافیکی به شدت به حافظه روی تراشه وابسته است که یک منبع بسیار محدود است، و در نتیجه اندازه مسئله‌ای که می‌تواند حل شود محدود می‌شود.

        معماری‌های جدید پردازنده‌های مرکزی تغییرات قابل توجهی داشته‌اند که منجر به قابلیت‌های محاسباتی با کارایی بالا شده‌اند. اولاً، به دلیل محدودیت‌های موازی‌سازی در سطح دستورالعمل\footnote{\hspace{2pt}Parallelism Level Instruction - ILP} و محدودیت‌های فرکانسی، عملکرد پردازنده‌های مرکزی از طریق طرح چند هسته‌ای افزایش یافته است. ثانیاً، معماری‌های مدرن پردازنده‌های مرکزی فشار بیشتری بر روی توسعه دستورات SIMD وارد می‌کنند. الگوریتم‌های منظم از معماری فعلی پردازنده‌های مرکزی بهره برده‌اند، مانند مرتب‌سازی، محاسبات شابلونی و ضرب ماتریس.

        متأسفانه توجه کمی به مدل‌سازی بهینه‌سازی کلونی مورچه‌ها به‌صورت موازی برای سازگاری با معماری پردازنده‌های مرکزی مبتنی بر SIMD شده است. اگرچه مدل موجود بهینه‌سازی کلونی مورچه‌ها مبتنی بر پردازنده‌های گرافیکی ممکن است به‌طور نظری در معماری پردازنده‌های چند هسته‌ای مبتنی بر SIMD قابل اعمال باشد، اما عملکرد آن به‌دلیل ویژگی‌های متفاوت معماری‌های موازی، ویژگی‌های محاسباتی الگوریتم بهینه‌سازی کلونی مورچه‌ها و تفاوت‌های بهینه‌سازی کامپایلرها به‌سختی تضمین می‌شود. پیاده‌سازی بهینه‌شده نسخه بهینه‌سازی کلونی مورچه‌ها مبتنی بر پردازنده‌های چند هسته‌ای SIMD همچنان یک چالش فنی محسوب می‌شود.

        در کنفرانس شفاهی ،META’14 رویکردی برای برداری‌سازی الگوریتم بهینه‌سازی کلونی مورچه‌ها در معماری پردازنده‌های گرافیکی پیشنهاد شد. اما اینکه آیا این رویکرد برای معماری پردازنده‌های چند هسته‌ای مبتنی بر SIMD مناسب است یا خیر، همچنان نامشخص است و اینکه بهینه‌سازی کلونی مورچه‌ها به‌صورت موازی مبتنی بر پردازنده‌های مرکزی چگونه در مقایسه با روش‌های مبتنی بر پردازنده‌های گرافیکی عمل می‌کنند نیز هنوز روشن نشده است.

        این مقاله ایده‌ای برای تسریع بهینه‌سازی کلونی مورچه‌ها به‌صورت موازی و کاملاً توسعه‌یافته برای مسئله فروشنده دوره‌گرد بر روی پردازنده‌های چند هسته‌ای SIMD ارائه می‌دهد. در الگوریتم ما، مراحل بحرانی، ساخت مسیر و به‌روزرسانی فرومون، موازی‌سازی و بهینه‌سازی شده‌اند. دستاوردهای اصلی ما به شرح زیر هستند:

        \begin{enumerate}

            \item تا جایی که می‌دانیم، این اولین الگوریتم بهینه‌سازی کلونی مورچه‌ها به‌صورت موازی است که از هر دو نوع موازی‌سازی در سطح کار\footnote{\hspace{2pt}task-level} و سطح برداری\footnote{\hspace{2pt}vector-level} در پردازنده‌های چند هسته‌ای مبتنی بر SIMD بهره می‌برد.

            \item ما یک مدل جدید از بهینه‌سازی کلونی مورچه‌ها به‌صورت برداری موازی بر روی پردازنده‌های چند هسته‌ای SIMD ارائه می‌دهیم که مدل عمومی موازی‌سازی کار که به‌طور گسترده استفاده می‌شود را توسعه می‌دهد.

            \item در مرحله ساخت مسیر، یک روش جدید انتخاب متناسب با تناسب به نام چرخ رولت مبتنی بر بردار طراحی می‌کنیم تا موازی‌سازی در سطح بردار بر روی پردازنده‌های چند هسته‌ای SIMD بهبود یابد.

            \item الگوریتم خود را با استفاده از مسائل استاندارد TSPLIB در بازه‌ای از 198 تا 4461 شهر ارزیابی می‌کنیم و به یک فاکتور سرعت‌افزایی حداکثر 57.8 برابر در مقایسه با نسخه ترتیبی استاندارد پردازنده مرکزی دست می‌یابیم.

            \item از همه مهم‌تر، الگوریتم خود را با داده‌های قبلی بهینه‌سازی کلونی مورچه‌ها به‌صورت موازی مبتنی بر پردازنده‌های گرافیکی که با کارایی بالا در مقالات گذشته پیشنهاد شده‌اند، مقایسه می‌کنیم. نتایج نشان‌دهنده پتانسیل قوی بهینه‌سازی کلونی مورچه‌ها به‌صورت موازی مبتنی بر پردازنده‌های چند هسته‌ای SIMD است.

        \end{enumerate}

        باقی‌مانده این مقاله به شرح زیر سازماندهی شده است. ابتدا، بهینه‌سازی کلونی مورچه‌ها برای مسئله فروشنده دوره‌گرد را به‌طور مختصر معرفی می‌کنیم. مطالعات مرتبط در بخش 2 مورد بحث قرار می‌گیرند. سپس، رویکردهای ما برای بهینه‌سازی کلونی مورچه‌ها به‌صورت موازی بر روی پردازنده‌های چند هسته‌ای SIMD را در بخش 3 ارائه می‌دهیم. روش‌شناسی آزمایش‌های ما در بخش 4 توضیح داده شده و ارزیابی عملکرد الگوریتم ما در این بخش مورد بحث قرار می‌گیرد. در نهایت، یافته‌های خود را جمع‌بندی کرده و پیشنهاداتی برای کارهای آینده ارائه می‌دهیم.

    \section{پیش‌زمینه}

        \subsection{بهینه‌سازی کلونی مورچه‌ها برای مسئله فروشنده دوره‌گرد}

            مسئله فروشنده دوره‌گرد یک مسئله NP-سخت در بهینه‌سازی ترکیبی است و در پژوهش‌های عملیاتی و علوم کامپیوتر نظری اهمیت دارد. هدف مسئله فروشنده دوره‌گرد پیدا کردن یک چرخه همیلتونی با کمترین وزن در یک گراف جهت‌دار و وزندار کامل است، که در آن $G=(V,A,d)$ و $V=1,2,...,n$ مجموعه‌ای از رأس‌ها (شهرها)، $A=\{(i,j)|(i,j)\in V\times V\}$ مجموعه‌ای از یال‌ها (مسیرها) و $d:A\rightarrow N$ تابعی است که به هر یال $(i, j)$ یک وزن یا فاصله (عدد صحیح مثبت) $d_{ij}$ را اختصاص می‌دهد.

            دوریگو و همکارانش این مسئله را با استفاده از بهینه‌سازی کلونی مورچه‌ها حل کرده‌اند، که از تعداد زیادی مورچه مصنوعی استفاده می‌کند که جستجوهای موازی روی یک گراف انجام می‌دهند. هر مورچه به‌طور مستقل بر روی گراف حرکت می‌کند تا زمانی که به تمام شهرهای گراف سفر کند. این فرایند معمولاً به مرحله ساخت مسیر معروف است و پس از آن یک راه‌حل ارائه می‌شود. برای به دست آوردن یک راه‌حل بهتر، هر مورچه فرومون‌ها را در مسیر خود تقویت می‌کند تا سایر مورچه‌ها را هدایت کند. مورچه‌ها به صورت تصادفی شهر بعدی را بر اساس اطلاعات اکتشافی به‌دست‌آمده از فاصله بین شهرها و رد فرومون‌ها انتخاب می‌کنند. با این حال، یک فرایند تبخیر فرومون نیز اعمال می‌شود تا از گیر افتادن در یک راه‌حل بهینه محلی جلوگیری شود.

            الگوریتم 1 چارچوب کد ترتیبی برای بهینه‌سازی کلونی مورچه‌ها را نشان می‌دهد. ابتدا، تمام ساختارهای داده‌ای برای مسئله فروشنده دوره‌گرد، شامل فهرست شهرهای بازدید شده و فاصله شهرها، مقداردهی اولیه می‌شوند. سپس، در مرحله ساخت مسیر، $m$ مورچه به ترتیب به $n$ شهر سفر می‌کنند. پس از آن، هر مسیر می‌تواند با استفاده از یک فرایند جستجوی محلی بهبود یابد. در مرحله به‌روزرسانی فرومون، هر مورچه به‌صورت جداگانه فرومون‌ها را بر روی مسیرهای $n$ طی‌شده خود رسوب می‌دهد. این مراحل به‌صورت تکراری انجام می‌شوند تا زمانی که معیار توقف حاصل شود.

            




\end{document}