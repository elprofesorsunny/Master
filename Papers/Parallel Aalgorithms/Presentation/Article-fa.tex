\documentclass[a4paper,10pt]{article}

\usepackage{dblfnote}
\usepackage{lipsum}

\usepackage{geometry}
\geometry{top=2cm, bottom=2cm, left=1.5cm, right=1.5cm}

\usepackage{xepersian}
\settextfont{Vazirmatn-Regular.ttf}

\title{\huge بهینه‌سازی کلونی مورچه‌ها به‌صورت موازی بر روی پردازنده‌های چند هسته‌ای SIMD}
\author{محمد خورشیدی روزبهانی\\40215741002013 \and شارا شاهوردیان\\40215741002032}
\date{}

\linespread{1.5}

\begin{document}

    \maketitle

    % \vspace{5pt}

    \begin{abstract}
        
        بهینه‌سازی کلونی مورچه‌ها\footnote{\hspace{2pt}Optimization Colony Ant - ACO} یک فراابتکاری\footnote{\hspace{2pt}Metaheuristics} مبتنی بر جمعیت برای حل مسائل سخت بهینه‌سازی ترکیبی است. مطالعات بسیاری به تسریع بهینه‌سازی کلونی مورچه‌ها با استفاده از سخت‌افزارهای موازی، به‌ویژه واحدهای پردازش گرافیکی\footnote{\hspace{2pt}Unit Processing Graphics - GPU}، اختصاص یافته است. با این حال، به دلیل الگوی نامنظم (تصادفی) در دسترسی به داده‌ها و جریان کنترل در الگوریتم‌های بهینه‌سازی کلونی مورچه‌ها، عملکرد روش‌های مبتنی بر واحدهای پردازش گرافیکی به دلیل محدودیت‌های سخت‌افزاری محدود می‌شود. پردازش مبتنی بر SIMD در پردازنده‌های برای بهینه‌سازی کلونی مورچه‌ها در ادبیات گذشته به ندرت مورد بررسی قرار گرفته و همچنان ناشناخته است که الگوریتم‌های بهینه‌سازی کلونی مورچه‌ها مبتنی بر پردازنده‌های چند هسته‌ای SIMD تا چه حد می‌توانند عملکرد داشته باشند. در این مقاله، ما مدلی از بهینه‌سازی کلونی مورچه‌ها موازی برداری برای معماری پردازنده‌های چند هسته‌ای SIMD ارائه و ارزیابی می‌کنیم. در مدل پیشنهادی، هر مورچه با یک هسته پردازنده مرتبط شده و ساخت مسیر هر مورچه با استفاده از دستورات برداری تسریع می‌شود. علاوه بر این، بر اساس مدل پیشنهادی، یک روش جدید انتخاب متناسب با تناسب به نام چرخ رولت مبتنی بر بردار\footnote{\hspace{2pt}Wheel Roulette Vector-based - VRW} در مرحله ساخت مسیر معرفی می‌شود. در این روش، مقادیر تناسب در خطوط SIMD گروه‌بندی شده و مجموع پیشوند به‌صورت موازی برداری محاسبه می‌شود. الگوریتم پیشنهادی بر روی نمونه‌های استاندارد مسئله فروشنده دوره‌گرد\footnote{\hspace{2pt}Problem Salesman Travelling The - TSP} با ابعاد 198 تا 4461 شهر آزمایش شده و نشان‌دهنده سرعت‌افزایی تا 57.8 برابر نسبت به نسخه تک‌ریسمانی پردازنده است. به‌طور قابل توجهی، ما رویکرد خود را با بهینه‌سازی‌های کلونی مورچه‌ها مبتنی بر واحدهای پردازش گرافیکی با کارایی بالا مقایسه کرده‌ایم و نتایج پتانسیل قوی بهینه‌سازی‌های کلونی مورچه‌ها موازی مبتنی بر پردازنده را نشان می‌دهد.

    \end{abstract}

    % \vspace{15pt}

    \section{مقدمه}

        حل مسائل بهینه‌سازی در دنیای واقعی پیچیده و زمان‌بر است، به‌ویژه برای پردازنده‌های مرکزی در مسائل بزرگ‌مقیاس. مدل‌سازی این مسائل نیز وابسته به نوع مسئله است. فراابتکارها روش‌های کارآمدی هستند که امکان دستیابی به یک حل رضایت‌بخش (یک بهینه تقریبی) را در زمانی معقول فراهم می‌کنند و چارچوبی الگوریتمی عمومی ارائه می‌دهند که می‌توان آن را با تغییرات اندک در مسائل مختلف به کار برد. بهینه‌سازی کلونی مورچه‌ها یک فراابتکار مبتنی بر جمعیت است که از رفتار اجتماعی مورچه‌ها الهام گرفته است. مورچه‌های مصنوعی به‌طور مستقل راه‌حل‌هایی را می‌سازند و از طریق مکانیزم استیگمرژی\footnote{\hspace{2pt}Stigmergy} با یکدیگر ارتباط برقرار می‌کنند. این فرآیند به‌صورت تکراری تا زمانی که معیار توقف حاصل شود، اجرا می‌شود. مهم‌ترین ویژگی بهینه‌سازی کلونی مورچه‌ها بازخورد مثبت است که از طریق فرومون‌های به‌جا مانده توسط مورچه‌ها، می‌تواند فرآیند ساخت راه‌حل را هدایت کند. این فراابتکار برای حل مسائل NP-سخت مانند مسئله فروشنده دوره‌گرد و مسئله تخصیص درجه دوم\footnote{\hspace{2pt}Problem Assignment Quadratic - QAP} مورد استفاده قرار گرفته است.

        

\end{document}