\documentclass[a4paper,10pt]{article}

\usepackage{geometry}
\geometry{
    left=1cm,
    right=1cm,
    top=2cm,
    bottom=2cm
}

\usepackage{fancyhdr}
\pagestyle{fancy}
\fancyhf{}
\fancyhead[R]{مهندسی نیازمندی‌ها}
\fancyhead[C]{تمرین 1}
\fancyhead[L]{\today}
\fancyfoot[R]{محمد خورشیدی روزبهانی - 40215741002013}
\fancyfoot[L]{\thepage}

\usepackage{xepersian}
\settextfont{Vazirmatn-Regular.ttf}

\linespread{1.5}

\begin{document}

    \paragraph{سوال:} منظور از روایی و پایایی  پرسشنامه چیست؟

    \paragraph{پاسخ:} روایی و پایایی پرسشنامه دو مفهوم مهم در روش‌شناسی پژوهش هستند که به اعتبار و قابلیت اطمینان پرسشنامه اشاره دارند. روایی به میزانی اشاره دارد که پرسشنامه واقعیت را اندازه‌گیری می‌کند یا به عبارتی دیگر میزانی که پرسش‌ها و محتوای آن واقعیت را به خوبی نمایش می‌دهند. پایایی نیز به میزانی اشاره دارد که پرسشنامه به طور قابل تکرار نتایجی مشابه را در شرایط مشابه تولید می‌کند. به عبارت دیگر پایایی نشان‌دهنده استمرار قابلیت اندازه‌گیری معیارهای مورد نظر در طول زمان و در مقابسه با شرایط مختلف است.

    به طور کلی مواردی برای این دو مفهوم وجود دارد که به شرح زیر عبارتند از:

    \begin{enumerate}
        
        \item روایی (Validaty)
        
        \begin{itemize}
            
            \item روایی پرسشنامه نشان‌دهنده میزانی است که پرسشنامه واقعیت را اندازه‌گیری می‌کند.

            \item برای بررسی روایی، معمولاً از روش‌های مختلفی مانند روایی محتوا، روایی سازمان‌دهی و روایی پیش‌بینی استفاده می‌شود. این موارد به تفکیک به توضیح زیر می‌باشند:
            
            \begin{itemize}
                
                \item روایی محتوا مربوط به اندازه‌گیری دقیق و جامع مفهوم مورد بررسی است.

                \item روایی سازمان‌دهی نشان‌دهنده توانایی پرسشنامه در گردآوری اطلاعات مطلوب با استفاده از ساختار و منطق مناسب است.
    
                \item روایی پیش‌بینی، مربوط به توانایی پرسشنامه در پیش‌بینی نتایج یا رفتارهای آینده  است.

            \end{itemize}

        \end{itemize}

        \item پایایی (Reliability)

        \begin{itemize}
            
            \item پایایی نشان‌دهنده استمرار قابلیت اندازه‌گیری معیارهای مورد نظر در طول زمان و در مقایسه با شرایط مختلف است.
            
            \item برای بررسی پایایی، معمولاً از روش‌های مختلفی مانند پایایی داخلی، پایایی اعتبارهای متعدد و پایایی زمانی استفاده می‌شود. این موارد به تفکیک به توضیح زیر می‌باشند:
            
            \begin{itemize}
                
                \item پایایی داخلی مربوط به اندازه‌گیری همبستگی بین موارد مختلف در پرسشنامه است.

                \item پایایی اعتبارهای متعدد مربوط به اندازه‌گیری تطابق نتایج با استفاده از ابزارهای مختلف یا اندازه‌گیری متعدد است.

                \item پایایی زمانی مربوط به استمراری بودن نتایج اندازه‌گیری در زمان مختلف است.

            \end{itemize}

        \end{itemize}

    \end{enumerate}

    درک صحیح این دو مفهوم و اعمال صحیح آن‌ها در فرآیند طراحی و ارزیابی پرسشنامه، اهمیت بسیاری در صحت و قابل اعتماد بودن داده‌های به دست آمده دارد.

\end{document}