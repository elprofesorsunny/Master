\documentclass[a4paper,10pt]{article}

\usepackage{geometry}
\geometry{
    left=1cm,
    right=1cm,
    top=2cm,
    bottom=2cm
}

\usepackage{fancyhdr}
\pagestyle{fancy}
\fancyhf{}
\fancyhead[R]{مهندسی نیازمندی‌ها}
\fancyhead[C]{تمرین 3}
\fancyhead[L]{\today}
\fancyfoot[R]{محمد خورشیدی روزبهانی - 40215741002013}
\fancyfoot[L]{\thepage}

\usepackage{xepersian}
\settextfont{Vazirmatn-Regular.ttf}

\linespread{1.5}

\begin{document}

    \paragraph{سوال:} یک سیستم مدریریت یادگیری مانند Connect Adobe یا SkyRoom یا \dots را در نظر بگیرید و برای آن سیستم دو جمله بنویسید که دارای تضاد ضعیف یا قوی باشند و سپس در ادامه برای آن تضاد راه‌حل مطرح کنید. (استفاده از یکی از چهار تکنیک رفع تضاد که ندریس شده است الزامی می‌باشد.)

    \paragraph{پاسخ 1:} در سیستم مدیریت یادگیری، تضادی وجود دارد بین امکان ارائه آموزش‌های تعاملی و زمان‌بندی محدود برای دسترسی به منابع آموزشی. اولین جمله، امکان انعطاف‌پذیری در برنامه‌ریزی آموزشی را برجسته می‌کند، در حالی که دومین جمله، نیاز به استانداردهای ثابت را بیشتر تاکید می‌کند.

    برای رفع تضاد 4 تکنیک وجود دارد که عبارتند از:

    \begin{itemize}
        
        \item اجتناب از شرایط مرزی Conditions) Boundary (Avoid

        \item بازیابی اظهارات متضاد Statements) Conflicting (Restore

        \item تضعیف اظهارات متضاد Statements) Conflivting (Weaken

        \item تخصیص منبع تضاد به هدف Target) to Source Conflict (Specialize

    \end{itemize}

    در ادامه برای رفع تضاد دو جمله بالا از تکنیک «بازیابی اظهارات متضاد» استفاده می‌شود به این صورت که می‌توانیم با ارائه زمان‌بندی انعطاف‌پذیرتر برای دسترسی به منابع آموزشی، تضاد بین این دو را حل کنیم.

    \noindent\hrulefill

    \paragraph{پاسخ 2:} در سیستم مدیریت یادگیری، تضادی وجود دارد بین امکان ارائه انعطاف‌پذیری در برنامه‌ریزی آموزشی و نیاز به استانداردهای ثابت. اولین جمله، امکان انعطاف‌پذیری در برنامه‌ریزی آموزشی را برجسته می‌کند، در حالی که دومین جمله، نیاز به استانداردهای ثابت را بیشتر می‌کند.

    برای رفع تضاد 4 تکنیک وجود دارد که عبارتند از:

    \begin{itemize}
        
        \item اجتناب از شرایط مرزی Conditions) Boundary (Avoid

        \item بازیابی اظهارات متضاد Statements) Conflicting (Restore

        \item تضعیف اظهارات متضاد Statements) Conflivting (Weaken

        \item تخصیص منبع تضاد به هدف Target) to Source Conflict (Specialize

    \end{itemize}

    در ادامه برای رفع تضاد دو جمله بالا از تکنیک «تخصیص منبع تضاد به هدف» استفاده می‌شود به این صورت که می‌توانیم با تعیین دقیق‌تر استانداردهای آموزشی برای هر دوره آموزشی، تضاد بین امکان انعطاف‌پذیری و نیاز به استانداردهای ثابت را برطرف کنیم.

\end{document}