\documentclass[a4paper,10pt]{article}

\usepackage{tabularx}
\usepackage{graphicx}
\usepackage{geometry}
\geometry{
    top=2cm,
    bottom=2cm,
    left=1cm,
    right=1cm
}

\usepackage{xepersian}
\settextfont{Vazirmatn-Regular.ttf}

\title{ML4RE - یادگیری ماشین برای مهندسی نیازمندی‌ها}
\author{}
\date{}

\linespread{1.5}

\begin{document}

    \maketitle

    \begin{abstract}
        
        مقدمه: تحقیقات در زمینه یادگیری ماشین برای مهندسی نیازمندی‌ها (ML4RE) به تدریج توجه بیشتری از سوی محققان و عملی‌کنندگان به خود جلب کرده است. اگرچه تحقیقات پیشگامانه پتانسیل استفاده از تکنیک‌های یادگیری ماشین برای بهبود فرآیندهای مهندسی نیازمندی‌ها را نشان داده‌اند، اما یک مرور نظام‌مند و جامع از ادبیات علمی که دیدگاه صنعتی را نیز در بر گیرد، در دانشگاه‌ها وجود ندارد. به‌ویژه، هیچ‌یک از مرورهای موجود در زمینه ML4RE به ادبیات خاکستری که عمدتاً از منابع عملی‌کنندگان منشأ می‌گیرد و بازتاب‌دهنده مسائل و چالش‌های واقعی در عمل است، توجه نکرده‌اند.

        هدف: در این مقاله، ما یک بررسی نظام‌مند از انتشارات علمی در زمینه ML4RE انجام می‌دهیم و آن را با نظرات عملی‌کنندگان از Stack Overflow تکمیل می‌کنیم تا یک مرور جامع از ادبیات ارائه دهیم. هدف تحقیق ما ارائه یک دیدگاه جامع از پیشرفت‌های کنونی در تحقیقات ML4RE، بیان سوالات و چالش‌های اصلی در عمل مهندسی نیازمندی‌ها، درک فاصله بین تحقیق و عمل، و ارائه بینش‌های خود درباره چگونگی توسعه عملی این حوزه دانشگاهی در آینده است.

        روش: ما به صورت نظام‌مند 207 مقاله علمی در زمینه ML4RE از سال 2010 تا 2022 را بررسی کردیم و همچنین 375 سوال مرتبط با مهندسی نیازمندی‌ها در Stack Overflow و پاسخ‌های مربوطه را تحلیل کردیم. تحلیل ما شامل روندها، فعالیت‌ها و وظایف متمرکز بر مهندسی نیازمندی‌ها، راه‌حل‌های به‌کاررفته و داده‌های مرتبط بود. در نهایت، یک تحلیل مشترک انجام دادیم و نتایج هر دو بخش را با هم مقایسه کردیم.
        
        نتایج: بر اساس نتایج آماری از ادبیات جمع‌آوری‌شده، ما یک نقشه راه علمی را خلاصه کرده و تفاوت‌ها را تحلیل کردیم و توصیه‌های پژوهشی ارائه دادیم. پیشنهادات ما شامل توسعه دستیاران هوشمند پاسخگویی به سوالات با استفاده از مدل‌های زبان بزرگ، ادغام یادگیری ماشین در ابزارهای صنعتی و ترویج همکاری بین دانشگاه و صنعت است.

        نتیجه‌گیری: این مطالعه با ارائه یک دیدگاه جامع از ML4RE، بیان تفاوت‌های بین تحقیق و عمل، و پیشنهاد راه‌حل‌های عملی برای پر کردن شکاف بین دانشگاه و صنعت، به پیشرفت این حوزه کمک می‌کند.

    \end{abstract}

\end{document}