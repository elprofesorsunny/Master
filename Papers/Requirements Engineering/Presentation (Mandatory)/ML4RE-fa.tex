\documentclass[a4paper,10pt]{article}

\usepackage{tabularx}
\usepackage{graphicx}
\usepackage{geometry}
\geometry{
    top=2cm,
    bottom=2cm,
    left=1cm,
    right=1cm
}

\usepackage{xepersian}
\settextfont{Vazirmatn-Regular.ttf}

\title{ML4RE - یادگیری ماشین برای مهندسی نیازمندی‌ها}
\author{}
\date{}

\linespread{1.5}

\begin{document}

    \maketitle

    % MARK: abstract

    \begin{abstract}
        
        مقدمه: تحقیقات در زمینه یادگیری ماشین برای مهندسی نیازمندی‌ها (ML4RE) به تدریج توجه بیشتری از سوی محققان و عملی‌کنندگان به خود جلب کرده است. اگرچه تحقیقات پیشگامانه پتانسیل استفاده از تکنیک‌های یادگیری ماشین برای بهبود فرآیندهای مهندسی نیازمندی‌ها را نشان داده‌اند، اما یک مرور نظام‌مند و جامع از ادبیات علمی که دیدگاه صنعتی را نیز در بر گیرد، در دانشگاه‌ها وجود ندارد. به‌ویژه، هیچ‌یک از مرورهای موجود در زمینه ML4RE به ادبیات خاکستری که عمدتاً از منابع عملی‌کنندگان منشأ می‌گیرد و بازتاب‌دهنده مسائل و چالش‌های واقعی در عمل است، توجه نکرده‌اند.

        هدف: در این مقاله، ما یک بررسی نظام‌مند از انتشارات علمی در زمینه ML4RE انجام می‌دهیم و آن را با نظرات عملی‌کنندگان از Stack Overflow تکمیل می‌کنیم تا یک مرور جامع از ادبیات ارائه دهیم. هدف تحقیق ما ارائه یک دیدگاه جامع از پیشرفت‌های کنونی در تحقیقات ML4RE، بیان سوالات و چالش‌های اصلی در عمل مهندسی نیازمندی‌ها، درک فاصله بین تحقیق و عمل، و ارائه بینش‌های خود درباره چگونگی توسعه عملی این حوزه دانشگاهی در آینده است.

        روش: ما به صورت نظام‌مند 207 مقاله علمی در زمینه ML4RE از سال 2010 تا 2022 را بررسی کردیم و همچنین 375 سوال مرتبط با مهندسی نیازمندی‌ها در Stack Overflow و پاسخ‌های مربوطه را تحلیل کردیم. تحلیل ما شامل روندها، فعالیت‌ها و وظایف متمرکز بر مهندسی نیازمندی‌ها، راه‌حل‌های به‌کاررفته و داده‌های مرتبط بود. در نهایت، یک تحلیل مشترک انجام دادیم و نتایج هر دو بخش را با هم مقایسه کردیم.
        
        نتایج: بر اساس نتایج آماری از ادبیات جمع‌آوری‌شده، ما یک نقشه راه علمی را خلاصه کرده و تفاوت‌ها را تحلیل کردیم و توصیه‌های پژوهشی ارائه دادیم. پیشنهادات ما شامل توسعه دستیاران هوشمند پاسخگویی به سوالات با استفاده از مدل‌های زبان بزرگ، ادغام یادگیری ماشین در ابزارهای صنعتی و ترویج همکاری بین دانشگاه و صنعت است.

        نتیجه‌گیری: این مطالعه با ارائه یک دیدگاه جامع از ML4RE، بیان تفاوت‌های بین تحقیق و عمل، و پیشنهاد راه‌حل‌های عملی برای پر کردن شکاف بین دانشگاه و صنعت، به پیشرفت این حوزه کمک می‌کند.

    \end{abstract}

    % MARK: Introduction
    
    \section{مقدمه}

    مهندسی نیازمندی‌ها (RE) یک مرحله اساسی در مراحل اولیه مهندسی نرم‌افزار (SE) است. اگرچه پژوهشگران به طور مستمر در حال بررسی روش‌ها و تکنیک‌هایی برای تسهیل فرآیندهای نیازمندی هستند، اما کل فرآیند مهندسی نیازمندی‌ها همچنان نیاز به تلاش دستی زیادی دارد (مثلاً استخراج نیازمندی‌های ذی‌نفعان از طریق مصاحبه یا طبقه‌بندی نیازمندی‌ها بر اساس یک طبقه‌بندی خاص). دلیل اصلی این موضوع این است که فعالیت‌های RE معمولاً نیاز به دانش عمیق حوزه و مهارت‌های تحلیل پیشرفته دارند که به طور کامل قابل اتوماسیون نیست.

    در سال‌های اخیر، توسعه سریع فناوری یادگیری ماشین (ML) با بهبود قدرت محاسباتی تحریک شده است. کاربردهای موفق ML در زمینه‌هایی مانند پردازش زبان طبیعی، شناسایی تصویر و داده‌کاوی فرصت‌هایی را برای استفاده از تکنیک‌های ML در زمینه RE فراهم کرده است. استفاده از فناوری ML در RE یک رویکرد هوشمندانه‌تر و کارآمدتر برای مدیریت داده‌های نیازمندی‌ها ارائه می‌دهد. به عنوان مثال، ML می‌تواند در طبقه‌بندی خودکار نیازمندی‌ها کمک کند زیرا می‌تواند اطلاعات بالقوه نیازمندی‌ها را خلاصه کند.

    علاوه بر این، با توسعه سریع تکنیک‌های اطلاعاتی، کار و زندگی روزمره ما دیجیتالی می‌شوند. در نتیجه، داده‌های مرتبط با نیازمندی‌ها بیشتر و بیشتر دیجیتالی و به‌طور عمومی در دسترس قرار می‌گیرند، که پژوهش در زمینه یادگیری ماشین برای مهندسی نیازمندی‌ها (ML4RE) را ترویج می‌کند. به عنوان مثال، بررسی‌های کاربران از برنامه‌های موبایلی به طور گسترده‌ای برای استخراج نیازمندی‌های کاربران مورد بررسی قرار گرفته‌اند.

    تحقیقات قبلی ML4RE را مورد بررسی قرار داده‌اند. اقبال و همکاران [2] یک بررسی برای به‌دست‌آوردن نمای کلی از چگونگی کمک تکنیک‌های ML به فعالیت‌های RE انجام دادند. علاوه بر این، زمانی و همکاران [3] یک مطالعه نگاشت از کاربردهای ML در RE انجام دادند و 65 مقاله را برای ارزیابی اثربخشی ML در اتوماسیون وظایف RE تحلیل کردند. کارهای آن‌ها بر کل فرآیند RE متمرکز بود و نحوه تأثیرگذاری و تسهیل تکنیک‌های ML در مراحل مختلف را روشن کردند.

    به علاوه، برخی تحقیقات به فعالیت‌ها یا وظایف خاص RE می‌پردازند. به عنوان مثال، لیم و همکاران [4] رویکردهای پیشرفته فعلی برای استخراج نیازمندی‌های مبتنی بر داده از منابع داده پویا را بررسی کردند. ما متوجه شدیم که این مطالعات عمدتاً بر انتشارات علمی متمرکز بوده و از ادغام بینش‌های حاصل از منابع ادبیات خاکستری، مانند وبلاگ‌ها و انجمن‌های صنعتی غافل بوده‌اند.

    بر خلاف انتشارات علمی که عمدتاً توسط پژوهشگران منتشر می‌شوند، ادبیات خاکستری به‌طور مداوم توسط عملی‌کنندگان تولید می‌شود و بر "وضعیت عمل" نور می‌تاباند [5]. همان‌طور که در [6] اشاره شده، ادغام ادبیات خاکستری در مرورهای نظام‌مند ادبیات می‌تواند فاصله بین پژوهش‌های علمی و عملی را پر کند و دیدگاه جامع‌تری از چالش‌ها و راه‌حل‌ها ارائه دهد.

    اگرچه تعداد مرورهای نظام‌مند ادبیات که ادبیات خاکستری را در مطالعات SE در نظر گرفته‌اند در حال افزایش است [7,8]، اما در RE به اندازه کافی رایج نیستند. برای پر کردن این شکاف در زمینه RE، این مقاله قصد دارد یک مرور نظام‌مند از ادبیات در زمینه ML4RE انجام دهد که با بینش‌های حاصل از ادبیات خاکستری منابع شده از Stack Overflow تکمیل شود.

    هدف این مرور ادبیات سه بخشی است. بخش سفید شامل مرور 207 مقاله منتشر شده بین سال‌های 2010 تا 2022 است که به‌طور خاص بر ML4RE متمرکز است. در همین حال، بخش خاکستری شامل تحلیل 375 سوال و پاسخ‌های مربوطه جمع‌آوری شده از مباحث Stack Overflow درباره فعالیت‌های RE در همان دوره است. در نهایت، تحلیل مشترک ما شامل مقایسه نتایج این دو بخش برای تشخیص شباهت‌ها و تفاوت‌های آن‌ها است. ما روندها، فعالیت‌های RE، وظایف RE، راه‌حل‌ها و داده‌های موجود در ادبیات را تحلیل می‌کنیم.

    نتایج تحقیق نشان می‌دهد که هر دو بخش به تحلیل RE و مستندسازی نیازمندی‌ها علاقه‌مند هستند. فراتر از شباهت‌ها، بخش سفید تمایل به تمرکز بر استخراج نیازمندی‌ها دارد، در حالی که بخش خاکستری بیشتر بر مدیریت نیازمندی‌ها تأکید دارد. بخش سفید استفاده از تکنولوژی‌های ML مانند SVM، CNN و BERT را برجسته می‌کند. در مقابل، بخش خاکستری بیشتر بر ابزارهایی مانند Microsoft TFS، Jira و IBM Rational DOORS و تکنیک‌های ML مانند LDA، POS، TF-IDF و شبکه‌های عصبی تکیه دارد. علاوه بر این، بخش خاکستری توجه ویژه‌ای به داستان کاربر و مورد استفاده دارد که در بخش سفید نسبتاً کمتر مورد بررسی قرار گرفته است.

    بر اساس این یافته‌ها، ما یک نقشه راه علمی خلاصه کرده و تحلیل دقیقی از تفاوت‌های بین بخش سفید و خاکستری ارائه می‌دهیم. سپس پیشنهادات پژوهشی ارائه می‌دهیم، از جمله توسعه دستیاران هوشمند پاسخگویی به سوالات با استفاده از مدل‌های زبان بزرگ و ادغام یادگیری ماشین در ابزارهای صنعتی. همچنین، همکاری بیشتر بین دانشگاه و صنعت را برای درک عمیق‌تر مشکلات پژوهشی واقعی و داده‌ها تشویق می‌کنیم.

    در خلاصه، این مقاله چهار کمک اصلی را ارائه می‌دهد. اولاً، یک نمای جامع از وضعیت فعلی پژوهش‌های ML4RE ارائه می‌دهیم. دوماً، شرایط واقعی عملی‌کنندگان RE را از طریق ادبیات خاکستری حاصل از Stack Overflow بررسی می‌کنیم. سوماً، فاصله بین پژوهش و عمل در حوزه ML4RE را به‌ویژه در زمینه‌هایی که کمتر مورد توجه پژوهشگران قرار گرفته‌اند، برجسته می‌کنیم. و در نهایت، برای پر کردن فاصله بین صنعت و دانشگاه، پیشنهادات پژوهشی عملی در ML4RE ارائه می‌دهیم.

    در بخش‌های باقی‌مانده این مقاله، کارهای مرتبط را در بخش 2 ارائه می‌دهیم. بخش 3 پروتکل تحقیق برای مرور نظام‌مند ادبیات ما را ارائه می‌دهد. در سه بخش بعدی، نتایج این بررسی و پاسخ به سوالات پژوهشی را ارائه می‌دهیم. بخش 4 نتایج بخش سفید، بخش 5 بر بخش خاکستری تمرکز می‌کند و بخش 6 نتایج تحلیل مشترک را ارائه می‌دهد. بر اساس نتایج، در بخش 7 به بحث پرداخته و چندین پیشنهاد ارائه می‌دهیم. بخش 8 شامل تحلیل تهدیدات به اعتبار این بررسی است. در نهایت، مقاله را در بخش 9 نتیجه‌گیری می‌کنیم.
    
    % MARK: Related work

    \section{کارهای مرتبط}

    کارهای مرتبط با بررسی ادبیات ما در حوزه یادگیری ماشین برای مهندسی نیازمندی‌ها (ML4RE) از دو منبع مختلف به دست آمده است. مجموعه اول شامل بررسی‌های ادبیات مرتبط با ML4RE می‌باشد. مجموعه دوم شامل بررسی‌های سیستماتیکی ادبیات در حوزه مهندسی نرم‌افزار (SE) است. در دو زیربخش پیش رو، جزئیات کارهای مرتبط از این دو منبع را به طور دقیق ارائه خواهیم داد.

        \subsection{بررسی‌های ادبیات در مورد یادگیری ماشین برای مهندسی نیازمندی‌ها (ML4RE):}

            در این بخش، یک مجموعه از بررسی‌های ادبیات مرتبط با یادگیری ماشین برای مهندسی نیازمندی‌ها (ML4RE) را ارائه می‌دهیم. ما چندین بررسی ادبیات را پیدا کرده‌ایم، برخی به کلیه فرآیند مهندسی نیازمندی‌ها می‌پردازند، در حالی که برخی دیگر بر روی فعالیت‌ها یا وظایف خاص مهندسی نیازمندی‌ها تمرکز دارند. بنابراین، این زیربخش به دو بخش تقسیم شده است تا این کارها را به تفصیل معرفی کند.

            \subsubsection{فرآیند کامل مهندسی نیازمندی‌ها}

                دو مقاله به کلیه فرآیند مهندسی نیازمندی‌ها متمرکز شده‌اند و یک بررسی کلی از نحوه کاربرد تکنیک‌های یادگیری ماشین در مراحل مختلف مهندسی نیازمندی‌ها ارائه داده‌اند. در مقاله Iqbal و همکاران [2]، بررسی‌ای بر روی مقالات تحقیقاتی انجام شده است تا چگونگی کمک یادگیری ماشین به مهندسی نیازمندی‌ها را بررسی کند. آن‌ها تأثیر یادگیری ماشین را در پنج مرحله اصلی مهندسی نیازمندی‌ها مشاهده کرده و مسائل خاصی که توسط یادگیری ماشین حل شده‌اند، ویژگی‌ها، الگوریتم‌های ML و مجموعه داده‌ها را بررسی کرده‌اند. با این حال، مقالاتی که آن‌ها بررسی کردند از یک فرآیند جستجوی سیستماتیک به دست نیامده بودند و بنابراین، نتیجه‌گیری‌های به دست آمده ممکن است سیستماتیک و جامع نباشد.

                Zamani و همکاران [3] یک مطالعه نگاشتی از کاربردهای یادگیری ماشین در مهندسی نیازمندی‌ها انجام دادند، با تجزیه و تحلیل 65 مقاله برای ارزیابی کارآیی یادگیری ماشین در اتوماسیون وظایف مهندسی نیازمندی. این مطالعه تکنیک‌ها، چالش‌ها، مجموعه داده‌ها و معیارهای ارزیابی این مطالعات را شناسایی می‌کند. مقایسه با بررسی جامع ما، این مقاله بیشتر بر جنبه‌های تجربی یادگیری ماشین در مهندسی نیازمندی‌ها تمرکز دارد و بینش‌های خاصی را در کارآمدی عملی ML در این زمینه ارائه می‌دهد. علاوه بر این، مقاله Zhao و همکاران [9] بر روی NLP4RE تمرکز داشتند، با تحلیل 404 مطالعه برای درک کاربرد پردازش زبان طبیعی در مهندسی نیازمندی‌ها. با توجه به تداخلات بین NLP و ML، این تحقیق را به عنوان یکی از کارهای مرتبط برای تحلیل می‌پذیریم.
            
            
            \subsubsection{بخشی از فرآیند مهندسی نیازمندی‌ها}

                در فرآیند استخراج نیازمندی‌ها، G.C. Sampada و همکاران [10] یک نگاه کلی از رویکردهای مختلف برای اتوماسیون استخراج و مشخصه‌گذاری نیازمندی‌ها در چرخه توسعه نرم‌افزار ارائه دادند. Lim و همکاران [4] وضعیت فعلی روش‌های پیشروی استخراج نیازمندی‌های مبتنی بر داده از منابع داده پویا را بررسی کردند و شکاف‌های تحقیق را شناسایی کردند. Cheligeer و همکاران [11] با انتخاب 86 مقاله، مطالعاتی را که فناوری‌های ML و NLP را در استخراج نیازمندی‌ها شامل می‌شوند، خلاصه و تحلیل کردند. آن‌ها تکنیک‌های مختلف برای ساخت روش‌های استخراج نیازمندی مبتنی بر ML را به پنج بخش دسته‌بندی کردند.

                در فرآیند طبقه‌بندی نیازمندی‌ها، Alrumaih و همکاران [12] به بررسی مطالعات تحقیقی در زمینه طبقه‌بندی نیازمندی‌ها پرداختند و محدودیت‌ها را بررسی کردند تا پیشنهادهای بهبودی ارائه دهند. Perez و همکاران [13] کاربردهای تکنیک‌های ML در طبقه‌بندی نیازمندی‌های نرم‌افزار را بر اساس 13 مقاله بررسی کردند و الگوریتم‌های طبقه‌بندی مکررترین و مجموعه داده‌های آموزشی مکررترین را خلاصه کردند. Khelifa و همکاران [14] بررسی کردند که آیا تکنیک‌های یادگیری ماشین در طبقه‌بندی نیازمندی‌ها نرم‌افزاری و طبقه‌بندی درخواست‌های تغییرات نیازمندی‌ها قابل اعمال هستند. به علاوه، Kadebu و همکاران [15] بر روی مهندسی نیازمندی‌های امنیتی تمرکز کردند و کاربردهای تکنیک‌های ML در استخراج و طبقه‌بندی نیازمندی‌های امنیتی را بررسی کردند.

                در فرآیند مدیریت نیازمندی‌ها، Xu و همکاران [16] هشت روش ML را که در مدیریت نیازمندی‌ها استفاده شده است خلاصه کردند و 18 شاخص ارزیابی برای مدیریت نیازمندی‌ها در روش ML مشخص کردند. کار آن‌ها به عنوان یک درک اولیه از گستره‌ی وسیعی از تکنیک‌های ML در مدیریت نیازمندی‌ها خدمت می‌کند، در حالی که برخی مطالعات به تفصیل به وظایف خاص می‌پردازند.

                Achimugu و همکاران [17] به بررسی تکنیک‌های اولویت‌بندی نیازمندی‌های نرم‌افزار از طریق 73 مقاله مرتبط پرداختند و چندین محدودیت در تکنیک‌های اولویت‌بندی موجود را مورد بررسی قرار دادند. به علاوه، Li و همکاران [18] با انجام یک مطالعه نگاشت سیستماتیک با 26 مطالعه، 32 فناوری ML برای پیگیری نیازمندی‌ها را خلاصه کردند. از مطالب فوق مشخص است که تعداد زیادی از بررسی‌های ادبیات عالی در زمینه ML4RE وجود دارد. با این حال، در حال حاضر، کمبودی در تحقیقات وجود دارد که ادغام بخش خاکستری را که نماینده جنبه‌های صنعتی است، در نظر بگیرد. هدف کار ما پر کردن این شکاف است با جامع نگاه داشتن به دیدگاه‌های دانشگاهی و صنعتی.
        
        \subsection{بررسی‌های ادبیات در مهندسی نرم‌افزار}

            در زمینه مهندسی نرم‌افزار، تعداد زیادی بررسی ادبیات سیستماتیک وجود دارد. ما مقالات مرتبطی را که شامل ادبیات خاکستری هستند انتخاب کرده‌ایم. آن‌ها را بر اساس سه حوزه موضوعی دسته‌بندی کرده‌ایم که هرکدام به ترتیب معرفی می‌شوند:

            \subsubsection{منابع متدولوژی‌های توسعه نرم‌افزار}

                منابع متدولوژی‌های توسعه نرم‌افزار به مدل‌ها یا سیستم‌های ارزش‌گذاری مهندسی نرم‌افزار است که توسط توسعه‌دهندگان گسترده در فرآیند توسعه نرم‌افزار پذیرفته می‌شود. متدولوژی‌های معروف در توسعه نرم‌افزار شامل Agile، DevOps و DevSecOps است که بر اساس آخرین مورد بر اساس DevOps بررسی می‌شود. França و همکاران [19] یک بررسی ادبیات انجام دادند با هدف توصیف DevOps از دیدگاه‌های مختلف. Amaro و همکاران [20] به هدف روشن‌سازی قابلیت‌های DevOps و ارتباط آن‌ها با شیوه‌های عملیاتی DevOps پرداختند. با پیشرفت DevOps، امنیت برای مهندسی نرم‌افزار اهمیت بیشتری پیدا می‌کند. DevSecOps با یکپارچگی روش‌های امنیتی مدرن و DevOps برای اجرای این امر به وجود آمده است. Myrbakken و Colomo-Palacios [8] یک بررسی ادبیات انجام دادند تا یک دید کلی از تعریف، اهمیت، مزایا و چالش‌های DevSecOps ارائه دهند. برای کیفیت پیاده‌سازی DevSecOps، Prates و همکاران [21] یک بررسی ادبیات انجام دادند تا معیارهایی که تیم‌ها می‌توانند برای اندازه‌گیری کارایی پیاده‌سازی متدولوژی DevSecOps در سازمان‌ها استفاده کنند، شناسایی کنند. 

                بیشتر و بیشتر شرکت‌های IT به معماری خدمات میکرو بازیافته تا کسب و کار خود را ارائه دهند. Soldani و همکاران [22] ادبیات خاکستری صنعتی را درباره دردها و سودهای معماری میکروسرویس‌ها به صورت سیستماتیک انتخاب و تجزیه و تحلیل کردند. Bhandari و Colomo-Palacios [23] بر روی هولاکراسی برای تیم‌های توسعه نرم‌افزار تمرکز کردند. برخی از اعمال به اینکه چگونه از ML برای کمک به DevOps در توسعه استفاده می‌شود. Recupito [24] یک بررسی ادبیات چند صداگذاری انجام داد تا ابزارهای MLOps و قابلیت‌های آنها در خودکارسازی لوله‌های یادگیری ماشین با شیوه‌های عملیاتی DevOps را بررسی کند.
            
            \subsubsection{مهندسی نرم‌افزار عمومی}

                بعضی از تحقیقات به ادغام صداهای حرفه‌ایان در حوزه گسترده‌تر مهندسی نرم‌افزار متمرکز شده‌اند. Kamei و همکاران [25] یک مطالعه سومی انجام دادند تا درکی از استفاده تحقیقات ثانویه از ادبیات خاکستری به دست آورند. با توجه به وضعیت محققان در مهندسی نرم‌افزار که هنوز با ارتباط کم تحقیقات با نیازهای حرفه‌ایان درگیر بودند، Garousi و همکاران [26] یک بررسی ادبیات انجام دادند. آن‌ها درک‌هایی از علل کم‌ارتباطی و پیشنهادهایی برای بهبود آن را به دست آوردند. Rainer و Williams [27] یک مطالعه سومی را در مورد تحقیقات به شیوه‌های عملی نرم‌افزاری با استفاده از اسناد شبیه به وبلاگ انجام دادند. Alves و همکاران [28] یک طبقه‌بندی جامع از شیوه‌های استفاده شده در صنعت برای ساخت سیستم‌های یادگیری ماشین ارائه دادند، که برای سازمان‌ها برای بهبود و مدیریت فرآیندها و شیوه‌های ML آموزنده است. Heiland و همکاران [29] یک دیدگاه کلی از الگوهای طراحی برای سیستم‌های مبتنی بر هوش مصنوعی ارائه دادند، که شامل الگوهای جدید و تطبیق‌یافته است، جمع‌آوری شده از طریق یک بررسی ادبیات چند صداگذاری.

            \subsubsection{بخش‌های خاص در مهندسی نرم‌افزار}

                بررسی‌های ادبیات در بخش‌های مختلف مهندسی نرم‌افزار وجود دارد. در حوزه آزمون نرم‌افزار، Raulamo-Jurvanen و همکاران [30] یک بررسی ادبیات خاکستری انجام دادند تا مشکلات پتانسیلی فرآیندهای موجود و فرصت‌های ارزیابی جامع ابزار را شناسایی کنند. برای خودکارسازی آزمون، Garousi و Mäntylä [7] بررسی ادبیاتی در مورد زمان و چه چیزی را باید در آزمون نرم‌افزار خودکارسازی کردند. Garousi و همکاران [31] بر روی ارزیابی رشد آزمون و بهبود فرآیند آزمون تمرکز داشتند و بررسی ادبیاتی را انجام دادند. Garousi و Küçük [32] یک نقشه‌برداری ادبیات چند صداگذاری را در مورد بوی‌های آزمون در هر دو ادبیات علمی و خاکستری انجام دادند. Felderer و Garousi [33] آزمون نرم‌افزار را در صنعت و دانشگاه مورد بررسی قرار دادند و پیشنهادات خود را درباره بهبود ارتباط و همکاری بین صنعت و دانشگاه در آزمون نرم‌افزار ارائه دادند. در حوزه مهندسی نیازها، Tripathi و همکاران [34] از بررسی ادبیات برای یافتن ادبیات علمی و خاکستری استفاده کردند. آن‌ها بررسی کردند که چگونه استارتاپ‌های نرم‌افزاری از استخراج نیاز، مستندسازی، اولویت‌بندی و اعتبارسنجی نیاز استفاده می‌کنند. به طور کلی، در منظر علمی مهندسی نرم‌افزار، توسعه‌های قابل توجهی در بررسی‌های ادبیات سیستماتیک دیده شده است، با انجام اعمال بسیار عالی که چشم‌اندازهای از ادبیات خاکستری را برای تحلیل ترکیب می‌کنند. با این حال، تحلیل ما نشان می‌دهد که از بین این اعمال، هنوز به تفکیک در ML4RE پرداخته نشده است، در حالی که کار ما این نقطه را پر می‌کند.

    % MARK: Research protocol

    \section{پروتکل تحقیقاتی}
    
        ما از ساختار استاندارد پروتکل مطالعه نقشه‌برداری سیستماتیک در مهندسی نرم‌افزار که توسط Kitchenham و همکاران [35] توسعه داده شده است، استفاده کردیم. مطالعه ما شامل دو بخش سفید و خاکستری بود، که هر کدام به صورت جداگانه بررسی شدند. سپس بخش مشترکی را توسعه دادیم تا یافته‌ها را از هر دو بخش یکی کنیم.

        \subsection{هدف‌ها و پرسش‌های تحقیق}

            به طور خاص، ما از روش Goal-Question-Metric [36] استفاده کرده‌ایم. اهداف ما شامل بررسی دقیق منظر فعلی تحقیقات ML4RE می‌شود. علاوه بر این، ما به بررسی نیازهای عملی در مهندسی نیازها و روش‌های ML که از ادبیات خاکستری استخراج شده‌اند علاقه‌مندیم. در نهایت، ما تمرکز داریم بر مقایسه تمرکز تحقیقات دانشگاهی در دامنه مشکل و روش‌های استفاده شده در دامنه حل مسئله با شیوه‌های عملی در Stack Overflow. این اهداف منجر به فرمول‌بندی پرسش‌های تحقیقی (RQ) زیر می‌شود:

            \begin{itemize}
            \item RQ1: وضعیت فعلی تحقیقات دانشگاهی در ML4RE چیست؟
            \item RQ2: وضعیت فعلی کاربردهای ML4RE در Stack Overflow چیست؟
            \item RQ3: چگونه دیدگاه‌های Stack Overflow درباره ML4RE با یافته‌های تحقیقات دانشگاهی همخوانی دارند یا از آنها متفاوت هستند؟
            \end{itemize}

            برای پاسخ به این RQها، ما یک سری برچسب‌ها به عنوان معیارها تعریف می‌کنیم تا داده‌ها یا اطلاعاتی که باید از مقالات استخراج شود را مشخص کنیم. در بخش‌های بعدی، جزئیات فرآیند جستجوی ادبیات برای بخش‌های سفید و خاکستری را شرح می‌دهیم. همچنین معیارهای استفاده شده برای پاسخ به RQها و فرآیند استخراج و ترکیب داده را توضیح می‌دهیم.

        \subsection{فرآیند انتخاب بخش سفید}

            Fig. 1 نشان دهنده روند فرآیند انتخاب در بخش سفید است، و تعداد مقالات استخراج شده در هر مرحله نیز در شکل آمده است. زیربخش‌های زیر جزئیات هر مرحله را ارائه می‌دهند.

            \begin{figure}
                \centering
                \includegraphics[width=0.9\textwidth]{Image/fig-1.jpg}
                \includegraphics[width=0.9\textwidth]{Image/table-1.png}
            \end{figure}

            \subsubsection{جستجوی سیستماتیک}

                ما جستجوی سیستماتیک بخش سفید خود را با استفاده از رشته جستجوی پیش‌تعریف شده در چهار پایگاه داده علمی انجام می‌دهیم. فرآیند جستجوی سیستماتیک در بخش سفید شامل سه مرحله است. ما هر مرحله را به ترتیبی که در شکل ۱ نشان داده شده است، توضیح می‌دهیم.

                انتخاب منابع جستجو. منابع جستجو می‌توانند به طور قابل توجهی کیفیت بررسی سیستماتیک ادبیات را تحت تأثیر قرار دهند. همانطور که توسط بسیاری از مطالعات پیشنهاد شده است [35،37]، منابع جستجو باید شامل چندین پایگاه داده معتبر باشند. به طور خاص، ما چهار پایگاه داده علمی را انتخاب کرده‌ایم: IEEE Xplore، ACM Digital Library، SpringerLink و Scopus. این چهار پایگاه داده می‌توانند بیشتر از ادبیات مرتبط با مهندسی نیازها را پوشش دهند [38].

                تعریف رشته جستجو. موضوع اصلی مطالعه ما یادگیری ماشین برای مهندسی نیازها است. ما رشته جستجو را به دو بخش تقسیم کرده‌ایم: بخش مهندسی نیازها و بخش تکنیک‌های یادگیری ماشین. این دو بخش با استفاده از عملگر 'AND' به هم متصل شده‌اند. به طور خاص، هر بخش شامل اصطلاحات کلیدی مرتبط است و اصطلاحات کلیدی در هر بخش با استفاده از عملگر 'OR' به هم متصل شده‌اند. برای اطمینان از پوشش جستجوی جامع و کاهش تلاش‌های انتخاب دستی، چندین دوره انجام شد تا کلمات کلیدی تعیین و بهبود یابند. جزئیات کامل رشته‌های جستجو در جدول ۱ توضیح داده شده است. علاوه بر این، جستجوی snowballing در بخش ۳.۲.۵ برای گسترش دامنه جستجو به کار گرفته می‌شود.

                حذف تکرار. ما برای پیدا کردن مقالاتی که عنوان، چکیده یا کلمات کلیدی آن‌ها شامل رشته جستجوی پیش‌تعریف شده در چهار پایگاه داده هستند و محدوده سال انتشار را بین سال‌های ۲۰۱۰ تا ۲۰۲۲ قرار داده‌ایم. مهم باید توجه داشت که SpringerLink فقط جستجوی متن کامل را پشتیبانی می‌کند، که نیازمند بررسی‌های اضافی در نتایج جستجوی این پایگاه داده است. به طور خاص، ما از توابع Google Sheets برای ادغام عنوان، چکیده و کلمات کلیدی هر مقاله استفاده کردیم و آن‌ها را با استفاده از عبارات منظم پیش‌تعریف شده با یکدیگر مطابقت دادیم. این فرآیند تعداد مقالات از ۷۳۶۷ مقاله در SpringerLink به ۲۸۲ مقاله کاهش داد. سپس، ما ۳۸۷ مقاله از IEEE Xplore، ۷۴ مقاله از ACM Digital Library، ۲۸۲ مقاله از SpringerLink و ۱۹۷۲ مقاله از Scopus به دست آوردیم. پس از حذف تمام تکرارها، تعداد ۱۸۴۵ مقاله باقی مانده است.
            
            \subsubsection{فیلتر سیستماتیک}

                نتایج اولیه جستجو از موتورهای جستجوی علمی شامل موارد زیادی است که نمی‌خواهیم مانند سخنرانی‌های اصلی، پوسترها یا مقالاتی که بسیار کوتاه هستند. علاوه بر این، موتورهای جستجو مختلف جستجوی ما را در مقیاس‌های مختلف انجام می‌دهند، اگرچه تلاش کرده‌ایم جستجوی ما را به فیلدهای عنوان، چکیده و کلمات کلیدی محدود کنیم. بنابراین، سه معیار پیش از حذف (PEC) تعریف می‌کنیم تا این نوع مقالات را از بین ببریم و بار کاری بعدی را کاهش دهیم. به دلیل استفاده از Google Sheets برای ثبت مقالات، این معیارهای انحصاری ساده از طریق توابع به صورت خودکار قابل اعمال هستند. ارزش ذکر دارد که برای PEC3 از تطبیق regex برای عنوان، چکیده و کلمات کلیدی هر مقاله با رشته جستجوی تعریف شده استفاده کردیم. پس از فیلترینگ سیستماتیک، 1107 مقاله باقی مانده است.

                \begin{itemize}
                    \item PEC1: طول مقاله کمتر از شش صفحه است.
                    \item PEC2: مقاله به زبان انگلیسی نیست.
                    \item PEC3: محتوای ترکیب شده عنوان، چکیده و کلمات کلیدی مقاله با رشته جستجوی تعریف شده مطابقت ندارد.
                \end{itemize}
            
            \subsubsection{انتخاب سه‌گانه}

                تمام مقالاتی که پس از فیلترینگ خودکار باقی مانده‌اند، سپس به صورت دستی بررسی می‌شوند تا از لحاظ ارتباط با موضوع مناسب بودند. معیار کلی انتخابی که ما تعریف کرده‌ایم این است که مقاله باید یک روش را پیشنهاد دهد که به طور اصلی از یادگیری ماشین برای کمک به وظایف مرتبط با مهندسی نیازها استفاده می‌کند.

                معیارهای اضافه و حذف. برای بهبود اجرای معیار انتخابی، آن را به سری معیارهای اضافه و حذف (IC و EC) که در جدول ۲ نشان داده شده است، تجزیه و تحلیل کرده‌ایم. اگر مقاله با هریک از معیارهای حذف مطابقت داشته باشد، آن را حذف می‌کنیم؛ و تنها در صورتی که تمامی معیارهای اضافه را برآورده کند، آن را در می‌آوریم.
                
                پروتکل انتخاب. دو نویسنده هر مقاله را از نظر معیارهای اضافه و حذف مورد بررسی قرار دادند. ابتدا مطالعه آزمایشی را بر روی ۴۰ مقاله به صورت تصادفی انجام دادیم. اگر نظرات مغایر وجود داشت، دو نویسنده نگرانی‌های خود را صریحاً بیان کرده و آن‌ها را حل کردند. از این مطالعه آزمایشی برای بهبود معیارهای ما استفاده کردیم و اجازه دادیم تا دو نویسنده بر جزییات عملی موافق شوند. پس از رسیدن به توافق در انتخاب مقالات، دو نویسنده مقالات باقیمانده را انتخاب کردند. اگر تضادی وجود داشت که دو نویسنده نتوانستند حل کنند، یک پژوهشگر سوم برای بررسی اضافی وارد می‌شد.
                
                فرآیند انتخاب. همانطور که در شکل ۱ نشان داده شده است، ما فرآیند بازبینی دستی را به سه دور تقسیم کرده‌ایم. در دور ۱، بازبینان فقط عنوان و چکیده هر مقاله را می‌خوانند تا تصمیم بگیرند که آن را حذف یا حفظ کنند. تمام مقالات باقیمانده را به دور بعد منتقل می‌کنیم. در دور ۲، بازبینان هم عنوان و هم چکیده، مقدمه و نتیجه‌گیری هر مقاله را می‌خوانند. همانطور که قبلاً بود، این مقالات را تا دور بعد حفظ می‌کنیم، به جز مقالاتی که حذف شده‌اند. در دور ۳، بازبینان نیاز دارند که متن کامل را بخوانند و سپس تصمیم نهایی را بگیرند. پس از انجام سه دور انتخاب دستی، ما ۲۰۳ مقاله را انتخاب کردیم.
            
            \subsubsection{افزایش دانش حوزه}

                در سال‌های اخیر، یادگیری ماشین به عنوان یک تکنیک گسترده، کاربردهای فراوانی در مهندسی نیازها داشته است. در طول این سال‌ها، تمرکز تحقیقات ما بر روی حوزه یادگیری ماشین برای مهندسی نیازها بوده است. به طور خاص، ما مقالات مربوطه را از جلسات معمولی مهندسی نیازها از جمله مجلات RE، REJ، REFSQ و AIRE از سال 2010 به بعد جمع‌آوری و خلاصه نموده‌ایم. ما این مقالات را از بین 1845 مقاله انتخاب نموده و به یک فرآیند انتخاب سه مرحله‌ای تحت آنها تحت‌پرسی کرده‌ایم. پس از این مرحله، ما مجموعاً 217 مقاله را جمع‌آوری کرده‌ایم، که شامل 14 مقاله جدید است.

                \begin{figure}
                    \centering
                    \includegraphics[width=0.9\textwidth]{Image/table-2.png}
                \end{figure}
            
            \subsubsection{جستجوی برفی}

                برای افزایش جامعیت مطالعه خود، ما جستجوی برفی را با اسکن مراجع مقالات انتخاب شده از طریق روش قبلی انجام دادیم.

                استخراج مراجع: ما حدود ۹۰۰۰ مرجع را از تمامی ۲۱۷ مقاله در اولین دور جستجوی برفی خود استخراج کردیم.
                
                فیلتر مراجع: ابتدا ما به صورت خودکار و سیستماتیک ۹۰۰۰ مرجع را فیلتر کردیم. به طور خاص، ما مراجعی را که با رشته جستجوی مشخص شده در جدول ۱ همخوانی داشتند را انتخاب کردیم. پس از آن، ۲۴۴ مرجع انتخاب شد. پیش از استخراج عنوان مقالات از این مراجع به صورت دستی، تمامی مقالات قبل از سال ۲۰۱۰ و کمتر از شش صفحه را حذف کردیم. سپس مراجع تکراری را حذف نمودیم و ۱۰ مورد یکتا به عنوان نامزدهای مراحل بعدی باقی ماندند.
                
                انتخاب سه مرحله‌ای: این روند انتخاب سه مرحله‌ای همانند بخش ۳.۲.۳ است. پس از سه مرحله انتخاب، یک مقاله به تنهایی انتخاب شد و جستجوی برفی مکرر در این مقاله هیچ مقاله جدیدی تولید نکرد. به عبارت دیگر، پس از جستجوی برفی، ما به جمع کل ۲۱۸ مقاله رسیدیم.
            
            \subsubsection{شناسایی و فیلتر کارهای گسترده‌تر}

                ما متوجه شدیم که در 218 مقاله، 11 جفت ارتباط گسترده‌تری دارند. ما مقالات گسترده‌تر را به عنوان نتیجه‌ای از این کارهای تحقیقاتی در نظر گرفتیم. بنابراین، مقالات قبل از گسترش را از مجموعه ما حذف کردیم. در نهایت، در بخش سفید، مجموعه داده نهایی ما شامل 207 مقاله است. ما این مقالات را با شماره‌های آنها در تحلیل بعدی شناسایی می‌کنیم.
        
        \subsection{فرایند انتخاب بخش خاکستری}

            شکل ۲ روند فرآیند انتخاب در بخش خاکستری را نشان می‌دهد و تعداد سوال‌های به دست آمده در هر مرحله نیز در شکل قید شده است. در زیربخش‌های بعدی جزئیات هر یک از این مراحل آورده شده است.
        
                \subsubsection{انتخاب منبع جستجو}

                    بر اساس دسته‌بندی گاروسی و همکاران [39]، Stack Overflow به عنوان یک منبع ادبیات خاکستری از رده دوم با اعتبار متوسط در نظر گرفته می‌شود که امکان تحلیل کمّی را فراهم می‌آورد. ما Stack Overflow را به عنوان منبع داده در بخش خاکستری انتخاب کردیم به دلیل طبیعت قابل دسترس و قابل تحلیل آن.

                    \begin{figure}
                        \centering
                        \includegraphics[width=0.9\textwidth]{Image/table-3.png}
                    \end{figure}

                \subsubsection{فیلترینگ بر اساس برچسب‌ها}
                    ما مجموعاً 23,758,292 سوال در Stack Overflow پیدا کردیم. برای شناسایی و فیلتر کردن سوال‌های مرتبط با مهندسی نیازمندی‌ها، از برچسب‌ها به عنوان ابزارهای دسته‌بندی استفاده کردیم. به طور خاص، شش برچسب زیر را انتخاب کردیم: requirements، user-stories، system-requirements، requirements-management، minimum-requirements، و REtags and MLtags که جزئیات محتوای REtags و MLtags در جدول ۳ ذکر شده است. سپس از API Stack Overflow استفاده کردیم تا تمام سوالاتی که دارای هر کدام از این برچسب‌ها از سال ۲۰۱۰ تا ۲۰۲۲ بودند، استخراج کنیم. این تلاش به تعداد سوال‌های متناظر زیر منجر شد: ۵۳۰، ۱۸۳، ۶۲، ۳۱، ۲۲ و ۷۶.

                    حذف تکرار: در Stack Overflow، سوالات ممکن است با چندین برچسب مرتبط باشند و ممکن است بیش از یکی از برچسب‌های انتخاب شده ما را شامل شوند. پس از حذف تکرارها، تعداد ۸۶۱ سوال باقی مانده است.

                \subsubsection{انتخاب دستی}

                    تمرکز ما بر روی چالش‌های عملی مواجهه شده در مهندسی نیازمندی‌ها و استخراج نیازمندی‌های عملی از آن‌ها است. بنابراین، تمامی 861 سوال که با موفقیت از فیلتر کردن بر اساس برچسب‌ها گذر کرده‌اند، به صورت دستی بررسی شده‌اند از نظر ارتباط با مهندسی نیازمندی‌ها. دو نویسنده هر سوال را بر اساس معیار انتخابی ارزیابی کرده‌اند. یک مطالعه پیلوت اولیه شامل ۵۰ سوال به صورت تصادفی انتخاب شده انجام شد تا نظرات متضاد را مورد بررسی قرار دهد و به نویسندگان کمک کند تا به توافق برسند. سپس، دو نویسنده سایر سوالات را ارزیابی کردند. در نهایت، 375 سوال برای بخش خاکستری انتخاب شدند.
        
        \subsection{استخراج داده و ترکیب آنها}

            این بخش شرح فرآیندهای استخراج داده و سنتز آنها برای بخش‌های سفید، خاکستری و مشترک را ارائه می‌دهد. ما یک مجموعه از برچسب‌ها را شناسایی کردیم که اطلاعاتی که باید از نتایج جستجو استخراج شود به عنوان معیارها عمل می‌کنند، با هدف پاسخگویی به سوالات تحقیقاتی براساس اطلاعات به دست آمده.

            
            \begin{figure}
                \centering
                \includegraphics[width=0.9\textwidth]{Image/fig-2.jpg}
            \end{figure}

                \subsubsection{استخراج و ترکیب داده‌های سفید}

                    برای پاسخ به سوال تحقیق شماره ۱ (RQ1)، قصد داریم چهار جنبه اطلاعات را از مقالات استخراج کنیم که هرکدام شامل مجموعه‌ای از برچسب‌ها است که در شکل ۳ نشان داده شده است. توضیحات این چهار جنبه و برچسب‌های مربوط به آنها به شرح زیر است:

                    \begin{enumerate}
                        \item اطلاعات اولیه
                        \begin{enumerate}
                            \item این جنبه به درک وضعیت ابتدایی مقالات می‌پردازد و شامل چهار برچسب زیر است:
                            \begin{enumerate}
                                \item سال‌های انتشار
                                \item انواع انتشارات
                                \item محل‌های انتشار
                                \item ارجاعات (Citations)
                            \end{enumerate}
                        \end{enumerate}
                        
                        \item حوزه مسئله:
                        \begin{itemize}
                            \item این جنبه به درک تمرکز جامعه علمی بر مهندسی نیازمندی‌ها می‌پردازد و شامل دو برچسب زیر است:
                            \begin{itemize}
                                \item فعالیت‌های مهندسی نیازمندی‌ها (RE activities)
                                \item وظایف مهندسی نیازمندی‌ها (RE tasks). برای فعالیت‌های مهندسی نیازمندی‌ها، مطابق با دانش نرم‌افزاری (SWEBOK)، هر مقاله را به یکی از پنج فعالیت زیر دسته‌بندی می‌کنیم: استخراج نیازمندی‌ها، تجزیه و تحلیل نیازمندی‌ها، مشخصه‌سازی نیازمندی‌ها، اعتبارسنجی نیازمندی‌ها و مدیریت نیازمندی‌ها. برای وظایف مهندسی نیازمندی‌ها، وظایف خاص مهندسی نیازمندی‌ها از هر مقاله استخراج می‌شود بر اساس توضیحات آن.
                            \end{itemize}
                        \end{itemize}
                        
                        \item حوزه راه‌حل:
                        \begin{itemize}
                            \item این جنبه به بررسی استفاده از روش‌های یادگیری ماشین در ML4RE (یادگیری ماشین برای مهندسی نیازمندی‌ها) می‌پردازد و شامل سه برچسب زیر است:
                            \begin{itemize}
                                \item وظایف ML (ML tasks)
                                \item تکنیک‌های ML (ML techniques)
                                \item ابزارهای ML (ML tools). برای وظایف ML، انواع وظایف ML از هر مقاله استخراج می‌شود مانند طبقه‌بندی و خوشه‌بندی. سپس، تکنیک‌ها و ابزارهای خاص ML استخراج می‌شود که در هر مقاله استفاده شده‌اند. برای قابلیت تکرار، انواع مواد قابل تکرار ارائه شده در هر مقاله مانند داده، کدهای مدل و نمایش‌های آزمایش ثبت می‌شود.
                            \end{itemize}
                        \end{itemize}
                        
                        \item داده:
                        \begin{itemize}
                            \item این جنبه به مطالعه داده‌های تجزیه و تحلیل شده در ML4RE می‌پردازد و شامل دو برچسب زیر است:
                            \begin{itemize}
                                \item انواع داده‌ها (Data types)
                                \item منابع داده (Data sources). برای انواع داده‌ها، ابتدا داده‌ها را به چهار نوع عمومی دسته‌بندی می‌کنیم و سپس داده‌های خاص بر اساس محتوای مقاله ثبت می‌شود. چهار نوع عمومی داده عبارتند از: آثار نیازمندی، نظرات کاربران، داده‌های دامنه و آثار در مهندسی نرم‌افزار (SE). برای منابع داده، آنها را به یکی از پنج دسته زیر دسته‌بندی می‌کنیم: عمومی، به دست آمده، خصوصی، اصلی و بدون ذکر. همچنین، منابع داده خاص استخراج می‌شود.
                            \end{itemize}
                        \end{itemize}
                        
                        \item مدل:
                        \begin{itemize}
                            \item این جنبه به بررسی مدل‌های استفاده شده در ML4RE می‌پردازد و شامل دو برچسب زیر است:
                            \begin{itemize}
                                \item انواع مدل‌ها (Model types)
                                \item روش‌های ارزیابی مدل‌ها (Model evaluation methods). برای انواع مدل‌ها، انواع مدل‌های استفاده شده در هر مقاله مانند شبکه‌های عصبی مصنوعی، درخت تصمیم، رگرسیون و ... ثبت می‌شود. برای روش‌های ارزیابی مدل‌ها، روش‌های مختلفی مانند دقت، بازخوانی، دقت و بازخوانی متوازن، ماتریس درهم‌ریختگی و ... مورد بررسی قرار می‌گیرد.
                            \end{itemize}
                        \end{itemize}
                    \end{enumerate}
                    

                    این جنبه‌ها و برچسب‌ها به عنوان معیارها برای تجزیه و تحلیل سیستماتیک اطلاعات به دست آمده از مقالات استفاده می‌شوند تا به پاسخگویی به سوالات تحقیقی بپردازیم.


                    \begin{figure}
                        \centering
                        \includegraphics[width=0.5\textwidth]{Image/fig-3.jpg}
                    \end{figure}

                \subsubsection{استخراج و سنتز داده‌های خاکستری}

                    برای پاسخ به سوال تحقیق شماره ۲ (RQ2)، طراحی شده است که چهار جنبه اطلاعات را از سوالات و پاسخ‌ها استخراج کنیم، هرکدام شامل مجموعه‌ای از برچسب‌ها هستند که در شکل ۴ نشان داده شده است. توضیحات این چهار جنبه و برچسب‌های مربوط به آنها به شرح زیر است:

                    \begin{enumerate}
                        \item اطلاعات اولیه:
                        \begin{itemize}
                            \item این جنبه به درک وضعیت ابتدایی سوالات می‌پردازد، شامل چهار برچسب زیر است:
                            \begin{itemize}
                                \item سال‌های ارسال
                                \item انواع سوالات
                                \item وضعیت سوال
                                \item تعداد پاسخ‌ها. برای انواع سوالات، بر اساس محتوا، سوالات را به پنج نوع دسته‌بندی می‌کنیم: تعریف‌های مفهومی، روش‌های عملی در عمل، مشکلات خاص در عمل مهندسی نیازمندی‌ها، مشکلات در استفاده از ابزار و نیاز به راه‌حل‌ها. برای وضعیت سوال، ثبت می‌کنیم که آیا پرسش کننده سوال پاسخ را قبول کرده است یا خیر.
                            \end{itemize}
                        \end{itemize}
                        
                        \item حوزه مسئله:
                        \begin{itemize}
                            \item این جنبه به مطالعه چالش‌های عملی مهندسی نیازمندی‌ها که در جامعه Stack Overflow مواجه می‌شوند می‌پردازد، شامل دو برچسب زیر است:
                            \begin{itemize}
                                \item فعالیت‌های مهندسی نیازمندی‌ها (RE activities)
                                \item وظایف مهندسی نیازمندی‌ها (RE tasks). مشابه بخش سفید، هر سوال را بر اساس SWEBOK به یکی از پنج فعالیت مهندسی نیازمندی‌ها دسته‌بندی می‌کنیم. علاوه بر این، وظایف شناسایی شده در بخش سفید را در زمان تعیین وظایف RE به سوالات در نظر می‌گیریم.
                            \end{itemize}
                        \end{itemize}
                        
                        \item حوزه راه‌حل:
                        \begin{itemize}
                            \item این جنبه به بررسی استفاده از ابزارها در جامعه Stack Overflow می‌پردازد، شامل دو برچسب زیر است:
                            \begin{itemize}
                                \item ابزارها (Tools)
                                \item یادگیری ماشین (ML). ما ابزارهای مهندسی نیازمندی‌ها مطرح شده در سوالات و پاسخ‌ها را جداگانه ثبت می‌کنیم. برای ML، اطلاعات مربوط به ML را بر اساس فرآیند در بخش سفید استخراج می‌کنیم.
                            \end{itemize}
                        \end{itemize}
                        
                        \item داده:
                        \begin{itemize}
                            \item این جنبه به مطالعه داده‌های تجزیه و تحلیل شده در جامعه Stack Overflow می‌پردازد، شامل تنها یک برچسب زیر است:
                            \begin{itemize}
                                \item انواع داده‌ها (Data types). ابتدا انواع خاص داده‌های پردازش شده در هر سوال را ثبت می‌کنیم. سپس، همه انواع داده‌های خاص را به دسته‌های داده‌های کلی‌تری که در بخش سفید استفاده شده‌اند، گروه‌بندی می‌کنیم.
                            \end{itemize}
                        \end{itemize}
                    \end{enumerate}

                    \begin{figure}
                        \centering
                        \includegraphics[width=0.5\textwidth]{Image/fig-4.jpg}
                    \end{figure}
                
            
            \subsubsection{ترکیب و تحلیل داده‌ها به صورت مشترک}

                برای پاسخ به سوال تحقیق شماره ۳ (RQ3)، قصد داریم چهار جنبه اطلاعات را از بخش سفید و بخش خاکستری مقایسه کنیم، هرکدام شامل مجموعه‌ای از برچسب‌ها که در شکل ۵ نشان داده شده است. توضیحات این چهار جنبه و برچسب‌های مربوط به آنها به شرح زیر است:

                \begin{enumerate}
                    \item اطلاعات اولیه:
                    \begin{itemize}
                        \item این جنبه برای مقایسه وضعیت ابتدایی بین تحقیقات علمی و کاربردهای عملی در Stack Overflow طراحی شده است، شامل یک برچسب زیر است:
                        \begin{itemize}
                            \item Trend (روند). به طور خاص، روند برای تمایز دادن تمرکز دانشگاهی و عملیاتی در Stack Overflow با مقایسه سال‌های انتشار در بخش سفید و سال‌های ارسال در بخش خاکستری مورد استفاده قرار می‌گیرد.
                        \end{itemize}
                    \end{itemize}
                    
                    \item حوزه مسئله:
                    \begin{itemize}
                        \item این جنبه برای مقایسه تاکید بر مهندسی نیازمندی‌ها بین دو بخش طراحی شده است، شامل دو برچسب زیر است:
                        \begin{itemize}
                            \item فعالیت‌های مهندسی نیازمندی‌ها (RE activities)
                            \item وظایف مهندسی نیازمندی‌ها (RE tasks). ما مقایسه‌ای بر اساس فعالیت‌های وظایف RE استخراج شده در بخش سفید و خاکستری انجام می‌دهیم.
                        \end{itemize}
                    \end{itemize}
                    
                    \item حوزه راه‌حل:
                    \begin{itemize}
                        \item این جنبه برای مقایسه روش‌های استفاده شده توسط دانشگاه و عملیات در Stack Overflow در حل مشکلات مهندسی نیازمندی‌ها طراحی شده است، شامل یک برچسب زیر است:
                        \begin{itemize}
                            \item ابزارها و ML (Tools\&ML). ما این مقایسه را بر اساس تکنیک‌های ML استخراج شده در بخش سفید در مقابل ابزارها و ML در بخش خاکستری انجام می‌دهیم.
                        \end{itemize}
                    \end{itemize}
                    
                    \item داده:
                    \begin{itemize}
                        \item این جنبه برای مقایسه داده‌های مورد بررسی در تحقیقات علمی با انواع داده‌های مورد بحث و تجزیه و تحلیل در جامعه Stack Overflow طراحی شده است، شامل یک برچسب زیر است:
                        \begin{itemize}
                            \item انواع داده‌ها (Data types). مقایسه بر اساس انواع داده استخراج شده در بخش سفید و خاکستری صورت می‌پذیرد.
                        \end{itemize}
                    \end{itemize}
                \end{enumerate}
                

                مانند انتخاب دستی قبلی، دو پژوهشگر برای هر مقاله به منظور انجام پردازش داده تعیین شده‌اند تا اعتبار پژوهش‌ها تضمین شود.

                \begin{figure}
                    \centering
                    \includegraphics[width=0.5\textwidth]{Image/fig-5.jpg}
                \end{figure}
            

                    


\end{document}