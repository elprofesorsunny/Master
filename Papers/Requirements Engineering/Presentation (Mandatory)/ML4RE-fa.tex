\documentclass[a4paper,10pt]{article}

\usepackage{tabularx}
\usepackage{graphicx}
\usepackage{geometry}
\geometry{
    top=2cm,
    bottom=2cm,
    left=1cm,
    right=1cm
}

\usepackage{xepersian}
\settextfont{Vazirmatn-Regular.ttf}

\title{ML4RE - یادگیری ماشین برای مهندسی نیازمندی‌ها}
\author{}
\date{}

\linespread{1.5}

\begin{document}

    \maketitle

    % MARK: abstract

    \begin{abstract}
        
        مقدمه: تحقیقات در زمینه یادگیری ماشین برای مهندسی نیازمندی‌ها (ML4RE) به تدریج توجه بیشتری از سوی محققان و عملی‌کنندگان به خود جلب کرده است. اگرچه تحقیقات پیشگامانه پتانسیل استفاده از تکنیک‌های یادگیری ماشین برای بهبود فرآیندهای مهندسی نیازمندی‌ها را نشان داده‌اند، اما یک مرور نظام‌مند و جامع از ادبیات علمی که دیدگاه صنعتی را نیز در بر گیرد، در دانشگاه‌ها وجود ندارد. به‌ویژه، هیچ‌یک از مرورهای موجود در زمینه ML4RE به ادبیات خاکستری که عمدتاً از منابع عملی‌کنندگان منشأ می‌گیرد و بازتاب‌دهنده مسائل و چالش‌های واقعی در عمل است، توجه نکرده‌اند.

        هدف: در این مقاله، ما یک بررسی نظام‌مند از انتشارات علمی در زمینه ML4RE انجام می‌دهیم و آن را با نظرات عملی‌کنندگان از Stack Overflow تکمیل می‌کنیم تا یک مرور جامع از ادبیات ارائه دهیم. هدف تحقیق ما ارائه یک دیدگاه جامع از پیشرفت‌های کنونی در تحقیقات ML4RE، بیان سوالات و چالش‌های اصلی در عمل مهندسی نیازمندی‌ها، درک فاصله بین تحقیق و عمل، و ارائه بینش‌های خود درباره چگونگی توسعه عملی این حوزه دانشگاهی در آینده است.

        روش: ما به صورت نظام‌مند 207 مقاله علمی در زمینه ML4RE از سال 2010 تا 2022 را بررسی کردیم و همچنین 375 سوال مرتبط با مهندسی نیازمندی‌ها در Stack Overflow و پاسخ‌های مربوطه را تحلیل کردیم. تحلیل ما شامل روندها، فعالیت‌ها و وظایف متمرکز بر مهندسی نیازمندی‌ها، راه‌حل‌های به‌کاررفته و داده‌های مرتبط بود. در نهایت، یک تحلیل مشترک انجام دادیم و نتایج هر دو بخش را با هم مقایسه کردیم.
        
        نتایج: بر اساس نتایج آماری از ادبیات جمع‌آوری‌شده، ما یک نقشه راه علمی را خلاصه کرده و تفاوت‌ها را تحلیل کردیم و توصیه‌های پژوهشی ارائه دادیم. پیشنهادات ما شامل توسعه دستیاران هوشمند پاسخگویی به سوالات با استفاده از مدل‌های زبان بزرگ، ادغام یادگیری ماشین در ابزارهای صنعتی و ترویج همکاری بین دانشگاه و صنعت است.

        نتیجه‌گیری: این مطالعه با ارائه یک دیدگاه جامع از ML4RE، بیان تفاوت‌های بین تحقیق و عمل، و پیشنهاد راه‌حل‌های عملی برای پر کردن شکاف بین دانشگاه و صنعت، به پیشرفت این حوزه کمک می‌کند.

    \end{abstract}

    % MARK: Introduction
    
    \section{مقدمه}

    مهندسی نیازمندی‌ها (RE) یک مرحله اساسی در مراحل اولیه مهندسی نرم‌افزار (SE) است. اگرچه پژوهشگران به طور مستمر در حال بررسی روش‌ها و تکنیک‌هایی برای تسهیل فرآیندهای نیازمندی هستند، اما کل فرآیند مهندسی نیازمندی‌ها همچنان نیاز به تلاش دستی زیادی دارد (مثلاً استخراج نیازمندی‌های ذی‌نفعان از طریق مصاحبه یا طبقه‌بندی نیازمندی‌ها بر اساس یک طبقه‌بندی خاص). دلیل اصلی این موضوع این است که فعالیت‌های RE معمولاً نیاز به دانش عمیق حوزه و مهارت‌های تحلیل پیشرفته دارند که به طور کامل قابل اتوماسیون نیست.

    در سال‌های اخیر، توسعه سریع فناوری یادگیری ماشین (ML) با بهبود قدرت محاسباتی تحریک شده است. کاربردهای موفق ML در زمینه‌هایی مانند پردازش زبان طبیعی، شناسایی تصویر و داده‌کاوی فرصت‌هایی را برای استفاده از تکنیک‌های ML در زمینه RE فراهم کرده است. استفاده از فناوری ML در RE یک رویکرد هوشمندانه‌تر و کارآمدتر برای مدیریت داده‌های نیازمندی‌ها ارائه می‌دهد. به عنوان مثال، ML می‌تواند در طبقه‌بندی خودکار نیازمندی‌ها کمک کند زیرا می‌تواند اطلاعات بالقوه نیازمندی‌ها را خلاصه کند.

    علاوه بر این، با توسعه سریع تکنیک‌های اطلاعاتی، کار و زندگی روزمره ما دیجیتالی می‌شوند. در نتیجه، داده‌های مرتبط با نیازمندی‌ها بیشتر و بیشتر دیجیتالی و به‌طور عمومی در دسترس قرار می‌گیرند، که پژوهش در زمینه یادگیری ماشین برای مهندسی نیازمندی‌ها (ML4RE) را ترویج می‌کند. به عنوان مثال، بررسی‌های کاربران از برنامه‌های موبایلی به طور گسترده‌ای برای استخراج نیازمندی‌های کاربران مورد بررسی قرار گرفته‌اند.

    تحقیقات قبلی ML4RE را مورد بررسی قرار داده‌اند. اقبال و همکاران [2] یک بررسی برای به‌دست‌آوردن نمای کلی از چگونگی کمک تکنیک‌های ML به فعالیت‌های RE انجام دادند. علاوه بر این، زمانی و همکاران [3] یک مطالعه نگاشت از کاربردهای ML در RE انجام دادند و 65 مقاله را برای ارزیابی اثربخشی ML در اتوماسیون وظایف RE تحلیل کردند. کارهای آن‌ها بر کل فرآیند RE متمرکز بود و نحوه تأثیرگذاری و تسهیل تکنیک‌های ML در مراحل مختلف را روشن کردند.

    به علاوه، برخی تحقیقات به فعالیت‌ها یا وظایف خاص RE می‌پردازند. به عنوان مثال، لیم و همکاران [4] رویکردهای پیشرفته فعلی برای استخراج نیازمندی‌های مبتنی بر داده از منابع داده پویا را بررسی کردند. ما متوجه شدیم که این مطالعات عمدتاً بر انتشارات علمی متمرکز بوده و از ادغام بینش‌های حاصل از منابع ادبیات خاکستری، مانند وبلاگ‌ها و انجمن‌های صنعتی غافل بوده‌اند.

    بر خلاف انتشارات علمی که عمدتاً توسط پژوهشگران منتشر می‌شوند، ادبیات خاکستری به‌طور مداوم توسط عملی‌کنندگان تولید می‌شود و بر "وضعیت عمل" نور می‌تاباند [5]. همان‌طور که در [6] اشاره شده، ادغام ادبیات خاکستری در مرورهای نظام‌مند ادبیات می‌تواند فاصله بین پژوهش‌های علمی و عملی را پر کند و دیدگاه جامع‌تری از چالش‌ها و راه‌حل‌ها ارائه دهد.

    اگرچه تعداد مرورهای نظام‌مند ادبیات که ادبیات خاکستری را در مطالعات SE در نظر گرفته‌اند در حال افزایش است [7,8]، اما در RE به اندازه کافی رایج نیستند. برای پر کردن این شکاف در زمینه RE، این مقاله قصد دارد یک مرور نظام‌مند از ادبیات در زمینه ML4RE انجام دهد که با بینش‌های حاصل از ادبیات خاکستری منابع شده از Stack Overflow تکمیل شود.

    هدف این مرور ادبیات سه بخشی است. بخش سفید شامل مرور 207 مقاله منتشر شده بین سال‌های 2010 تا 2022 است که به‌طور خاص بر ML4RE متمرکز است. در همین حال، بخش خاکستری شامل تحلیل 375 سوال و پاسخ‌های مربوطه جمع‌آوری شده از مباحث Stack Overflow درباره فعالیت‌های RE در همان دوره است. در نهایت، تحلیل مشترک ما شامل مقایسه نتایج این دو بخش برای تشخیص شباهت‌ها و تفاوت‌های آن‌ها است. ما روندها، فعالیت‌های RE، وظایف RE، راه‌حل‌ها و داده‌های موجود در ادبیات را تحلیل می‌کنیم.

    نتایج تحقیق نشان می‌دهد که هر دو بخش به تحلیل RE و مستندسازی نیازمندی‌ها علاقه‌مند هستند. فراتر از شباهت‌ها، بخش سفید تمایل به تمرکز بر استخراج نیازمندی‌ها دارد، در حالی که بخش خاکستری بیشتر بر مدیریت نیازمندی‌ها تأکید دارد. بخش سفید استفاده از تکنولوژی‌های ML مانند SVM، CNN و BERT را برجسته می‌کند. در مقابل، بخش خاکستری بیشتر بر ابزارهایی مانند Microsoft TFS، Jira و IBM Rational DOORS و تکنیک‌های ML مانند LDA، POS، TF-IDF و شبکه‌های عصبی تکیه دارد. علاوه بر این، بخش خاکستری توجه ویژه‌ای به داستان کاربر و مورد استفاده دارد که در بخش سفید نسبتاً کمتر مورد بررسی قرار گرفته است.

    بر اساس این یافته‌ها، ما یک نقشه راه علمی خلاصه کرده و تحلیل دقیقی از تفاوت‌های بین بخش سفید و خاکستری ارائه می‌دهیم. سپس پیشنهادات پژوهشی ارائه می‌دهیم، از جمله توسعه دستیاران هوشمند پاسخگویی به سوالات با استفاده از مدل‌های زبان بزرگ و ادغام یادگیری ماشین در ابزارهای صنعتی. همچنین، همکاری بیشتر بین دانشگاه و صنعت را برای درک عمیق‌تر مشکلات پژوهشی واقعی و داده‌ها تشویق می‌کنیم.

    در خلاصه، این مقاله چهار کمک اصلی را ارائه می‌دهد. اولاً، یک نمای جامع از وضعیت فعلی پژوهش‌های ML4RE ارائه می‌دهیم. دوماً، شرایط واقعی عملی‌کنندگان RE را از طریق ادبیات خاکستری حاصل از Stack Overflow بررسی می‌کنیم. سوماً، فاصله بین پژوهش و عمل در حوزه ML4RE را به‌ویژه در زمینه‌هایی که کمتر مورد توجه پژوهشگران قرار گرفته‌اند، برجسته می‌کنیم. و در نهایت، برای پر کردن فاصله بین صنعت و دانشگاه، پیشنهادات پژوهشی عملی در ML4RE ارائه می‌دهیم.

    در بخش‌های باقی‌مانده این مقاله، کارهای مرتبط را در بخش 2 ارائه می‌دهیم. بخش 3 پروتکل تحقیق برای مرور نظام‌مند ادبیات ما را ارائه می‌دهد. در سه بخش بعدی، نتایج این بررسی و پاسخ به سوالات پژوهشی را ارائه می‌دهیم. بخش 4 نتایج بخش سفید، بخش 5 بر بخش خاکستری تمرکز می‌کند و بخش 6 نتایج تحلیل مشترک را ارائه می‌دهد. بر اساس نتایج، در بخش 7 به بحث پرداخته و چندین پیشنهاد ارائه می‌دهیم. بخش 8 شامل تحلیل تهدیدات به اعتبار این بررسی است. در نهایت، مقاله را در بخش 9 نتیجه‌گیری می‌کنیم.
    
    % MARK: Related work

    \section{کارهای مرتبط}

    کارهای مرتبط با بررسی ادبیات ما در حوزه یادگیری ماشین برای مهندسی نیازمندی‌ها (ML4RE) از دو منبع مختلف به دست آمده است. مجموعه اول شامل بررسی‌های ادبیات مرتبط با ML4RE می‌باشد. مجموعه دوم شامل بررسی‌های سیستماتیکی ادبیات در حوزه مهندسی نرم‌افزار (SE) است. در دو زیربخش پیش رو، جزئیات کارهای مرتبط از این دو منبع را به طور دقیق ارائه خواهیم داد.

        \subsection{بررسی‌های ادبیات در مورد یادگیری ماشین برای مهندسی نیازمندی‌ها (ML4RE):}

            در این بخش، یک مجموعه از بررسی‌های ادبیات مرتبط با یادگیری ماشین برای مهندسی نیازمندی‌ها (ML4RE) را ارائه می‌دهیم. ما چندین بررسی ادبیات را پیدا کرده‌ایم، برخی به کلیه فرآیند مهندسی نیازمندی‌ها می‌پردازند، در حالی که برخی دیگر بر روی فعالیت‌ها یا وظایف خاص مهندسی نیازمندی‌ها تمرکز دارند. بنابراین، این زیربخش به دو بخش تقسیم شده است تا این کارها را به تفصیل معرفی کند.

            \subsubsection{فرآیند کامل مهندسی نیازمندی‌ها}

                دو مقاله به کلیه فرآیند مهندسی نیازمندی‌ها متمرکز شده‌اند و یک بررسی کلی از نحوه کاربرد تکنیک‌های یادگیری ماشین در مراحل مختلف مهندسی نیازمندی‌ها ارائه داده‌اند. در مقاله Iqbal و همکاران [2]، بررسی‌ای بر روی مقالات تحقیقاتی انجام شده است تا چگونگی کمک یادگیری ماشین به مهندسی نیازمندی‌ها را بررسی کند. آن‌ها تأثیر یادگیری ماشین را در پنج مرحله اصلی مهندسی نیازمندی‌ها مشاهده کرده و مسائل خاصی که توسط یادگیری ماشین حل شده‌اند، ویژگی‌ها، الگوریتم‌های ML و مجموعه داده‌ها را بررسی کرده‌اند. با این حال، مقالاتی که آن‌ها بررسی کردند از یک فرآیند جستجوی سیستماتیک به دست نیامده بودند و بنابراین، نتیجه‌گیری‌های به دست آمده ممکن است سیستماتیک و جامع نباشد.

                Zamani و همکاران [3] یک مطالعه نگاشتی از کاربردهای یادگیری ماشین در مهندسی نیازمندی‌ها انجام دادند، با تجزیه و تحلیل 65 مقاله برای ارزیابی کارآیی یادگیری ماشین در اتوماسیون وظایف مهندسی نیازمندی. این مطالعه تکنیک‌ها، چالش‌ها، مجموعه داده‌ها و معیارهای ارزیابی این مطالعات را شناسایی می‌کند. مقایسه با بررسی جامع ما، این مقاله بیشتر بر جنبه‌های تجربی یادگیری ماشین در مهندسی نیازمندی‌ها تمرکز دارد و بینش‌های خاصی را در کارآمدی عملی ML در این زمینه ارائه می‌دهد. علاوه بر این، مقاله Zhao و همکاران [9] بر روی NLP4RE تمرکز داشتند، با تحلیل 404 مطالعه برای درک کاربرد پردازش زبان طبیعی در مهندسی نیازمندی‌ها. با توجه به تداخلات بین NLP و ML، این تحقیق را به عنوان یکی از کارهای مرتبط برای تحلیل می‌پذیریم.
            
            
            \subsubsection{بخشی از فرآیند مهندسی نیازمندی‌ها}

                در فرآیند استخراج نیازمندی‌ها، G.C. Sampada و همکاران [10] یک نگاه کلی از رویکردهای مختلف برای اتوماسیون استخراج و مشخصه‌گذاری نیازمندی‌ها در چرخه توسعه نرم‌افزار ارائه دادند. Lim و همکاران [4] وضعیت فعلی روش‌های پیشروی استخراج نیازمندی‌های مبتنی بر داده از منابع داده پویا را بررسی کردند و شکاف‌های تحقیق را شناسایی کردند. Cheligeer و همکاران [11] با انتخاب 86 مقاله، مطالعاتی را که فناوری‌های ML و NLP را در استخراج نیازمندی‌ها شامل می‌شوند، خلاصه و تحلیل کردند. آن‌ها تکنیک‌های مختلف برای ساخت روش‌های استخراج نیازمندی مبتنی بر ML را به پنج بخش دسته‌بندی کردند.

                در فرآیند طبقه‌بندی نیازمندی‌ها، Alrumaih و همکاران [12] به بررسی مطالعات تحقیقی در زمینه طبقه‌بندی نیازمندی‌ها پرداختند و محدودیت‌ها را بررسی کردند تا پیشنهادهای بهبودی ارائه دهند. Perez و همکاران [13] کاربردهای تکنیک‌های ML در طبقه‌بندی نیازمندی‌های نرم‌افزار را بر اساس 13 مقاله بررسی کردند و الگوریتم‌های طبقه‌بندی مکررترین و مجموعه داده‌های آموزشی مکررترین را خلاصه کردند. Khelifa و همکاران [14] بررسی کردند که آیا تکنیک‌های یادگیری ماشین در طبقه‌بندی نیازمندی‌ها نرم‌افزاری و طبقه‌بندی درخواست‌های تغییرات نیازمندی‌ها قابل اعمال هستند. به علاوه، Kadebu و همکاران [15] بر روی مهندسی نیازمندی‌های امنیتی تمرکز کردند و کاربردهای تکنیک‌های ML در استخراج و طبقه‌بندی نیازمندی‌های امنیتی را بررسی کردند.

                در فرآیند مدیریت نیازمندی‌ها، Xu و همکاران [16] هشت روش ML را که در مدیریت نیازمندی‌ها استفاده شده است خلاصه کردند و 18 شاخص ارزیابی برای مدیریت نیازمندی‌ها در روش ML مشخص کردند. کار آن‌ها به عنوان یک درک اولیه از گستره‌ی وسیعی از تکنیک‌های ML در مدیریت نیازمندی‌ها خدمت می‌کند، در حالی که برخی مطالعات به تفصیل به وظایف خاص می‌پردازند.

                Achimugu و همکاران [17] به بررسی تکنیک‌های اولویت‌بندی نیازمندی‌های نرم‌افزار از طریق 73 مقاله مرتبط پرداختند و چندین محدودیت در تکنیک‌های اولویت‌بندی موجود را مورد بررسی قرار دادند. به علاوه، Li و همکاران [18] با انجام یک مطالعه نگاشت سیستماتیک با 26 مطالعه، 32 فناوری ML برای پیگیری نیازمندی‌ها را خلاصه کردند. از مطالب فوق مشخص است که تعداد زیادی از بررسی‌های ادبیات عالی در زمینه ML4RE وجود دارد. با این حال، در حال حاضر، کمبودی در تحقیقات وجود دارد که ادغام بخش خاکستری را که نماینده جنبه‌های صنعتی است، در نظر بگیرد. هدف کار ما پر کردن این شکاف است با جامع نگاه داشتن به دیدگاه‌های دانشگاهی و صنعتی.
        
        \subsection{بررسی‌های ادبیات در مهندسی نرم‌افزار}

            در زمینه مهندسی نرم‌افزار، تعداد زیادی بررسی ادبیات سیستماتیک وجود دارد. ما مقالات مرتبطی را که شامل ادبیات خاکستری هستند انتخاب کرده‌ایم. آن‌ها را بر اساس سه حوزه موضوعی دسته‌بندی کرده‌ایم که هرکدام به ترتیب معرفی می‌شوند:

            \subsubsection{منابع متدولوژی‌های توسعه نرم‌افزار}

                منابع متدولوژی‌های توسعه نرم‌افزار به مدل‌ها یا سیستم‌های ارزش‌گذاری مهندسی نرم‌افزار است که توسط توسعه‌دهندگان گسترده در فرآیند توسعه نرم‌افزار پذیرفته می‌شود. متدولوژی‌های معروف در توسعه نرم‌افزار شامل Agile، DevOps و DevSecOps است که بر اساس آخرین مورد بر اساس DevOps بررسی می‌شود. França و همکاران [19] یک بررسی ادبیات انجام دادند با هدف توصیف DevOps از دیدگاه‌های مختلف. Amaro و همکاران [20] به هدف روشن‌سازی قابلیت‌های DevOps و ارتباط آن‌ها با شیوه‌های عملیاتی DevOps پرداختند. با پیشرفت DevOps، امنیت برای مهندسی نرم‌افزار اهمیت بیشتری پیدا می‌کند. DevSecOps با یکپارچگی روش‌های امنیتی مدرن و DevOps برای اجرای این امر به وجود آمده است. Myrbakken و Colomo-Palacios [8] یک بررسی ادبیات انجام دادند تا یک دید کلی از تعریف، اهمیت، مزایا و چالش‌های DevSecOps ارائه دهند. برای کیفیت پیاده‌سازی DevSecOps، Prates و همکاران [21] یک بررسی ادبیات انجام دادند تا معیارهایی که تیم‌ها می‌توانند برای اندازه‌گیری کارایی پیاده‌سازی متدولوژی DevSecOps در سازمان‌ها استفاده کنند، شناسایی کنند. 

                بیشتر و بیشتر شرکت‌های IT به معماری خدمات میکرو بازیافته تا کسب و کار خود را ارائه دهند. Soldani و همکاران [22] ادبیات خاکستری صنعتی را درباره دردها و سودهای معماری میکروسرویس‌ها به صورت سیستماتیک انتخاب و تجزیه و تحلیل کردند. Bhandari و Colomo-Palacios [23] بر روی هولاکراسی برای تیم‌های توسعه نرم‌افزار تمرکز کردند. برخی از اعمال به اینکه چگونه از ML برای کمک به DevOps در توسعه استفاده می‌شود. Recupito [24] یک بررسی ادبیات چند صداگذاری انجام داد تا ابزارهای MLOps و قابلیت‌های آنها در خودکارسازی لوله‌های یادگیری ماشین با شیوه‌های عملیاتی DevOps را بررسی کند.
            
            \subsubsection{مهندسی نرم‌افزار عمومی}

                بعضی از تحقیقات به ادغام صداهای حرفه‌ایان در حوزه گسترده‌تر مهندسی نرم‌افزار متمرکز شده‌اند. Kamei و همکاران [25] یک مطالعه سومی انجام دادند تا درکی از استفاده تحقیقات ثانویه از ادبیات خاکستری به دست آورند. با توجه به وضعیت محققان در مهندسی نرم‌افزار که هنوز با ارتباط کم تحقیقات با نیازهای حرفه‌ایان درگیر بودند، Garousi و همکاران [26] یک بررسی ادبیات انجام دادند. آن‌ها درک‌هایی از علل کم‌ارتباطی و پیشنهادهایی برای بهبود آن را به دست آوردند. Rainer و Williams [27] یک مطالعه سومی را در مورد تحقیقات به شیوه‌های عملی نرم‌افزاری با استفاده از اسناد شبیه به وبلاگ انجام دادند. Alves و همکاران [28] یک طبقه‌بندی جامع از شیوه‌های استفاده شده در صنعت برای ساخت سیستم‌های یادگیری ماشین ارائه دادند، که برای سازمان‌ها برای بهبود و مدیریت فرآیندها و شیوه‌های ML آموزنده است. Heiland و همکاران [29] یک دیدگاه کلی از الگوهای طراحی برای سیستم‌های مبتنی بر هوش مصنوعی ارائه دادند، که شامل الگوهای جدید و تطبیق‌یافته است، جمع‌آوری شده از طریق یک بررسی ادبیات چند صداگذاری.

            \subsubsection{بخش‌های خاص در مهندسی نرم‌افزار}

                بررسی‌های ادبیات در بخش‌های مختلف مهندسی نرم‌افزار وجود دارد. در حوزه آزمون نرم‌افزار، Raulamo-Jurvanen و همکاران [30] یک بررسی ادبیات خاکستری انجام دادند تا مشکلات پتانسیلی فرآیندهای موجود و فرصت‌های ارزیابی جامع ابزار را شناسایی کنند. برای خودکارسازی آزمون، Garousi و Mäntylä [7] بررسی ادبیاتی در مورد زمان و چه چیزی را باید در آزمون نرم‌افزار خودکارسازی کردند. Garousi و همکاران [31] بر روی ارزیابی رشد آزمون و بهبود فرآیند آزمون تمرکز داشتند و بررسی ادبیاتی را انجام دادند. Garousi و Küçük [32] یک نقشه‌برداری ادبیات چند صداگذاری را در مورد بوی‌های آزمون در هر دو ادبیات علمی و خاکستری انجام دادند. Felderer و Garousi [33] آزمون نرم‌افزار را در صنعت و دانشگاه مورد بررسی قرار دادند و پیشنهادات خود را درباره بهبود ارتباط و همکاری بین صنعت و دانشگاه در آزمون نرم‌افزار ارائه دادند. در حوزه مهندسی نیازها، Tripathi و همکاران [34] از بررسی ادبیات برای یافتن ادبیات علمی و خاکستری استفاده کردند. آن‌ها بررسی کردند که چگونه استارتاپ‌های نرم‌افزاری از استخراج نیاز، مستندسازی، اولویت‌بندی و اعتبارسنجی نیاز استفاده می‌کنند. به طور کلی، در منظر علمی مهندسی نرم‌افزار، توسعه‌های قابل توجهی در بررسی‌های ادبیات سیستماتیک دیده شده است، با انجام اعمال بسیار عالی که چشم‌اندازهای از ادبیات خاکستری را برای تحلیل ترکیب می‌کنند. با این حال، تحلیل ما نشان می‌دهد که از بین این اعمال، هنوز به تفکیک در ML4RE پرداخته نشده است، در حالی که کار ما این نقطه را پر می‌کند.
                
\end{document}