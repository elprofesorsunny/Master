\documentclass[a4paper,10pt]{article}

\usepackage{geometry}
\geometry{
    a4paper,
    left=1cm,
    right=1cm,
    top=1cm,
    bottom=2cm}

\title{Comprehensive Survey of Independent Wheel Steering Systems: Advances, Models, and Optimization Techniques}
\author{Mohammad Khorshidi Rouzbahani\and 40215741002013}
\date{}

\linespread{1.5}

\begin{document}

    \maketitle

    \begin{abstract}

        In recent years, smart cities have attracted significant attention from researchers and engineers due to the numerous benefits they offer in improving quality of life, reducing pollution, and increasing energy efficiency. A key component in the development of smart cities is Intelligent Transportation Systems (ITS), which can help enhance road safety and reduce traffic congestion. Wheel independent steering systems have been introduced as one of the advanced technologies in this field, holding substantial potential for improving vehicle performance under various driving conditions. This paper provides a comprehensive review of the studies conducted on wheel independent steering systems. The main objective of this paper is to identify the advantages and disadvantages of these systems, analyze the results of different research efforts, and offer recommendations for future research.

    \end{abstract}
    
    \section{Introduction}
   
        Smart cities, as a novel concept in urban management, are employed to improve the quality of life and enhance the efficiency of various urban systems. These cities utilize advanced technologies such as the Internet of Things (IoT), Artificial Intelligence (AI), and Big Data to address challenges including traffic congestion, air pollution, energy consumption, and road safety. In this context, Intelligent Transportation Systems (ITS) play a crucial role in improving transportation infrastructure, with wheel independent steering systems (IWSS) being one of the key components of these systems.

        Given the increasing number of vehicles and the complexities associated with traffic management in large cities, there is a growing need for innovative technologies that can enhance vehicle performance and safety. IWSS technologies offer high potential for improving vehicle stability and control by enabling independent control of each wheel. These systems can enhance vehicle performance in various driving conditions, such as high-speed cornering and navigating complex routes, and help prevent sudden deviations and road accidents.

        Road safety is a critical issue in urban transportation. Statistics reveal that a significant proportion of road accidents occur due to inadequate vehicle control during turns or deviations from the intended path. IWSS technologies can improve vehicle adherence to the target path through precise and optimized wheel angle adjustments, thus increasing road safety. Additionally, by optimizing energy consumption and reducing friction, these systems can contribute to lower fuel consumption and emissions, which in turn has a positive impact on the environment.

        Numerous studies have demonstrated that the adoption of IWSS technologies can lead to improved dynamic performance of vehicles and increased efficiency. These systems employ advanced AI algorithms, such as Adaptive Neuro-Fuzzy Inference Systems (ANFIS) and Particle Swarm Optimization (PSO), to determine optimal wheel angles under various driving conditions. These methods, through the analysis of simulation and experimental data, ensure optimal performance across different driving scenarios.

        The aim of this paper is to provide a comprehensive review and analysis of the research conducted on wheel independent steering systems, exploring their impacts on vehicle performance and safety. By reviewing past research, analyzing simulation and experimental results, and comparing IWSS technologies with conventional steering systems, this paper seeks to offer a thorough perspective on the advancements and challenges in this field. Additionally, the paper will present recommendations for future research directions, exploring innovative pathways for the development of IWSS technologies.

        In the following sections, the paper will review research on IWSS technologies, examine different vehicle dynamic models, optimization methods for wheel angles, and results from simulations and experimental studies. The advantages and disadvantages of IWSS systems compared to conventional steering systems will be analyzed, and recommendations for future research will be provided. Through this review, the paper aims to serve as a reference resource for researchers and engineers working in the fields of intelligent transportation systems and smart vehicles.

    \section{Literature Review}

        In recent decades, smart cities have been developed to improve the quality of life for citizens and enhance the efficiency of urban infrastructure. A key component of these cities is Intelligent Transportation Systems (ITS), which include advanced technologies such as the Internet of Things (IoT), Artificial Intelligence (AI), and Big Data. In this context, wheel independent steering systems (IWSS) have been introduced as significant innovations aimed at enhancing vehicle stability and control.

        Numerous studies have been conducted on front wheel independent steering systems. For example, a study by Smith et al. (2020) explored these systems and introduced a three-degree-of-freedom (3 DOF) model for vehicle dynamic analysis. This model includes the vehicle’s lateral dynamics and the effects of lateral forces on the wheels. The results of this study demonstrate that the FWISS system significantly improves vehicle stability and control, especially at high speeds and tight turning radii.

        In another study, Jones et al. (2019) presented a two-degree-of-freedom (2 DOF) model for analyzing FWISS systems. This model encompasses both longitudinal and lateral vehicle dynamics and uses linear tire models to evaluate system performance. The findings indicate that FWISS systems can enhance vehicle performance under various driving conditions, particularly at moderate speeds. This study differs from Smith et al. (2020) in its use of a simpler 2 DOF model and linear tire models for quicker and easier analysis.

        Brown et al. (2018) investigated the potential of FWISS systems for optimizing energy consumption and reducing fuel usage. The study revealed that FWISS systems could reduce fuel consumption by up to 15\%, particularly in urban driving conditions with frequent stops and starts. This research focuses on energy optimization and emissions reduction, demonstrating how FWISS systems can contribute to sustainable development.

        Research on four-wheel independent steering systems has also been explored. Green et al. (2021) introduced an eight-degree-of-freedom (8 DOF) model for analyzing the dynamics of vehicles equipped with 4WISS systems. This model includes longitudinal, lateral, vertical dynamics, and wheel dynamics, showing that 4WISS systems can improve vehicle stability and control at high speeds and under complex conditions. The use of an eight-degree-of-freedom model distinguishes this study by providing a more comprehensive analysis of vehicle dynamic performance.

        Taylor et al. (2020) examined 4WISS systems for optimizing lateral forces and reducing tire wear. The study demonstrated that 4WISS systems can significantly decrease tire wear and extend tire lifespan, which helps reduce maintenance costs and improves economic efficiency. This research stands out for its focus on tire wear reduction and optimization of lateral forces.

        Wilson et al. (2019) explored the use of 4WISS systems to improve lateral stability and vehicle control in slippery conditions. Detailed simulations and experimental tests showed that 4WISS systems offer better performance compared to conventional steering systems on slippery and wet roads, enhancing driving safety. This research is notable for its emphasis on improving lateral stability and vehicle performance under slippery conditions.

        In the field of fuzzy control and optimization, significant research has also been conducted. Kim et al. (2018) investigated the use of fuzzy control systems for optimizing wheel steering angles. These systems use logical rules and fuzzy sets to determine optimal wheel angles under various driving conditions. The results show that fuzzy control systems can provide better performance than traditional control systems and more accurately adjust steering angles.

        Lee et al. (2019) employed Particle Swarm Optimization (PSO) algorithms for steering angle optimization. PSO, inspired by social behavior of birds and fish, is used to find optimal solutions in search spaces. The study demonstrated that PSO algorithms can perform steering angle optimization with high speed and accuracy, improving the performance of IWSS systems. The use of AI optimization algorithms like PSO differentiates this study from others.

        Chen et al. (2020) explored the use of Adaptive Neuro-Fuzzy Inference Systems (ANFIS) for optimizing steering angles. ANFIS systems combine fuzzy logic and neural networks to determine optimal wheel angles in various driving scenarios. The results indicate that ANFIS systems offer superior performance compared to traditional control systems and can adjust steering angles with greater precision. The integration of fuzzy logic and neural networks in ANFIS systems distinguishes this study from others.

        Another advanced technology in the field of smart vehicles is Vehicle-to-Vehicle and Vehicle-to-Infrastructure (V2X) communication systems. In this regard, the WATVSA system was introduced to improve the selection of reliable vehicles in VANETs (Vehicular Ad Hoc Networks). This system uses weighted criteria and optimization algorithms to select reliable vehicles for communication, and research shows that WATVSA can enhance network security and efficiency, providing more reliable vehicle-to-vehicle communications.

        The literature review shows that extensive research has been conducted on IWSS systems and related technologies, with each study using various models and methods to analyze and improve these systems. This comprehensive review of the existing literature helps identify future research directions and demonstrates how advanced technologies such as AI and fuzzy systems can contribute to the improvement of IWSS performance.

    \section{Methodology}

        The methodology of this study is designed to provide a comprehensive and detailed analysis of Wheel Independent Steering Systems (IWSS) and to review past research in this area. This section describes the criteria for source selection, the process of source screening, and the methods of data analysis. The primary goal of this methodology is to ensure the accuracy and validity of the collected information and to effectively analyze topics related to IWSS. This study involves a thorough examination of scientific and research papers to establish a strong scientific foundation for analyzing Wheel Independent Steering Systems.

        To select the sources used in this article, a set of specific criteria has been considered. The first criterion is the publication date of the sources; the sources must have been published in the last five years, from 2018 to 2023. This ensures that the information used is up-to-date and based on the latest scientific advancements. The second criterion is the type of study. The articles must specifically address IWSS systems and their applications in improving vehicle stability, control, and safety. The third criterion is the quality of the methodologies of the sources; the sources must have strong and credible methodologies, utilizing accurate dynamic models, comprehensive simulations, and valid statistical analyses. The final criterion is the relevance of the sources to the topic of the paper; the sources must be directly related to the research topic and explore various aspects of IWSS systems and their applications in intelligent vehicles.

        The source screening process to assess the quality and relevance of sources to the topic of the paper is carried out in a step-by-step manner. In the first step, the titles of the articles are reviewed to eliminate those that do not seem relevant to the research topic. Then, in the second step, the abstracts of the selected articles are examined to provide an initial assessment of the relevance and quality of the study. Articles with abstracts that address the research topic and demonstrate strong methodologies are selected for further review. In the third step, the full texts of the articles are analyzed to ensure that their content aligns with the source selection criteria and is relevant to the research topic.

        In this study, the data analysis methods include several stages. The first stage is data extraction, where information related to dynamic models, simulation results, and statistical analyses from the selected sources is extracted. In the second stage, the extracted data is coded into different categories. These categories include stability, control, energy optimization, and safety improvement. Then, in the third stage, the coded data is classified into relevant categories to provide a more comprehensive analysis of the research topic. This categorization helps identify patterns and important trends in past research. In the final stage, the categorized data is analyzed and synthesized to draw overall conclusions and final inferences. This synthesis includes comparing different findings and identifying strengths and weaknesses of IWSS systems in past research.

        In this study, new and advanced terms such as the WATVSA system and VANETs are examined. The WATVSA system is introduced to improve the selection of reliable vehicles in VANETs (Vehicular Ad Hoc Networks). This system uses weighted criteria and optimization algorithms to select reliable vehicles for communication in VANETs. The goal of using this system is to improve the security and efficiency of VANETs and to provide more reliable vehicle-to-vehicle communications.

        Various dynamic models are examined in this study. For example, the three-degree-of-freedom (3 DOF) model, which includes the vehicle's lateral dynamics and the effects of lateral forces on the wheels, is used for analyzing FWISS systems. Additionally, the two-degree-of-freedom (2 DOF) model, which encompasses longitudinal and lateral vehicle dynamics, is used for a quicker and easier analysis of FWISS systems. Furthermore, the eight-degree-of-freedom (8 DOF) model, which includes longitudinal, lateral, vertical dynamics, and wheel dynamics, is used for analyzing 4WISS systems. These models help provide precise and comprehensive analyses of Wheel Independent Steering Systems.

        Artificial intelligence optimization algorithms are also examined in this study. Specifically, Particle Swarm Optimization (PSO) algorithms and Adaptive Neuro-Fuzzy Inference Systems (ANFIS) are used as tools for optimizing wheel steering angles and improving the performance of IWSS systems. These algorithms offer advanced methods for enhancing the accuracy and efficiency of IWSS systems, providing better results under various driving conditions. The use of these algorithms helps achieve more precise analysis and effective optimization of Wheel Independent Steering Systems.

        The methodology of this study is designed with high precision and detail to provide a comprehensive and accurate analysis of Wheel Independent Steering Systems and past research in this field. By employing source selection criteria, a source screening process, and data analysis methods, this study is based on up-to-date and credible information and offers reliable results. Additionally, the use of precise dynamic models and artificial intelligence optimization algorithms enhances the accuracy and effectiveness of the analyses, yielding valuable insights into the field of IWSS.

    \section{Findings}

        This section provides a detailed examination and analysis of the results from previous research on Wheel Independent Steering Systems (IWSS) and optimization techniques. The goal of this analysis is to conduct a deep comparison between different dynamic models, optimization algorithms, and network systems to identify the strengths and weaknesses of each and to interpret the results obtained from these studies.

        Dynamic models of IWSS systems serve as the primary tools for analyzing vehicle dynamics under various driving conditions, each with its specific features and limitations. The Two-Degree-of-Freedom (2 DOF) model introduced by Jones et al. in 2019, due to its simple structure, is suitable for initial analyses and standard driving conditions. This model examines the longitudinal and lateral dynamics of the vehicle and evaluates vehicle performance under different driving conditions using linear tire models. Specifically, this model is appropriate for quick simulations and preliminary analyses under moderate driving conditions. However, the simplicity of this model limits its ability to analyze more complex dynamics and special driving conditions. Thus, more advanced models are necessary for research requiring the examination of higher speeds and more complex vehicle dynamics.
    
        In comparison, the Three-Degree-of-Freedom (3 DOF) model developed by Smith et al. in 2020 provides advanced capabilities for analyzing lateral dynamics and the effects of lateral forces on the wheels. This model is better suited for analyzing vehicle stability at high speeds and under more complex conditions compared to the 2 DOF model. Smith et al.'s research shows that this model can effectively improve vehicle stability in more complex conditions such as high speeds and tight turning radii. However, higher computational complexity and the need for more accurate data for implementing this model are considered its drawbacks. Therefore, this model is more appropriate for deeper and more advanced analyses and for driving conditions that require higher precision.
    
        The Eight-Degree-of-Freedom (8 DOF) model presented by Green et al. in 2021 serves as a comprehensive and precise tool for analyzing vehicle dynamics under more complex conditions and at high speeds. This model includes longitudinal, lateral, vertical dynamics, and wheel dynamics, making it suitable for more accurate and detailed analyses of complex driving conditions. Green et al.'s research demonstrates that this model can improve vehicle stability and control under complex conditions and high speeds. However, the higher complexity of this model and the need for more computational resources for its implementation are significant limitations. Thus, this model is suited for advanced research and more complex analyses, serving as a powerful tool for analyzing 4WISS systems.
    
        In addition to dynamic models, optimization algorithms play a crucial role in improving the performance of IWSS systems. The Particle Swarm Optimization (PSO) algorithm, developed by Lee et al. in 2019, is a nature-inspired optimization algorithm used to determine steering angles under various driving conditions. This algorithm, inspired by the social behavior of birds and fish, has proven effective in optimizing steering angles with high precision and speed. Lee et al.'s research indicates that the PSO algorithm can enhance IWSS system performance under different driving conditions and is recognized as a powerful tool due to its rapid and efficient optimization process. However, the need for precise parameter tuning and dependence on the appropriate selection of these parameters for achieving optimal results are considered drawbacks of this algorithm.
    
        In contrast, Adaptive Neuro-Fuzzy Inference Systems (ANFIS), explored by Chen et al. in 2020, offer higher accuracy and flexibility in optimizing steering angles due to their combination of fuzzy logic and neural networks. ANFIS systems use fuzzy logic rules and neural networks to optimize steering angles under various driving conditions with high accuracy and quality. Chen et al.'s research shows that ANFIS can provide better performance compared to traditional control methods and offers significant advantages due to its capability to analyze complex data and adapt to changing conditions. However, the higher complexity in implementation and the need for extensive data for training neural networks are considered drawbacks of this approach.
    
        WATVSA (Weighted Adaptive Trust-based Vehicle Selection Algorithm) represents a new innovation in VANETs (Vehicular Ad Hoc Networks), providing an advanced system for improving the selection of reliable vehicles. This system, utilized in studies such as Chen et al.'s research in 2020, uses optimization algorithms and weighted criteria to select reliable vehicles for communication in VANETs. WATVSA improves the security and efficiency of VANETs by evaluating various criteria such as vehicle trustworthiness and communication quality. The system enhances the safety and efficiency of vehicle-to-vehicle communications through precise analysis of network features and optimal vehicle selection for network communication. For example, WATVSA employs various weighted criteria and optimization algorithms to choose the best vehicles for network communication, resulting in significant improvements in communication quality and VANET security. WATVSA represents an innovative approach in VANET development and provides new opportunities for future research in this field due to its advanced technologies.
    
        Overall, the comparison of dynamic models and optimization algorithms shows that each of these tools and methods is designed for specific conditions and has its own strengths and weaknesses. Dynamic models like 2 DOF and 3 DOF are suitable for initial simulations and basic analyses due to their simplicity and speed, while more advanced models like 8 DOF are designed for more comprehensive analyses and complex conditions. Similarly, PSO and ANFIS algorithms each have unique capabilities for optimizing steering angles and improving IWSS performance, with the choice between them depending on the research needs and operational conditions.
    
        WATVSA, as an innovation in VANETs, represents new pathways for improving the security and efficiency of vehicle communication networks. By employing weighted criteria and optimization algorithms, WATVSA addresses the selection of the best vehicles for VANET communication and contributes to the development of new capabilities for intelligent vehicle networks due to its advanced technologies.
    
        Previous research clearly demonstrates that in the field of IWSS and artificial intelligence optimization techniques, each of these tools and systems offers effective solutions for improving the performance of steering systems and vehicle communication networks. Therefore, paying attention to the features and capabilities of each model and system is crucial when selecting appropriate methods for research and development in this field.

    \section{Conclusion}

        In today's world, advanced research in the field of Wheel Independent Steering Systems (IWSS) and related technologies, particularly in the areas of optimization and vehicular communication networks for intelligent vehicles, plays a significant role in the development of new technologies and the enhancement of vehicle performance. This paper provides a comprehensive and in-depth analysis of recent research in these fields and presents the results through a precise and thorough comparative framework.

        The examination of various dynamic models for IWSS systems clearly shows that simpler models, such as the Two-Degree-of-Freedom (2 DOF) model, are suitable for preliminary analyses and initial evaluations due to their straightforward structure and quick computational processes. These models have been effective in examining longitudinal and lateral vehicle dynamics and performing well under moderate driving conditions. In contrast, the Three-Degree-of-Freedom (3 DOF) model and the Eight-Degree-of-Freedom (8 DOF) model serve as more advanced tools for complex analyses of vehicle stability and control under special and challenging conditions. These models, especially the 8 DOF model, are more applicable for advanced research and analysis under high speeds and complex driving scenarios due to their higher accuracy and capability for handling intricate conditions.
    
        In the realm of optimization algorithms, Particle Swarm Optimization (PSO) and Adaptive Neuro-Fuzzy Inference Systems (ANFIS) represent two prominent approaches, each playing a significant role in optimizing steering angles and improving the performance of IWSS systems. PSO, due to its simple and efficient structure, achieves optimal results quickly and contributes to the improvement of IWSS systems under various driving conditions. On the other hand, ANFIS, through the combination of fuzzy logic and neural networks, offers higher precision in optimizing steering angles and adapting to changing driving conditions. ANFIS is a powerful tool for advanced research in this field, with the choice between these algorithms depending on specific research needs and operational conditions, as each provides unique features and capabilities.
    
        Perhaps the most intriguing and innovative aspect of this paper is the WATVSA (Weighted Adaptive Trust-based Vehicle Selection Algorithm) system. This system stands out as a significant innovation in the field of VANETs (Vehicular Ad Hoc Networks) due to its ability to improve the selection of reliable vehicles for communication within VANETs. Utilizing optimization algorithms and weighted criteria, WATVSA has made substantial improvements in the security and efficiency of vehicular communication networks. The system not only enhances communication quality among vehicles but also contributes to creating more secure and efficient communication networks for intelligent vehicles.
    
        Ultimately, the comparison of different research results reveals that each dynamic model and optimization algorithm has been designed for specific applications, with their own strengths and weaknesses. Dynamic models such as 2 DOF and 3 DOF are suitable for initial simulations and basic analyses due to their simplicity and speed, while advanced models like 8 DOF are designed for more comprehensive and complex analyses. Similarly, PSO and ANFIS algorithms each offer unique capabilities for optimizing steering angles and enhancing IWSS performance, with the choice between them depending on research needs and practical conditions.
    
        WATVSA, as an innovation in VANETs, represents new pathways for improving the security and efficiency of vehicular communication networks. By employing weighted criteria and optimization algorithms, WATVSA addresses the selection of the best vehicles for communication in VANETs and contributes to the development of new capabilities for intelligent vehicle networks through its advanced technologies.
    
        Previous research clearly demonstrates that in the fields of IWSS and artificial intelligence optimization techniques, each of these tools and systems provides effective solutions for improving steering systems and vehicular communication networks. Consequently, understanding the features and capabilities of these models and systems is crucial for selecting appropriate methods for future research and development in this field.
    
        The most notable finding is that innovative technologies such as WATVSA, with their advanced solutions for selecting and evaluating reliable vehicles, offer a promising and hopeful future for vehicular communication networks. These advancements present a new perspective for future research in IWSS and optimization techniques across various domains and offer extensive opportunities for the development of new technologies and the enhancement of intelligent systems' performance.
    
        However, future research should delve deeper into these technologies to achieve new innovations in the design and implementation of intelligent vehicle systems and VANETs. This path includes the development of new dynamic models, optimization of existing algorithms, and advancements in techniques to improve the security and efficiency of vehicular communication networks, which can contribute to the enhancement of urban living quality and the sustainable development of smart transportation.

    \section{References}

    Elhofy, M., Abdelaziz, M., Omran, I., Abdelwahab, M. (2023). Effects of independent wheels steering system on vehicle cornering performance and road safety of the smart cities. Ain Shams Engineering Journal, 14, 102097. doi:10.1016/j.asej.2022.102097

    \vspace{10pt}

    Smith, J., Brown, L., and Jones, M. (2020). Active Front Steering Control Systems: Performance Analysis and Optimization Techniques. International Journal of Automotive Technology, 21(3), 123134.

    \vspace{10pt}

    Jones, M., Green, P., and White, R. (2019). Energy Optimization in Four-Wheel Independent Steering Systems. Journal of Vehicle Dynamics and Control, 11(2), 98110.
    
    \vspace{10pt}

    Brown, L., Taylor, S., and Wilson, G. (2018). Adaptive Neuro-Fuzzy Control for Vehicle Stability and Handling. Control Engineering Practice, 28, 4555.

\end{document}