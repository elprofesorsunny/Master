\documentclass{book}

\usepackage{geometry}
\geometry{
    a4paper,
    left=1cm,
    right=1cm,
    top=2cm,
    bottom=2cm}
\usepackage{xepersian}
\settextfont{Vazirmatn-Regular.ttf}

\title{ارائه درس رایانش ابری}
\author{محمد خورشیدی روزبهانی\\40215741002013 \and شارا شاهوردیان\\40215741002032}
\date{}

\begin{document}

    \maketitle

    \tableofcontents

    \newpage

    \chapter{خوبی‌های ابری}

        ظهور رایانش ابری مزایای مدل خدماتی را به کاربران رایانش ارائه می‌دهد. کاربران رایانش اکنون به عنوان مشترکان یا مصرف‌کنندگان شناخته می‌شوند زیرا به سمت رایانش ابری حرکت می‌کنند. رایانش ابری به مشترکان خود از طریق شبکه داخلی و اینترنت ارائه می‌شود. مشترکان می‌توانند به تسهیلات رایانشی به صورت اشتراکی در هر زمان و هر مکان دسترسی داشته باشند.

        تعداد زیادی از مزایای رایانش ابری کاربران را به سمت خود جلب می‌کند. اما هر نوآوری جدید با چالش‌هایی همراه است و رایانش ابری هم استثنا نیست. این فصل مزایای مختلف رایانش ابری را بررسی می‌کند و همچنین چالش‌هایی که در پیش روی آن قرار دارد را مطرح می‌کند.
    
        بزرگ‌ترین چالش مربوط به امنیت داده و مسائل پیروی از استانداردها است. بیشتر چالش‌های بحرانی دیگر به دلیل عدم وجود استانداردهای باز هستند که تامین‌کنندگان ابرها از استاندارد یا فناوری اختصاصی خود استفاده می‌کنند. نکته مثبت این است که تلاش‌های قابل توجهی برای حل این مسائل صورت گرفته است. علاوه بر این، این فصل به طور مختصر نقش خدمات وب در توسعه رایانش ابری را 
        معرفی می‌کند.

        \section{منشا اصطلاح رایانش ابری}

            اصل واژه "رایانش ابری" به اوایل دهه ۱۹۹۰ برمی‌گردد. در آن روزهای ابتدایی طراحی شبکه، مهندسان شبکه به ترسیم نمودارهای شبکه که دستگاه‌ها و اتصالات مختلف را نشان می‌دادند، علاقه داشتند. در چنین نمودارهایی، آنها عرصه‌های شبکه بیرونی را با نماد ابر نشان می‌دادند زیرا جزئیات آنها در دسترس آنها نبود. این در آن دوره به عنوان "ابر شبکه" یا "ابر" در صنعت شبکه‌ها شناخته می‌شد، اما امروزه ما به همان معنا "رایانش ابری" را نمی‌گوییم.

            با آغاز فعالیت‌های رایانش کاربردی به سوی انتهای قرن گذشته، شرکت‌های نرم‌افزاری بزرگ بر روی ارائه برنامه‌ها از طریق اینترنت تمرکز کردند. خدمات ایمیل در طی این دوره به سرعت گسترش یافت زیرا تامین‌کنندگان شروع به ارائه این امکان به کاربران خود کردند. و پرچمدارترین اقدام به نظر از Salesforce.com آمد که در سال ۱۹۹۹ برنامه‌های تجاری را برای شرکت‌ها از طریق اینترنت ارائه داد. اما همه این تلاش‌ها به عنوان بخشی از توسعه تسهیلات رایانش کاربردی دیده می‌شد. تا آن زمان هیچ رایانش ابری ظاهر نشده بود.

    %%%%%%%%%%%%%%%%%%%%%%%%%%%%%%%%%%%%%%%%%%%%%%%%%%%%%%%%%%%%%%%%%%%%%%%%%%%%%%%%%%%%%%%%%%%%%%%%
    %%%%%%%       Shara        %%%%%%%%%%%%%%%%%%%%%%%%%%%%%%%%%%%%%%%%%%%%%%%%%%%%%%%%%%%%%%%%%%%%%
    %%%%%%%%%%%%%%%%%%%%%%%%%%%%%%%%%%%%%%%%%%%%%%%%%%%%%%%%%%%%%%%%%%%%%%%%%%%%%%%%%%%%%%%%%%%%%%%%

    \subsection{امکان پرداخت بر اساس استفاده}
    محاسبات ابری هزینه‌ای را برای مشترکانشان در زمانی که از آن استفاده نمی‌شوند، دریافت نمی‌کنند. حتی هزینه‌ی آن ثابت نیست؛ بلکه بستگی به مدت استفاده دارد. به جای آن، هر استفاده‌ای اندازه‌گیری می‌شود و کاربران بر اساس مصرف خود، هزینه‌ی مناسبی پرداخت می‌کنند. این باعث کاهش هزینه‌های محاسباتی می‌شود.

    \section{قدرت محاسباتی و ذخیره‌سازی نامحدود}
    در محاسبات ابری، کاربران می‌توانند به طور آسان به قدرت محاسباتی شبیه به یک سوپرکامپیوتر با هزینه‌ای مناسب دسترسی پیدا کنند، در صورت نیاز. در روش سنتی، تنها شرکت‌های بزرگ می‌توانستند هزینه‌های کامپیوتری پیشرفته را تحمل کنند. ذخیره‌سازی نیز یک مسئله مهم برای کاربران است. ابر به اندازه‌ای ذخیره‌سازی فراهم می‌کند که لازم است. این تقریباً نامحدود است که به عنوان یک مزیت بزرگ برای کاربران می‌تواند مشاهده شود.

\end{document}