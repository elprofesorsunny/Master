\documentclass{book}


\usepackage{xcolor}
\usepackage{tcolorbox}

\usepackage{geometry}
\geometry{
    a4paper,
    left=1cm,
    right=1cm,
    top=2cm,
    bottom=2cm}
\usepackage{xepersian}
\settextfont{Vazirmatn-Regular.ttf}

\newtcolorbox{addinfo}[1][]{
    colback=black!5,
    colframe=black!30,
    title=اطلاعات اضافی
}

\linespread{1.5}

\title{ارائه درس رایانش ابری}
\author{محمد خورشیدی روزبهانی\\40215741002013 \and شارا شاهوردیان\\40215741002032}
\date{}

\begin{document}

    \maketitle

    \tableofcontents

    \newpage

    \chapter{خوبی‌های ابری}

        ظهور رایانش ابری مزایای مدل خدماتی را به کاربران رایانش ارائه می‌دهد. کاربران رایانش اکنون به عنوان مشترکان یا مصرف‌کنندگان شناخته می‌شوند زیرا به سمت رایانش ابری حرکت می‌کنند. رایانش ابری به مشترکان خود از طریق شبکه داخلی و اینترنت ارائه می‌شود. مشترکان می‌توانند به تسهیلات رایانشی به صورت اشتراکی در هر زمان و هر مکان دسترسی داشته باشند.

        تعداد زیادی از مزایای رایانش ابری کاربران را به سمت خود جلب می‌کند. اما هر نوآوری جدید با چالش‌هایی همراه است و رایانش ابری هم استثنا نیست. این فصل مزایای مختلف رایانش ابری را بررسی می‌کند و همچنین چالش‌هایی که در پیش روی آن قرار دارد را مطرح می‌کند.
    
        بزرگ‌ترین چالش مربوط به امنیت داده و مسائل پیروی از استانداردها است. بیشتر چالش‌های بحرانی دیگر به دلیل عدم وجود استانداردهای باز هستند که تامین‌کنندگان ابرها از استاندارد یا فناوری اختصاصی خود استفاده می‌کنند. نکته مثبت این است که تلاش‌های قابل توجهی برای حل این مسائل صورت گرفته است. علاوه بر این، این فصل به طور مختصر نقش خدمات وب در توسعه رایانش ابری را 
        معرفی می‌کند.

        \section{منشا اصطلاح رایانش ابری}

            اصل واژه "رایانش ابری" به اوایل دهه ۱۹۹۰ برمی‌گردد. در آن روزهای ابتدایی طراحی شبکه، مهندسان شبکه به ترسیم نمودارهای شبکه که دستگاه‌ها و اتصالات مختلف را نشان می‌دادند، علاقه داشتند. در چنین نمودارهایی، آنها عرصه‌های شبکه بیرونی را با نماد ابر نشان می‌دادند زیرا جزئیات آنها در دسترس آنها نبود. این در آن دوره به عنوان "ابر شبکه" یا "ابر" در صنعت شبکه‌ها شناخته می‌شد، اما امروزه ما به همان معنا "رایانش ابری" را نمی‌گوییم.

            با آغاز فعالیت‌های رایانش کاربردی به سوی انتهای قرن گذشته، شرکت‌های نرم‌افزاری بزرگ بر روی ارائه برنامه‌ها از طریق اینترنت تمرکز کردند. خدمات ایمیل در طی این دوره به سرعت گسترش یافت زیرا تامین‌کنندگان شروع به ارائه این امکان به کاربران خود کردند. و پرچمدارترین اقدام به نظر از Salesforce.com آمد که در سال ۱۹۹۹ برنامه‌های تجاری را برای شرکت‌ها از طریق اینترنت ارائه داد. اما همه این تلاش‌ها به عنوان بخشی از توسعه تسهیلات رایانش کاربردی دیده می‌شد. تا آن زمان هیچ رایانش ابری ظاهر نشده بود.

            \begin{addinfo}
                
                مفهوم محاسبات ابری با معنای فعلی‌اش حول سال ۲۰۰۶ در بازار ظاهر شد.
                
            \end{addinfo}

            در سال‌های ابتدای قرن فعلی، تعداد کمی از افراد صنعتی که در توسعه امکانات محاسبات کاربردی (utility computing facility) دخیل بودند، آن را به عنوان "ابر" نامیدند. اما تا سال ۲۰۰۶ این اصطلاح با مفهوم فعلی‌اش در صنعت تجاری ظاهر نشد. احتمالاً اولین بار اصطلاح "محاسبات ابری" در یک فروم رسمی توسط مدیر عامل گوگل آن زمان، اریک شمیت، در سال ۲۰۰۶ در یک کنفرانس استفاده شد. همچنین، در آن زمان استفاده گسترده‌ای از این اصطلاح مشاهده شد زیرا چندین شرکت مانند آمازون، مایکروسافت، آی‌بی‌ام تلاش‌های خود در زمینه محاسبات ابری را به دست عمومی رساندند. آمازون خدمات نوآورانه‌اش به نام ابر محاسبات الاستیک (Elastic Compute Cloud یا EC2) را در سال ۲۰۰۶ معرفی کرد.

            معنای واژه "ابر" به معنای مه، حجاب یا ابهام است. به معنای منطقی، این به منطقیت "یک مجموعه از منابع محاسباتی که جزئیات آن از دید کاربران مخفی است" می‌خورد.


        \section{ابتکارات اولیه}

            این ابتکار از سوی Salesforce.com در سال ۱۹۹۹ برای ارائه برنامه‌های کسب و کار (Enterprise) از طریق یک وب‌سایت "معمولی" به عنوان نخستین تلاش از این دست محسوب می‌شود. موفقیت این تلاش از سوی Salesforce، سایر شرکت‌های نرم‌افزاری را برای ارائه برنامه‌های کسب و کار از طریق اینترنت تشویق کرد. این لحظه تحولی بود که شرکت‌های فناوری محاسباتی شروع به ابتکار‌هایی در توسعه برنامه‌های کسب و کار مبتنی بر محاسبات ابری نمودند.

            \begin{addinfo}
                
                سرویس Salesforce.com که در سال ۱۹۹۹ راه‌اندازی شد، نخستین ابتکار تجاری موفق برای ارائه برنامه‌های کسب و کار از طریق اینترنت بود. این ابتکار اولین گام به سوی محاسبات ابری بود.

            \end{addinfo}

            پیشرفت بعدی اصلی از سوی آمازون با راه‌اندازی سرویس وب آمازون (Amazon Web Service یا AWS) در سال ۲۰۰۲ اتفاق افتاد که خدمات محاسباتی را از طریق اینترنت ارائه می‌داد. AWS مجموعه‌ای از خدمات از جمله ذخیره‌سازی را فراهم می‌کرد. آمازون نقش کلیدی در توسعه مدل محاسباتی مبتنی بر امکانات (utility model based computing services) در آن دوران ایفا کرد و به زودی بسیاری از شرکت‌های نرم‌افزاری شروع به به‌روزرسانی مراکز داده خود برای پشتیبانی از محاسبات امکانات نمودند. یک مرکز داده (Data Center) مخزن فیزیکی سازمان‌یافته‌ای از سیستم‌های محاسباتی و اجزای مرتبط مانند ذخیره‌سازی و شبکه‌سازی است که شرکت‌ها برای ساخت تسهیلات محاسباتی خود تحت نظر دارند.

            این حرکت به آرامی به سوی آنچه که امروز با نام محاسبات ابری شناخته می‌شود، متمایل شد. در سال ۲۰۰۶، آمازون سرویس وب EC2 را راه‌اندازی کرد، جایی که شرکت‌ها و افراد می‌توانستند کامپیوترهای (مجازی) را برای اجرای برنامه‌های کامپیوتری خود اجاره کنند. بعد از مدتی از راه‌اندازی EC2، "Google Docs" توسط گوگل (در سال ۲۰۰۶) محصولی دیگر در زمینه خدمات محاسبات ابری را به توجه عمومی معرفی کرد.

            در سال ۲۰۰۷، Salesforce.com سرویس دیگری به نام force.com را راه‌اندازی کرد، جایی که هر کسی می‌توانست برنامه‌ها را بسازد و وب‌سایت‌ها را راه‌اندازی کند. در سال ۲۰۰۹، ورود تجاری مایکروسافت به عرصه محاسبات ابری با راه‌اندازی "Windows Azure" رخ داد. هدف از Azure این بود که به مشتریان امکان اجرای برنامه‌های ویندوز خود را از طریق اینترنت فراهم کند.

            علاوه بر این ابتکارهای تجاری، در آن سال‌ها بسیاری از سازمان‌های تحقیقاتی و انجمن‌های منبع باز با آغاز ابتکارات محاسبات ابری خود، فعالیت‌های محاسبات ابری خود را آغاز کردند. به عنوان مثال، ناسا یک پلتفرم محاسبات ابری منبع باز به نام "Nebula" را در سال ۲۰۰۸ برای استفاده داخلی خود توسعه و راه‌اندازی کرد.

            \begin{addinfo}
                
                محاسبات ابری الاستیک (Elastic Compute Cloud یا EC2) که توسط آمازون در سال ۲۰۰۶ معرفی شد، اولین سرویس ابری ملموسی بود که در بازار عرضه شد.

            \end{addinfo}

        \section{محاسبات کاربردی}

            متن

    %%%%%%%%%%%%%%%%%%%%%%%%%%%%%%%%%%%%%%%%%%%%%%%%%%%%%%%%%%%%%%%%%%%%%%%%%%%%%%%%%%%%%%%%%%%%%%%%
    %%%%%%%       Shara        %%%%%%%%%%%%%%%%%%%%%%%%%%%%%%%%%%%%%%%%%%%%%%%%%%%%%%%%%%%%%%%%%%%%%
    %%%%%%%%%%%%%%%%%%%%%%%%%%%%%%%%%%%%%%%%%%%%%%%%%%%%%%%%%%%%%%%%%%%%%%%%%%%%%%%%%%%%%%%%%%%%%%%%

    \subsection{امکان پرداخت بر اساس استفاده}
    محاسبات ابری هزینه‌ای را برای مشترکانشان در زمانی که از آن استفاده نمی‌شوند، دریافت نمی‌کنند. حتی هزینه‌ی آن ثابت نیست؛ بلکه بستگی به مدت استفاده دارد. به جای آن، هر استفاده‌ای اندازه‌گیری می‌شود و کاربران بر اساس مصرف خود، هزینه‌ی مناسبی پرداخت می‌کنند. این باعث کاهش هزینه‌های محاسباتی می‌شود.

    \section{قدرت محاسباتی و ذخیره‌سازی نامحدود}
    در محاسبات ابری، کاربران می‌توانند به طور آسان به قدرت محاسباتی شبیه به یک سوپرکامپیوتر با هزینه‌ای مناسب دسترسی پیدا کنند، در صورت نیاز. در روش سنتی، تنها شرکت‌های بزرگ می‌توانستند هزینه‌های کامپیوتری پیشرفته را تحمل کنند. ذخیره‌سازی نیز یک مسئله مهم برای کاربران است. ابر به اندازه‌ای ذخیره‌سازی فراهم می‌کند که لازم است. این تقریباً نامحدود است که به عنوان یک مزیت بزرگ برای کاربران می‌تواند مشاهده شود.

\end{document}