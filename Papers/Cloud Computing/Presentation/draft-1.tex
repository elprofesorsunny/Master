\documentclass{book}

\usepackage{geometry}
\geometry{
    a4paper,
    left=1cm,
    right=1cm,
    top=2cm,
    bottom=2cm}
\usepackage{xepersian}
\settextfont{Vazirmatn-Regular.ttf}

\title{ارائه درس رایانش ابری}
\author{محمد خورشیدی روزبهانی\\40215741002013 \and شارا شاهوردیان\\40215741002032}
\date{}

\begin{document}

    \maketitle

    \tableofcontents

    \newpage

    \chapter{خوبی‌های ابری}

        ظهور رایانش ابری مزایای مدل خدماتی را به کاربران رایانش ارائه می‌دهد. کاربران رایانش اکنون به عنوان مشترکان یا مصرف‌کنندگان شناخته می‌شوند زیرا به سمت رایانش ابری حرکت می‌کنند. رایانش ابری به مشترکان خود از طریق شبکه داخلی و اینترنت ارائه می‌شود. مشترکان می‌توانند به تسهیلات رایانشی به صورت اشتراکی در هر زمان و هر مکان دسترسی داشته باشند.

        تعداد زیادی از مزایای رایانش ابری کاربران را به سمت خود جلب می‌کند. اما هر نوآوری جدید با چالش‌هایی همراه است و رایانش ابری هم استثنا نیست. این فصل مزایای مختلف رایانش ابری را بررسی می‌کند و همچنین چالش‌هایی که در پیش روی آن قرار دارد را مطرح می‌کند.
    
        بزرگ‌ترین چالش مربوط به امنیت داده و مسائل پیروی از استانداردها است. بیشتر چالش‌های بحرانی دیگر به دلیل عدم وجود استانداردهای باز هستند که تامین‌کنندگان ابرها از استاندارد یا فناوری اختصاصی خود استفاده می‌کنند. نکته مثبت این است که تلاش‌های قابل توجهی برای حل این مسائل صورت گرفته است. علاوه بر این، این فصل به طور مختصر نقش خدمات وب در توسعه رایانش ابری را 
        معرفی می‌کند.

        \section{منشا اصطلاح رایانش ابری}

            اصل واژه "رایانش ابری" به اوایل دهه ۱۹۹۰ برمی‌گردد. در آن روزهای ابتدایی طراحی شبکه، مهندسان شبکه به ترسیم نمودارهای شبکه که دستگاه‌ها و اتصالات مختلف را نشان می‌دادند، علاقه داشتند. در چنین نمودارهایی، آنها عرصه‌های شبکه بیرونی را با نماد ابر نشان می‌دادند زیرا جزئیات آنها در دسترس آنها نبود. این در آن دوره به عنوان "ابر شبکه" یا "ابر" در صنعت شبکه‌ها شناخته می‌شد، اما امروزه ما به همان معنا "رایانش ابری" را نمی‌گوییم.

            با آغاز فعالیت‌های رایانش کاربردی به سوی انتهای قرن گذشته، شرکت‌های نرم‌افزاری بزرگ بر روی ارائه برنامه‌ها از طریق اینترنت تمرکز کردند. خدمات ایمیل در طی این دوره به سرعت گسترش یافت زیرا تامین‌کنندگان شروع به ارائه این امکان به کاربران خود کردند. و پرچمدارترین اقدام به نظر از Salesforce.com آمد که در سال ۱۹۹۹ برنامه‌های تجاری را برای شرکت‌ها از طریق اینترنت ارائه داد. اما همه این تلاش‌ها به عنوان بخشی از توسعه تسهیلات رایانش کاربردی دیده می‌شد. تا آن زمان هیچ رایانش ابری ظاهر نشده بود.

    %%%%%%%%%%%%%%%%%%%%%%%%%%%%%%%%%%%%%%%%%%%%%%%%%%%%%%%%%%%%%%%%%%%%%%%%%%%%%%%%%%%%%%%%%%%%%%%%
    %%%%%%%       Shara        %%%%%%%%%%%%%%%%%%%%%%%%%%%%%%%%%%%%%%%%%%%%%%%%%%%%%%%%%%%%%%%%%%%%%
    %%%%%%%%%%%%%%%%%%%%%%%%%%%%%%%%%%%%%%%%%%%%%%%%%%%%%%%%%%%%%%%%%%%%%%%%%%%%%%%%%%%%%%%%%%%%%%%%

    \subsection{امکان پرداخت بر اساس استفاده}
    محاسبات ابری هزینه‌ای را برای مشترکانشان در زمانی که از آن استفاده نمی‌شوند، دریافت نمی‌کنند. حتی هزینه‌ی آن ثابت نیست؛ بلکه بستگی به مدت استفاده دارد. به جای آن، هر استفاده‌ای اندازه‌گیری می‌شود و کاربران بر اساس مصرف خود، هزینه‌ی مناسبی پرداخت می‌کنند. این باعث کاهش هزینه‌های محاسباتی می‌شود.

    \subsection{قدرت محاسباتی و ذخیره‌سازی نامحدود}
    در محاسبات ابری، کاربران می‌توانند به طور آسان به قدرت محاسباتی شبیه به یک سوپرکامپیوتر با هزینه‌ای مناسب دسترسی پیدا کنند، در صورت نیاز. در روش سنتی، تنها شرکت‌های بزرگ می‌توانستند هزینه‌های کامپیوتری پیشرفته را تحمل کنند. ذخیره‌سازی نیز یک مسئله مهم برای کاربران است. ابر به اندازه‌ای ذخیره‌سازی فراهم می‌کند که لازم است. این تقریباً نامحدود است که به عنوان یک مزیت بزرگ برای کاربران می‌تواند مشاهده شود.

    \subsection{کیفیت خدمات}
    در محاسبات سنتی، شرکت‌ها اغلب بخش‌های اصلی از کارهای مرتبط با محاسبات را به شرکای سومی برون‌سپاری می‌کردند. بنابراین، کیفیت خدمات به طور گسترده از تخصص شرکای سوم یا تیم‌های داخلی مدیریت آن وابسته بود. در حالی که در محاسبات ابری، کیفیت بالای خدمات (QoS) تضمین می‌شود زیرا توسط تامین‌کنندگان معتبر محاسباتی که دارای کارکنان آموزش‌دیده و تخصص دارند، ارائه می‌شود که به طور اختصاصی در زمینه محاسبات فعالیت می‌کنند.
    
    % box
    وقتی خدمات توسط تامین‌کنندگان معروف ارائه می‌شود، کیفیت خدمات تضمین می‌شود و این وظیفه تامین‌کننده می‌شود.

    \subsection{اعتماد‌پذیری}
    توانایی ارائه خدمات با کیفیت و پشتیبانی از توازن بار، پشتیبان‌گیری و بازیابی در صورت عدم موفقیت، باعث می‌شود که تامین‌کنندگان ابر معروف بسیار قابل اعتماد باشند که اغلب به عنوان یک مشکل بزرگ در محاسبات سنتی مطرح می‌شود. در محاسبات ابری، مشترکان دیگر نیازی به برنامه‌ریزی برای همه این وظایف پیچیده ندارند زیرا تامین‌کنندگان این مسائل را در دست می‌گیرند و آنها را بهتر انجام می‌دهند.

    \subsection{پیوستگی در دسترسی}
    تامین‌کنندگان معتبر ابر تقریباً اطمینان از دسترسی به خدمات 24 ساعته و 7 روزه را فراهم می‌کنند. آمار نشان داده است که زمان فعالیت سرویس (ارائه شده توسط تامین‌کنندگان معتبر) در طول یک سال معمولاً به کمتر از 99.9٪ نمی‌رسد. این پایداری مداوم تضمین‌شده خدمات ابر برای هر کسب و کاری یک مزیت بزرگ است.

    \subsection{استقلال مکانی/راحتی دسترسی}
    محاسبات ابری از طریق اینترنت در همه جا در دسترس است. کاربران می‌توانند از طریق هر دستگاه محاسباتی مانند کامپیوترهای شخصی یا دستگاه‌های قابل حمل مانند تبلت، لپ‌تاپ یا تلفن هوشمند به آن دسترسی پیدا کنند. تنها چیزی که برای بهره‌مندی از محاسبات ابری از طریق این دستگاه‌ها لازم است، دسترسی به اینترنت است، بدون توجه به مکان جغرافیایی یا منطقه زمانی.

    % box
    راحتی دسترسی و سرمایه‌گذاری کم، محاسبات ابری را به یک حوزه با حداقل مانع ورود تبدیل کرده‌است.

    \subsection{مقاومت بالا}
    مقاومت توانایی کاهش شدت و/یا مدت مختل شدن توسط شرایط ناخواسته است. سطح بالاتری از مقاومت ارزش بسیاری در محیط محاسباتی دارد. محاسبات ابری بر پایه زیرساخت محاسبات مقاوم ساخته شده‌است و بنابراین خدمات ابری به حملات و اشکالات بیشتری مقاومت دارند. مقاومت زیرساختی از طریق تکرار زیرساخت همراه با مکانیزم مؤثری برای پیش‌بینی، جذب و سازگاری دست‌یافته می‌شود. مصرف‌کنندگان ابر می‌توانند با بهره‌گیری از مقاومت منابع IT مبتنی بر ابر، اعتبار کسب‌وکارهای خود را افزایش دهند.

    \subsection{راه‌اندازی سریع}
    زمان راه‌اندازی در محیط ابر به طرز قابل توجهی کاهش یافته است نسبت به آنچه در محیط محاسبات سنتی بود. این امکان از طریق تأمین منابع سریع و خودکار در محیط ابر فراهم می‌شود. در یک بازار بسیار رقابتی، توانایی راه‌اندازی سریع مزایای کسب و کار قابل توجهی را به دنبال دارد.

    % box
    راه‌اندازی سریع‌تر سیستم یا برنامه مزیت تجاری را در بازار رقابتی به دست می‌آورد.

    \subsection{به‌روزرسانی‌های نرم‌افزاری خودکار}
    مسئله به‌روزرسانی نرم‌افزار در محیط محاسبات سنتی بسیار زحمت‌آور است. به‌طور دوره‌ای پچ‌های جدید منتشر می‌شوند و کاربران باید این پچ‌ها را به طور دوره‌ای اجرا کنند. در محیط محاسبات ابری، این به‌روزرسانی به‌صورت خودکار اتفاق می‌افتد. تامین‌کنندگان ابر همواره آخرین نسخه موجود هر نرم‌افزار را (مگر اینکه درخواست دیگری داده شده باشد) ارائه می‌دهند. محیط به‌روزرسانی‌شده تقریباً فوراً پس از ارائه در دسترس کاربران قرار می‌گیرد، و هر زمان کاربر بعدی وارد سیستم شود، موجود خواهد بود.

    \subsection{تهیه مجوز نیاز نیست}
    تهیه مجوز برنامه‌ها نیازمند ترتیبات بودجه‌ای جداگانه در محاسبات سنتی بود. علاوه بر این، برنامه‌های اضافی معمولاً با بسته‌های مجوزی ارائه می‌شدند. محاسبات ابری نیز این مشکل را برطرف کرده است. در اینجا، کاربران نیازی به تهیه هرگونه مجوز دوره‌ای برای استفاده از برنامه‌ها ندارند؛ بلکه به آنها اجازه می‌دهند که بر اساس استفاده‌شان از هر نرم‌افزار، پرداخت کنند (پس از پرداخت).
    
    % box
    در محاسبات ابری، مجوز نرم‌افزار دیگر نگرانی برای کاربران نیست.

    \subsection{امنیت در برابر فاجعه}
    خرابی سیستم به دلیل شکست فنی ناگهانی یا فاجعه طبیعی نگرانی اصلی برای کاربران است. به ویژه، هرگونه آسیب به دستگاه‌های ذخیره‌سازی فیزیکی ممکن است باعث ضرر تجاری عظیم شود. محاسبات ابری ارائه شده توسط تامین‌کنندگان معتبر دارای سیستم‌های بازیابی قدرتمندی در تنظیمات خود است. بنابراین، سیستم‌ها و داده‌ها از لحاظ ایمنی و امنیت، در محاسبات ابری نسبت به سیستم‌های قبلی بیشتر محافظت می‌شوند.

    \subsection{دوستی به محیط زیست}
    محاسبات ابری از محیط زیست دوستانه حمایت می‌کند. استفاده مناسب از منابع، نیاز کلی به منابع الکترونیکی را کاهش می‌دهد، بنابراین تولید زباله‌های الکترونیکی را نیز کاهش می‌دهد. این برای محیط زیست مفید است زیرا اگر زباله‌های الکترونیکی به درستی پردازش نشوند، برای اکوسیستم مضر هستند. علاوه بر این، کاهش نیاز به منابع منجر به کاهش تقاضا و بنابراین تولید منابع محاسباتی می‌شود. این کاهش در تولید الکترونیکی باعث کاهش انتشار کربن و کمک به کاهش کلی ردپای کربن می‌شود.

    % box
    محاسبات ابری یک رویکرد محاسباتی سبز و دوستدار محیط زیست است.


    این لیست طولانی از مزایایی که در بالا بحث شد، در شکل ۳.۲ نمایش داده شده‌اند. آنها نشان می‌دهند که محاسبات ابری چه قدر مفید است و چرا انقدر افراد به آن مشتاق هستند.

    % شکل

    \section{چالش‌های محاسبات ابری}

\end{document}