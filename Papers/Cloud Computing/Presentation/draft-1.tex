\documentclass{book}

\usepackage{geometry}
\geometry{
    a4paper,
    left=1cm,
    right=1cm,
    top=2cm,
    bottom=2cm}
\usepackage{xepersian}
\settextfont{Vazirmatn-Regular.ttf}

\title{ارائه درس رایانش ابری}
\author{محمد خورشیدی روزبهانی\\40215741002013 \and شارا شاهوردیان\\40215741002032}
\date{}

\begin{document}

    \maketitle

    \tableofcontents

    \newpage

    \chapter{خوبی‌های ابری}

    خورشیدی

    خورشید هستم.

    سلام من هستم. شارا 

    لورم ایپسوم متن ساختگی با تولید سادگی نامفهوم از صنعت چاپ، و با استفاده از طراحان گرافیک است، چاپگرها و متون بلکه روزنامه و مجله در ستون و سطرآنچنان که لازم است، و برای شرایط فعلی تکنولوژی مورد نیاز، و کاربردهای متنوع با هدف بهبود ابزارهای کاربردی می باشد، کتابهای زیادی در شصت و سه درصد گذشته حال و آینده، شناخت فراوان جامعه و متخصصان را می طلبد، تا با نرم افزارها شناخت بیشتری را برای طراحان رایانه ای علی الخصوص طراحان خلاقی، و فرهنگ پیشرو در زبان فارسی ایجاد کرد، در این صورت می توان امید داشت که تمام و دشواری موجود در ارائه راهکارها، و شرایط سخت تایپ به پایان رسد و زمان مورد نیاز شامل حروفچینی دستاوردهای اصلی، و جوابگوی سوالات پیوسته اهل دنیای موجود طراحی اساسا مورد استفاده قرار گیرد.

\end{document}