\documentclass{book}


\usepackage{xcolor}
\usepackage{tcolorbox}

\usepackage{geometry}
\geometry{
    a4paper,
    left=1cm,
    right=1cm,
    top=2cm,
    bottom=2cm}
\usepackage{xepersian}
\settextfont{Vazirmatn-Regular.ttf}

\newtcolorbox{addinfo}[1][]{
    colback=black!5,
    colframe=black!30,
    title=اطلاعات اضافی
}

\linespread{1.5}

\title{ارائه درس رایانش ابری}
\author{محمد خورشیدی روزبهانی\\40215741002013 \and شارا شاهوردیان\\40215741002032}
\date{}

\begin{document}

    \maketitle

    \tableofcontents

    \newpage

    \chapter{خوبی‌های ابری}

        ظهور رایانش ابری مزایای مدل خدماتی را به کاربران رایانش ارائه می‌دهد. کاربران رایانش اکنون به عنوان مشترکان یا مصرف‌کنندگان شناخته می‌شوند زیرا به سمت رایانش ابری حرکت می‌کنند. رایانش ابری به مشترکان خود از طریق شبکه داخلی و اینترنت ارائه می‌شود. مشترکان می‌توانند به تسهیلات رایانشی به صورت اشتراکی در هر زمان و هر مکان دسترسی داشته باشند.

        تعداد زیادی از مزایای رایانش ابری کاربران را به سمت خود جلب می‌کند. اما هر نوآوری جدید با چالش‌هایی همراه است و رایانش ابری هم استثنا نیست. این فصل مزایای مختلف رایانش ابری را بررسی می‌کند و همچنین چالش‌هایی که در پیش روی آن قرار دارد را مطرح می‌کند.
    
        بزرگ‌ترین چالش مربوط به امنیت داده و مسائل پیروی از استانداردها است. بیشتر چالش‌های بحرانی دیگر به دلیل عدم وجود استانداردهای باز هستند که تامین‌کنندگان ابرها از استاندارد یا فناوری اختصاصی خود استفاده می‌کنند. نکته مثبت این است که تلاش‌های قابل توجهی برای حل این مسائل صورت گرفته است. علاوه بر این، این فصل به طور مختصر نقش خدمات وب در توسعه رایانش ابری را 
        معرفی می‌کند.

        \section{منشا اصطلاح رایانش ابری}

            اصل واژه "رایانش ابری" به اوایل دهه ۱۹۹۰ برمی‌گردد. در آن روزهای ابتدایی طراحی شبکه، مهندسان شبکه به ترسیم نمودارهای شبکه که دستگاه‌ها و اتصالات مختلف را نشان می‌دادند، علاقه داشتند. در چنین نمودارهایی، آنها عرصه‌های شبکه بیرونی را با نماد ابر نشان می‌دادند زیرا جزئیات آنها در دسترس آنها نبود. این در آن دوره به عنوان "ابر شبکه" یا "ابر" در صنعت شبکه‌ها شناخته می‌شد، اما امروزه ما به همان معنا "رایانش ابری" را نمی‌گوییم.

            با آغاز فعالیت‌های رایانش کاربردی به سوی انتهای قرن گذشته، شرکت‌های نرم‌افزاری بزرگ بر روی ارائه برنامه‌ها از طریق اینترنت تمرکز کردند. خدمات ایمیل در طی این دوره به سرعت گسترش یافت زیرا تامین‌کنندگان شروع به ارائه این امکان به کاربران خود کردند. و پرچمدارترین اقدام به نظر از Salesforce.com آمد که در سال ۱۹۹۹ برنامه‌های تجاری را برای شرکت‌ها از طریق اینترنت ارائه داد. اما همه این تلاش‌ها به عنوان بخشی از توسعه تسهیلات رایانش کاربردی دیده می‌شد. تا آن زمان هیچ رایانش ابری ظاهر نشده بود.

            \begin{addinfo}
                
                مفهوم محاسبات ابری با معنای فعلی‌اش حول سال ۲۰۰۶ در بازار ظاهر شد.
                
            \end{addinfo}

            در سال‌های ابتدای قرن فعلی، تعداد کمی از افراد صنعتی که در توسعه امکانات محاسبات کاربردی (utility computing facility) دخیل بودند، آن را به عنوان "ابر" نامیدند. اما تا سال ۲۰۰۶ این اصطلاح با مفهوم فعلی‌اش در صنعت تجاری ظاهر نشد. احتمالاً اولین بار اصطلاح "محاسبات ابری" در یک فروم رسمی توسط مدیر عامل گوگل آن زمان، اریک شمیت، در سال ۲۰۰۶ در یک کنفرانس استفاده شد. همچنین، در آن زمان استفاده گسترده‌ای از این اصطلاح مشاهده شد زیرا چندین شرکت مانند آمازون، مایکروسافت، آی‌بی‌ام تلاش‌های خود در زمینه محاسبات ابری را به دست عمومی رساندند. آمازون خدمات نوآورانه‌اش به نام ابر محاسبات الاستیک (Elastic Compute Cloud یا EC2) را در سال ۲۰۰۶ معرفی کرد.

            معنای واژه "ابر" به معنای مه، حجاب یا ابهام است. به معنای منطقی، این به منطقیت "یک مجموعه از منابع محاسباتی که جزئیات آن از دید کاربران مخفی است" می‌خورد.


        \section{ابتکارات اولیه}

            این ابتکار از سوی Salesforce.com در سال ۱۹۹۹ برای ارائه برنامه‌های کسب و کار (Enterprise) از طریق یک وب‌سایت "معمولی" به عنوان نخستین تلاش از این دست محسوب می‌شود. موفقیت این تلاش از سوی Salesforce، سایر شرکت‌های نرم‌افزاری را برای ارائه برنامه‌های کسب و کار از طریق اینترنت تشویق کرد. این لحظه تحولی بود که شرکت‌های فناوری محاسباتی شروع به ابتکار‌هایی در توسعه برنامه‌های کسب و کار مبتنی بر محاسبات ابری نمودند.

            \begin{addinfo}
                
                سرویس Salesforce.com که در سال ۱۹۹۹ راه‌اندازی شد، نخستین ابتکار تجاری موفق برای ارائه برنامه‌های کسب و کار از طریق اینترنت بود. این ابتکار اولین گام به سوی محاسبات ابری بود.

            \end{addinfo}

            پیشرفت بعدی اصلی از سوی آمازون با راه‌اندازی سرویس وب آمازون (Amazon Web Service یا AWS) در سال ۲۰۰۲ اتفاق افتاد که خدمات محاسباتی را از طریق اینترنت ارائه می‌داد. AWS مجموعه‌ای از خدمات از جمله ذخیره‌سازی را فراهم می‌کرد. آمازون نقش کلیدی در توسعه مدل محاسباتی مبتنی بر امکانات (utility model based computing services) در آن دوران ایفا کرد و به زودی بسیاری از شرکت‌های نرم‌افزاری شروع به به‌روزرسانی مراکز داده خود برای پشتیبانی از محاسبات امکانات نمودند. یک مرکز داده (Data Center) مخزن فیزیکی سازمان‌یافته‌ای از سیستم‌های محاسباتی و اجزای مرتبط مانند ذخیره‌سازی و شبکه‌سازی است که شرکت‌ها برای ساخت تسهیلات محاسباتی خود تحت نظر دارند.

            این حرکت به آرامی به سوی آنچه که امروز با نام محاسبات ابری شناخته می‌شود، متمایل شد. در سال ۲۰۰۶، آمازون سرویس وب EC2 را راه‌اندازی کرد، جایی که شرکت‌ها و افراد می‌توانستند کامپیوترهای (مجازی) را برای اجرای برنامه‌های کامپیوتری خود اجاره کنند. بعد از مدتی از راه‌اندازی EC2، "Google Docs" توسط گوگل (در سال ۲۰۰۶) محصولی دیگر در زمینه خدمات محاسبات ابری را به توجه عمومی معرفی کرد.

            در سال ۲۰۰۷، Salesforce.com سرویس دیگری به نام force.com را راه‌اندازی کرد، جایی که هر کسی می‌توانست برنامه‌ها را بسازد و وب‌سایت‌ها را راه‌اندازی کند. در سال ۲۰۰۹، ورود تجاری مایکروسافت به عرصه محاسبات ابری با راه‌اندازی "Windows Azure" رخ داد. هدف از Azure این بود که به مشتریان امکان اجرای برنامه‌های ویندوز خود را از طریق اینترنت فراهم کند.

            علاوه بر این ابتکارهای تجاری، در آن سال‌ها بسیاری از سازمان‌های تحقیقاتی و انجمن‌های منبع باز با آغاز ابتکارات محاسبات ابری خود، فعالیت‌های محاسبات ابری خود را آغاز کردند. به عنوان مثال، ناسا یک پلتفرم محاسبات ابری منبع باز به نام "Nebula" را در سال ۲۰۰۸ برای استفاده داخلی خود توسعه و راه‌اندازی کرد.

            \begin{addinfo}
                
                محاسبات ابری الاستیک (Elastic Compute Cloud یا EC2) که توسط آمازون در سال ۲۰۰۶ معرفی شد، اولین سرویس ابری ملموسی بود که در بازار عرضه شد.

            \end{addinfo}

        \section{محاسبات کاربردی}

            محاسبات کاربردی یک مدل تجاری محاسباتی است که در آن یک طرف به نام فروشنده یا ارائه‌دهنده، تسهیلات محاسباتی را به صورت درخواستی ترتیب می‌دهد، از آن مراقبت می‌کند، و ارائه می‌دهد. استفاده از تسهیلات محاسباتی در سطح ارائه‌دهنده اندازه‌گیری می‌شود. مشترکان می‌توانند به صورت پرداختی به تسهیلات محاسباتی دسترسی پیدا کنند و هنگام نیاز از آن استفاده کنند. تعریف قیمت‌گذاری ممکن است به دو شکل باشد: اجاره ثابت یا بر اساس استفاده واقعی. در مدل اجاره ثابت، مبلغ ثابتی بر اساس نیاز مشترک در طول یک مدت ثابت (معمولاً به ماه‌ها شمرده می‌شود) دریافت می‌شود. در روش بر اساس استفاده واقعی، قیمت‌گذاری بر اساس مصرف واقعی خدمات توسط یک مشترک در یک دوره زمانی انجام می‌شود.

            \begin{addinfo}
                
                مدل محاسبات کاربردی (Utility computing)، پیاده‌سازی مدل کاربردی تحویل خدمات در محاسبات است.

            \end{addinfo}

            Figure 3.1 توصیف کننده دو جنبه مهم مدل کاربردی تحویل خدمات است. خدمات بر اساس تقاضای کاربر (به حجم کم یا زیاد) در دسترس است و این خدمات خدماتی اند که اندازه‌گیری می‌شوند. ارائه‌دهندگان خدمات می‌توانند بر اساس اندازه‌گیری استفاده شان از خدمات، هزینه را از مشترکین دریافت کنند.

            در مدل کاربردی، تسهیلات محاسباتی به همراه منابع محاسباتی (شامل پردازنده، ذخیره‌سازی و غیره) به صورت بسته‌بندی شده ارائه می‌شود که می‌توان از محل دور دسترسی پیدا کرد. هزینه محاسباتی در این مدل محاسباتی کاهش می‌یابد زیرا همان مجموعه منابع بین تعداد زیادی از مشترکین به اشتراک گذاشته می‌شود. در اینجا، مشترکین می‌توانند با استفاده از مسیرهای ارتباطی شبکه‌ای یا شبکه‌های اینترنت، به تقریباً تأمین ناپایانی از راه‌حل‌های محاسباتی دسترسی پیدا کنند.

            \begin{addinfo}
                
                تأمین منابع توسط طرف ارائه‌دهنده بر اساس تقاضای کاربران به عنوان "خدمات درخواستی" شناخته می‌شود.

            \end{addinfo}

            IBM، HP و Sun Microsystems در پایان قرن گذشته ابتکارات محاسباتی کاربردی خود را آغاز کردند. این شرکت‌ها سرمایه زیادی را برای انجام تحقیقات در مورد محاسبات درخواستی سرمایه‌گذاری کردند. آن‌ها رهبران در ابتکارهای توسعه محاسبات کاربردی بودند. بعدها، مایکروسافت، گوگل، آمازون و دیگران به این رقابت پیوستند.

            \subsection{مزایا}

                همانطور که در فصل ۱ بحث شده است، در اوایل قرن بیستم، ارائه برق به عنوان خدمات عمومی نگرانی‌ها و شکوفایی‌های مردم را برانگیخت. محاسبات به عنوان خدمات کاربردی نیز شبیه این نگرانی‌ها را به وجود می‌آورد. بحث‌ها در مورد مزایا و معایب برون‌سپاری محاسبات به عنوان خدمات کاربردی منظم است. دو گزینه در دسترس است؛ یا نگه‌داری مرکز داده خود، یا مصرف خدمات محاسبات ابری.

                برای برق بیش از یک قرن پیش، زمانی که برق به عنوان خدمات کاربردی در دسترس نبود، کسب‌وکارها باید تیم داخلی را نگه دارند یا کار نگه‌داری و اجرای نیروگاه‌های تولید برق را داخل ساختمان خود برون‌سپاری کنند. محاسبات در شکل سنتی خود به همان روش در مرکز داده خود سازمان حفظ می‌شود. اما چه اتفاقی می‌افتد زمانی که یک شرکت یا سازمان وظایف اجرای مرکز داده خود را به یک شخص ثالث برون‌سپاری می‌کند؟ یا چقدر مطلوب است که مرکز داده خود را با استفاده از تیم داخلی اجرا کند، زمانی که سازمان در زمینه محاسبات تخصص ندارد؟ در اینجا گزینه سوم وارد می‌شود. محاسبات ابری که تقریباً کلیه تسهیلات محاسباتی برون‌سپاری می‌شود اما به شکلی کاملاً متفاوت. سوال پیش می‌آید که آیا این گزینه ایمن‌تر از دو گزینه اولیه به شرکت‌های معتبری که در زمینه محاسبات تخصص دارند، مسئولیت‌ها را واگذار کنند. مدل کاربردی برون‌سپاری هزینه‌های عملیات فناوری اطلاعات را برای مشتریان به شکل قابل توجهی کاهش می‌دهد که یک مزیت بزرگ در این بازار بسیار چالشی است. محاسبات کاربردی همچنین مدل سرمایه‌گذاری را از سرمایه‌گذاری یکبار در سرمایه‌گذاری‌های کوچک و متغیر منتقل می‌کند.
            
                این نوع مدل محاسباتی نیز به ارائه‌دهندگان خدمات محاسباتی منافع می‌رساند. سرمایه‌گذاری‌های آنها در ساختارهای سخت‌افزاری و نرم‌افزاری می‌تواند چندین راه‌حل را فراهم کند و به تعداد زیادی از کاربران خدمات ارائه دهد. این در نهایت منجر به بازگشت بهتر سرمایه گذاری می‌شود.

                \begin{addinfo}
                    
                    مدل خدمات کاربردی امکان سرمایه‌گذاری اولیه بسیار پایینی را فراهم می‌کند و همچنین صرفه‌جویی در هزینه کلی برای مشترکین.

                \end{addinfo}

        \section{اندازه‌گیری و صورتحساب در ابر}

            در محاسبات ابری، اندازه‌گیری واقعی خدمات محاسباتی ممکن شده است. در مدل‌های محاسبات خوشه‌ای قبلی، تعدادی از عملیات اندازه‌گیری ابتدایی وجود داشت، اما این عملیات کافی برای اندازه‌گیری استفاده واقعی از خدمات نبود. پیشرفت‌های فناوری ترکیبی که در محاسبات ابری به کار گرفته شده است، این قابلیت را فراهم می‌کند که مصرف توسط مشترکین با دقت اندازه‌گیری شود.

            استفاده‌ها برای انواع مختلف تسهیلات مانند پردازش، ذخیره‌سازی یا پهنای باند شبکه اندازه‌گیری می‌شوند. مشترکین بر اساس استفاده‌شان از منابع محاسباتی صورتحساب می‌گیرند. به عنوان مثال، یک مشترک که از تسهیلات محاسباتی در محاسبات ابری استفاده می‌کند، بر اساس استفاده از قدرت محاسباتی (هم پردازنده و هم حافظه)، استفاده از ذخیره‌سازی (در صورت وجود) و مصرف پهنای باند شبکه خود در طول زمان صورتحساب می‌شود. مشترکینی که از تسهیلات ذخیره‌سازی در ابر استفاده می‌کنند، بر اساس حجم ذخیره‌سازی واقعی مورد استفاده‌شان صورتحساب می‌شوند. قابلیت اندازه‌گیری محاسباتی ابری منجر به صرفه‌جویی قابل توجهی در هزینه برای کاربران می‌شود.

            \begin{addinfo}
                
                در محاسبات سنتی، عملیات اندازه‌گیری ابتدایی کافی برای اندازه‌گیری استفاده واقعی از محاسبات نبود. اما محاسبات ابری با این توانایی مجهز شده است.

            \end{addinfo}

        \section{جداسازی عملیات مرکز داده}

            عملیات مرکز داده محاسباتی همواره برای بیشتر مصرف کنندگان محاسبات، به ویژه شرکت‌ها، بار سنگینی بوده است. عملیات مرکز داده شامل تنظیم فضا برای توسعه زیرساخت، اطمینان از تأمین برق بدون وقفه، ایجاد سیستم خنک‌کننده و به‌ویژه ساخت زیرساخت محاسباتی و غیره می‌شود. علاوه بر این‌ها، نگهداری سیستم، اجرای به‌روزرسانی‌ها، بازیابی در صورت خرابی سیستم یا حفاظت از سیستم در برابر حملات شبکه، همه فعالیت‌های انتهای مرکز داده هستند. مدل محاسبات ابری کاملاً عملیات مرکز داده را از وظایف محاسباتی کاربر پایانی (مانند توسعه برنامه) جدا می‌کند. توسعه دهندگان نرم‌افزار یا کاربران برنامه همیشه سعی کرده‌اند از وظایف مدیریت زیرساخت محاسباتی خود دوری کنند. بنابراین، در مدل محاسبات سنتی، برون‌سپاری مدیریت مرکز داده ویژگی متداولی بوده است.

            در محاسبات ابری، مراکز داده در یک انتهای دور اقامت دارند و توسط چند فروشنده محاسبات مدیریت می‌شوند. فروشنده همه چیز را تنظیم و مدیریت می‌کند. کاربران می‌توانند به طور کامل و کاملاً تمرکز خود را بر روی وظایف خاص خود متمرکز کنند. این امکان به عنوان یک تسکن بزرگ برای کاربران محاسباتی آمده است.

            \begin{addinfo}
                
                محاسبات ابری عملیات مرکز داده را از سایر فعالیت‌ها در انتهای کاربران جدا می‌کند.

            \end{addinfo}

        \section{مزایای رایانش ابری}

            محاسبات ابری یک تغییر دیگر در دامنه محاسبات را به وجود آورده است. برخلاف استفاده‌های معمول فناوری کامپیوتر، این امکان را فراهم می‌کند که محاسبات به عنوان یک خدمات کاربردی ارائه شود که بر اساس درخواست ارائه می‌شود. تسهیلات محاسباتی توسط ارائه‌دهندگان مدیریت می‌شود و می‌تواند به حجم استفاده یا زمان استفاده اندازه‌گیری شود.

            همه این ویژگی‌های محاسبات ابری چندین مزیت ارائه می‌دهد. این امکان را دارد که کاربران در هر زمان دلخواهشان به اندازه‌ی خودشان بخواهند استفاده کنند. این مزایا تأثیر گذاری بر انتخاب محاسبات ابری را نسبت به روش محاسباتی سنتی دارد. بخش بعدی به مزایای مختلفی که مشترکان محاسبات ابری می‌توانند بهره‌مند شوند، می‌پردازد.

            \subsection{هزینه اکتساب / خرید کمتر}

                در محاسبات سنتی، کاربران باید منابع محاسباتی را به میزان قابل توجهی در ابتدا خریداری یا تهیه کنند. محاسبات ابری طبق مدل خدمات کاربردی ارائه می‌شود. از آنجایی که فروشنده در این مدل تمام منابع مورد نیاز را تنظیم می‌کند، سرمایه‌گذاری اولیه مشترکان برای تهیه سخت‌افزار یا نرم‌افزار به طور چشمگیری کاهش می‌یابد. آنها نیازی به تنظیم هیچ چیزی جز سیستم‌های مشتری برای دسترسی به خدمات ابری ندارند. بنابراین، هزینه سرمایه‌گذاری اولیه کاربر به طور قابل توجهی کاهش می‌یابد.

                \begin{addinfo}
                    
                    سرمایه‌گذاری اولیه کاربرانی که به محاسبات ابری می‌پردازند، بسیار کم است.

                \end{addinfo}

            \subsection{کاهش هزینه عملیاتی}

                با مدل برون‌سپاری محاسبات کاربردی، هزینه اجرای هر سیستم به صورت ۲۴ ساعته به سمت ارائه‌دهنده متمایل می‌شود. مشترکان از مسئولیت مدیریت سیستم، نگهداری و پشتیبانی انرژی ۲۴ × ۷ و همچنین پشتیبانی از خنک‌کننده خلاص می‌شوند. این اساس صرفه‌جویی در هزینه است زیرا مشترکان می‌توانند خدمات را با پرداخت مبلغ بسیار کم استفاده کنند. از طرف دیگر، ارائه‌دهنده می‌تواند خدمات را با هزینه کم به مشترکان ارائه دهد به دلیل حجم کسب و کار خود (به دلیل وجود پایگاه مشتریان بزرگ).

                \begin{addinfo}
                    
                    مشترکان خدمات محاسبات ابری باید هزینه عملیاتی ناچیزی را تحمل کنند.

                \end{addinfo}

            \subsection{کاهش مسئولیت مدیریت سیستم}

                باشد مرکز داده برای شرکت‌ها یا یک دستگاه مستقل (کامپیوتر شخصی، لپ‌تاپ و غیره) برای کاربران عادی، مدیریت تنظیمات محاسباتی (هم سخت‌افزار و هم نرم‌افزار) برای مصرف‌کنندگان محاسبات سنتی یک سرگرمی اضافی است. مدل محاسبات ابری اکثر تسهیلات زیرساختی و سایر وظایف مدیریت سیستم را به سمت فروشندگان ابری منتقل می‌کند. تیم‌های اختصاصی در انتهای فروشنده از تمام این فعالیت‌ها مراقبت می‌کنند. بنابراین، کاربران می‌توانند احساس آسایش کنند و تنها بر روی حوزه (لایه) خاص محاسباتی خود تمرکز کنند بدون این که نگران مدیریت لایه‌های زیرین محاسباتی باشند.

                \begin{addinfo}
                    
                    محاسبات ابری باعث آزادی کاربران از وظیفه مدیریت سیستم محاسباتی زیرین می‌شود.

                \end{addinfo}

    %%%%%%%%%%%%%%%%%%%%%%%%%%%%%%%%%%%%%%%%%%%%%%%%%%%%%%%%%%%%%%%%%%%%%%%%%%%%%%%%%%%%%%%%%%%%%%%%
    %%%%%%%       Shara        %%%%%%%%%%%%%%%%%%%%%%%%%%%%%%%%%%%%%%%%%%%%%%%%%%%%%%%%%%%%%%%%%%%%%
    %%%%%%%%%%%%%%%%%%%%%%%%%%%%%%%%%%%%%%%%%%%%%%%%%%%%%%%%%%%%%%%%%%%%%%%%%%%%%%%%%%%%%%%%%%%%%%%%

    \subsection{امکان پرداخت بر اساس استفاده}
    محاسبات ابری هزینه‌ای را برای مشترکانشان در زمانی که از آن استفاده نمی‌شوند، دریافت نمی‌کنند. حتی هزینه‌ی آن ثابت نیست؛ بلکه بستگی به مدت استفاده دارد. به جای آن، هر استفاده‌ای اندازه‌گیری می‌شود و کاربران بر اساس مصرف خود، هزینه‌ی مناسبی پرداخت می‌کنند. این باعث کاهش هزینه‌های محاسباتی می‌شود.

    \subsection{قدرت محاسباتی و ذخیره‌سازی نامحدود}
    در محاسبات ابری، کاربران می‌توانند به طور آسان به قدرت محاسباتی شبیه به یک سوپرکامپیوتر با هزینه‌ای مناسب دسترسی پیدا کنند، در صورت نیاز. در روش سنتی، تنها شرکت‌های بزرگ می‌توانستند هزینه‌های کامپیوتری پیشرفته را تحمل کنند. ذخیره‌سازی نیز یک مسئله مهم برای کاربران است. ابر به اندازه‌ای ذخیره‌سازی فراهم می‌کند که لازم است. این تقریباً نامحدود است که به عنوان یک مزیت بزرگ برای کاربران می‌تواند مشاهده شود.

    \subsection{کیفیت خدمات}
    در محاسبات سنتی، شرکت‌ها اغلب بخش‌های اصلی از کارهای مرتبط با محاسبات را به شرکای سومی برون‌سپاری می‌کردند. بنابراین، کیفیت خدمات به طور گسترده از تخصص شرکای سوم یا تیم‌های داخلی مدیریت آن وابسته بود. در حالی که در محاسبات ابری، کیفیت بالای خدمات (QoS) تضمین می‌شود زیرا توسط تامین‌کنندگان معتبر محاسباتی که دارای کارکنان آموزش‌دیده و تخصص دارند، ارائه می‌شود که به طور اختصاصی در زمینه محاسبات فعالیت می‌کنند.
    
    % box
    وقتی خدمات توسط تامین‌کنندگان معروف ارائه می‌شود، کیفیت خدمات تضمین می‌شود و این وظیفه تامین‌کننده می‌شود.

    \subsection{اعتماد‌پذیری}
    توانایی ارائه خدمات با کیفیت و پشتیبانی از توازن بار، پشتیبان‌گیری و بازیابی در صورت عدم موفقیت، باعث می‌شود که تامین‌کنندگان ابر معروف بسیار قابل اعتماد باشند که اغلب به عنوان یک مشکل بزرگ در محاسبات سنتی مطرح می‌شود. در محاسبات ابری، مشترکان دیگر نیازی به برنامه‌ریزی برای همه این وظایف پیچیده ندارند زیرا تامین‌کنندگان این مسائل را در دست می‌گیرند و آنها را بهتر انجام می‌دهند.

    \subsection{پیوستگی در دسترسی}
    تامین‌کنندگان معتبر ابر تقریباً اطمینان از دسترسی به خدمات 24 ساعته و 7 روزه را فراهم می‌کنند. آمار نشان داده است که زمان فعالیت سرویس (ارائه شده توسط تامین‌کنندگان معتبر) در طول یک سال معمولاً به کمتر از 99.9٪ نمی‌رسد. این پایداری مداوم تضمین‌شده خدمات ابر برای هر کسب و کاری یک مزیت بزرگ است.

    \subsection{استقلال مکانی/راحتی دسترسی}
    محاسبات ابری از طریق اینترنت در همه جا در دسترس است. کاربران می‌توانند از طریق هر دستگاه محاسباتی مانند کامپیوترهای شخصی یا دستگاه‌های قابل حمل مانند تبلت، لپ‌تاپ یا تلفن هوشمند به آن دسترسی پیدا کنند. تنها چیزی که برای بهره‌مندی از محاسبات ابری از طریق این دستگاه‌ها لازم است، دسترسی به اینترنت است، بدون توجه به مکان جغرافیایی یا منطقه زمانی.

    % box
    راحتی دسترسی و سرمایه‌گذاری کم، محاسبات ابری را به یک حوزه با حداقل مانع ورود تبدیل کرده‌است.

    \subsection{مقاومت بالا}
    مقاومت توانایی کاهش شدت و/یا مدت مختل شدن توسط شرایط ناخواسته است. سطح بالاتری از مقاومت ارزش بسیاری در محیط محاسباتی دارد. محاسبات ابری بر پایه زیرساخت محاسبات مقاوم ساخته شده‌است و بنابراین خدمات ابری به حملات و اشکالات بیشتری مقاومت دارند. مقاومت زیرساختی از طریق تکرار زیرساخت همراه با مکانیزم مؤثری برای پیش‌بینی، جذب و سازگاری دست‌یافته می‌شود. مصرف‌کنندگان ابر می‌توانند با بهره‌گیری از مقاومت منابع IT مبتنی بر ابر، اعتبار کسب‌وکارهای خود را افزایش دهند.

    \subsection{راه‌اندازی سریع}
    زمان راه‌اندازی در محیط ابر به طرز قابل توجهی کاهش یافته است نسبت به آنچه در محیط محاسبات سنتی بود. این امکان از طریق تأمین منابع سریع و خودکار در محیط ابر فراهم می‌شود. در یک بازار بسیار رقابتی، توانایی راه‌اندازی سریع مزایای کسب و کار قابل توجهی را به دنبال دارد.

    % box
    راه‌اندازی سریع‌تر سیستم یا برنامه مزیت تجاری را در بازار رقابتی به دست می‌آورد.

    \subsection{به‌روزرسانی‌های نرم‌افزاری خودکار}
    مسئله به‌روزرسانی نرم‌افزار در محیط محاسبات سنتی بسیار زحمت‌آور است. به‌طور دوره‌ای پچ‌های جدید منتشر می‌شوند و کاربران باید این پچ‌ها را به طور دوره‌ای اجرا کنند. در محیط محاسبات ابری، این به‌روزرسانی به‌صورت خودکار اتفاق می‌افتد. تامین‌کنندگان ابر همواره آخرین نسخه موجود هر نرم‌افزار را (مگر اینکه درخواست دیگری داده شده باشد) ارائه می‌دهند. محیط به‌روزرسانی‌شده تقریباً فوراً پس از ارائه در دسترس کاربران قرار می‌گیرد، و هر زمان کاربر بعدی وارد سیستم شود، موجود خواهد بود.

    \subsection{تهیه مجوز نیاز نیست}
    تهیه مجوز برنامه‌ها نیازمند ترتیبات بودجه‌ای جداگانه در محاسبات سنتی بود. علاوه بر این، برنامه‌های اضافی معمولاً با بسته‌های مجوزی ارائه می‌شدند. محاسبات ابری نیز این مشکل را برطرف کرده است. در اینجا، کاربران نیازی به تهیه هرگونه مجوز دوره‌ای برای استفاده از برنامه‌ها ندارند؛ بلکه به آنها اجازه می‌دهند که بر اساس استفاده‌شان از هر نرم‌افزار، پرداخت کنند (پس از پرداخت).
    
    % box
    در محاسبات ابری، مجوز نرم‌افزار دیگر نگرانی برای کاربران نیست.

    \subsection{امنیت در برابر فاجعه}
    خرابی سیستم به دلیل شکست فنی ناگهانی یا فاجعه طبیعی نگرانی اصلی برای کاربران است. به ویژه، هرگونه آسیب به دستگاه‌های ذخیره‌سازی فیزیکی ممکن است باعث ضرر تجاری عظیم شود. محاسبات ابری ارائه شده توسط تامین‌کنندگان معتبر دارای سیستم‌های بازیابی قدرتمندی در تنظیمات خود است. بنابراین، سیستم‌ها و داده‌ها از لحاظ ایمنی و امنیت، در محاسبات ابری نسبت به سیستم‌های قبلی بیشتر محافظت می‌شوند.

    \subsection{دوستی به محیط زیست}
    محاسبات ابری از محیط زیست دوستانه حمایت می‌کند. استفاده مناسب از منابع، نیاز کلی به منابع الکترونیکی را کاهش می‌دهد، بنابراین تولید زباله‌های الکترونیکی را نیز کاهش می‌دهد. این برای محیط زیست مفید است زیرا اگر زباله‌های الکترونیکی به درستی پردازش نشوند، برای اکوسیستم مضر هستند. علاوه بر این، کاهش نیاز به منابع منجر به کاهش تقاضا و بنابراین تولید منابع محاسباتی می‌شود. این کاهش در تولید الکترونیکی باعث کاهش انتشار کربن و کمک به کاهش کلی ردپای کربن می‌شود.

    % box
    محاسبات ابری یک رویکرد محاسباتی سبز و دوستدار محیط زیست است.


    این لیست طولانی از مزایایی که در بالا بحث شد، در شکل ۳.۲ نمایش داده شده‌اند. آنها نشان می‌دهند که محاسبات ابری چه قدر مفید است و چرا انقدر افراد به آن مشتاق هستند.

    % شکل

    \section{چالش‌های محاسبات ابری}

\end{document}