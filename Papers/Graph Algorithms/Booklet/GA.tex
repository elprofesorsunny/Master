\documentclass[a4paper,10pt]{book}

\usepackage{geometry}
\geometry{
    right=1cm,
    left=1cm,
    top=2cm,
    bottom=2cm
}

\usepackage{dblfnote}
\usepackage{lipsum}

\usepackage{xepersian}
\settextfont{Vazirmatn-Regular.ttf}

\title{\Huge جزوه الگوریتم‌های گراف}
\author{استاد مربوطه:\\سرکار خانم دکتر معصومه دامرودی}
\date{نویسنده:\\محمد خورشیدی روزبهانی}

\linespread{1.5}

\begin{document}

    \maketitle

    سلام

    \tableofcontents

    \chapter{مقدمات}

        در دنیای امروزی پر از فناوری و ارتباطات، مفاهیمی مانند گراف‌ها به عنوان ابزارهای بسیار قدرتمندی در حل مسائل گوناگون مورد استفاده قرار می‌گیرند. گراف‌ها، نه تنها در علوم کامپیوتر بلکه در زمینه‌های مختلفی از جمله شبکه‌های اجتماعی، حمل و نقل، مهندسی، و بیولوژی نیز کاربرد دارند.

        مسئله اویلر
        \footnote{\hspace{2pt}Euler}
        و پل کونیگزبرگ
        \footnote{\hspace{2pt}bridges Koningsberg}
        به عنوان یکی از مسائل کلاسیک و جذاب در تئوری گراف مطرح است که تاریخچه طولانی‌اش به اوایل قرن هجدهم بازمی‌گردد. این مسئله ابتدا توسط ریاضی‌دان معروف لئونارد اویلر
        \footnote{\hspace{2pt}Euler Leonhard}
        مطرح شد و بعدها توسط رابرت لویس کونیگزبرگ
        \footnote{\hspace{2pt}Konigsberg Louis Robert}
        مورد تحقیق و توسعه قرار گرفت.

        پل کونیگزبرگ، یکی از شهرهای کوچک در استان پروسیا
        \footnote{\hspace{2pt}Prussia of Province}
        (اکنون بخشی از شهر کالینینگراد، روسیه
        \footnote{\hspace{2pt}Russia Kaliningrad,}
        )، دارای هفت جزیره‌ی متصل با پل‌هایی به نام‌های «کونیگزبرگ» بود. مسئله مطرح شده توسط مردم محلی این بود که آیا می‌توان از همه‌ی این پل‌ها گذر کرد و به ازای هر پل فقط یک بار وارد شد؟

        با بررسی دقیق ساختار شبکه از نقاط (جزیره‌ها) و یال‌ها (پل‌ها)، اویلر و کونیگزبرگ به این نتیجه رسیدند که می‌توان این مسئله را به یک مسئله‌ گرافی تبدیل کرد. آن‌ها نشان دادند که اگر یک گراف دارای یک مسیر اولیه باشد که همه‌ی یال‌ها را فرا بگیرد (مسیر اویلر)، آنگاه می‌توان یک دور یال‌ها را طی کرد که هر یال را فقط یک بار گذر کرده باشیم (دور اویلر). این مسئله نه تنها اهمیت تاریخی دارد بلکه به عنوان یک مسئله مبنایی در تئوری گراف و همچنین در الگوریتم‌های کاربردی مانند جستجوی در عمق و جستجوی در عرض مورد استفاده قرار می‌گیرد. توانایی حل این مسئله با استفاده از الگوریتم‌های مناسب از جمله نشان دهنده‌ی فهم عمیق و قدرت الگوریتم‌های گرافی است.
    
        سِر ویلیام همیلتون
        \footnote{\hspace{2pt}Hamilton William Sir}
        بودای (۸ مارس ۱۷۸۸ - ۶ سپتامبر ۱۸۵۶) دیپلمات، زبان‌شناس، و دانشمند بریتانیایی بود. او بیشتر به خاطر کشف‌هایش در زمینه حساب مجرد و هندسه معروف است. همیلتون مفهوم ویژه‌ای را به نام کواترنیو
        \footnote{\hspace{2pt}Quaternion}
        در جبر خطی ارائه داد و مفاهیم جدیدی را در هندسه معمولی و معادلات دیفرانسیل معرفی کرد. همیلتون همچنین به خاطر کارهای خود در زمینه فلسفه و مطالعات یونان باستان نیز شناخته می‌شود. او به زبان یونانی مسلط بود و در ترجمه اثرهای بزرگی از جمله سوکراتیکا
        \footnote{\hspace{2pt}Socratic}
        و الهینه
        \footnote{\hspace{2pt}Elenchus}
        و همچنین تالیف نظریه‌های خود درباره این آثار مشهور شد.

        گراف‌ها به عنوان ابزارهای قدرتمندی در بسیاری از زمینه‌ها و صنایع مورد استفاده قرار می‌گیرند. برخی از مزایا و کاربردهای کلیدی گراف‌ها در موارد امروزی عبارتند از:

        \begin{enumerate}
            
            \item شبکه‌های اجتماعی: گراف‌ها به طور گسترده در شبکه‌های اجتماعی مانند فیسبوک، توییتر، و لینکدین استفاده می‌شوند تا روابط بین افراد و شبکه‌های اجتماعی را مدل‌سازی کنند و الگوهای اجتماعی را بررسی کنند.

            \item جستجوی اطلاعات و موتورهای جستجو: موتورهای جستجو از گراف‌ها برای مدل‌سازی وابستگی بین صفحات و اطلاعات در وب استفاده می‌کنند تا جستجوی بهتری برای کاربران فراهم کنند.

            \item بهینه‌سازی مسائل: گراف‌ها به عنوان ابزاری برای بهینه‌سازی مسائل مانند مسائل مسیریابی، زمان‌بندی و تخصیص منابع استفاده می‌شوند.

            \item حوزه‌های حمل و نقل و مسائل شهری: در حوزه حمل و نقل، گراف‌ها برای مدل‌سازی شبکه‌های جاده، مسیرهای حمل و نقل عمومی و ترافیک شهری استفاده می‌شوند.

            \item زیست‌شناسی محاسباتی و زیست‌انفورماتیک: در زیست‌شناسی محاسباتی، گراف‌ها برای مدل‌سازی شبکه‌های تعاملات ژنتیکی، مسیرهای متابولیک و شبکه‌های پروتئینی استفاده می‌شوند.

            \item تجزیه و تحلیل شبکه‌های مخابراتی: در مخابرات، گراف‌ها برای مدل‌سازی و تحلیل شبکه‌های ارتباطی و ترافیک شبکه استفاده می‌شوند.

        \end{enumerate}

        به طور کلی، گراف‌ها به عنوان یک ابزار قدرتمند برای مدل‌سازی، تحلیل، و بهینه‌سازی سیستم‌ها و روابط پیچیده در موارد مختلف از جمله علوم کامپیوتر، مهندسی، علوم زندگی، و اقتصاد مورد استفاده قرار می‌گیرند.

        \section{گراف چیست؟}

            سلام
        
\end{document}