\documentclass[a4paper,10pt]{article}

\usepackage{geometry}
\geometry{
    a4paper,
    left=1cm,
    right=1cm,
    top=1cm,
    bottom=1cm}
\usepackage{xepersian}
\settextfont{Vazirmatn-Regular.ttf}


\title{گزارش الگوریتم تکاملی مبتنی بر متقاطع یکپارچه برای رنگ‌آمیزی نمودارهای دارای وزن راس}
\author{محمد خورشیدی روزبهانی\\40215741002013 \and شارا شاهوردیان\\40215741002032}
\date{}

\linespread{1.5}

\begin{document}

    \maketitle
    
    \vspace{0.5cm}

    \begin{abstract}

        رنگ‌آمیزی نمودار یکی از مسائل اصلی بهینه‌سازی است که به طور گسترده در ادبیات مورد مطالعه قرار گرفته است. در این مطالعه، یک الگوریتم تکاملی نوآورانه به نام الگوریتم تکاملی مبتنی بر متقاطع یکپارچه با عملگر متقاطع منحصر به فرد خود و تکنیک جستجوی محلی برای رنگ‌آمیزی نمودارهای دارای وزن راس پیشنهاد می‌شود. عملگر متقاطع یکپارچه به منظور استفاده از اطلاعات خاص حوزه در افراد و تکنیک جستجوی محلی به منظور کاوش در حلقه‌های مجاور با استفاده از عملیات‌های تعویض وزن هدفمند می‌باشد. عملکرد کار پیشنهادی بر روی مجموعه داده‌های مصنوعی و نمونه‌های DIMACS با مقایسه آن با الگوریتم‌های تکاملی پیشرو از ادبیات ارزیابی می‌شود. مطالعه تجربی نشان می‌دهد که الگوریتم ما در ۷۱٪ موارد آزمایشی بهتر از کار مرتبط عمل کرده است و در ۱۷٪ موارد آزمایشی مصنوعی به نتیجه مشابهی می‌رسد. آزمایش‌های انجام شده بر روی نمونه‌های DIMACS نشان می‌دهد که الگوریتم ما تعداد بهترین رنگ‌ها را در ۷۰ از ۷۳ نمودار پیدا می‌کند، بنابراین کار پیشنهادی در زمان معقول موفق به رنگ‌آمیزی نمودارهای دارای وزن راس می‌شود.
    
    \end{abstract}

    \section{مقدمه}

        سلام

\end{document}
