\documentclass[a4paper,10pt]{article}

\usepackage{geometry}
\geometry{
    a4paper,
    left=1cm,
    right=1cm,
    top=1cm,
    bottom=2cm}
\usepackage{xepersian}
\settextfont{Vazirmatn-Regular.ttf}


\title{گزارش الگوریتم تکاملی مبتنی بر متقاطع یکپارچه برای رنگ‌آمیزی نمودارهای دارای وزن راس}
\author{محمد خورشیدی روزبهانی\\40215741002013 \and شارا شاهوردیان\\40215741002032}
\date{}

\linespread{1.5}

\begin{document}

    \maketitle
    
    \vspace{0.5cm}

    \begin{abstract}

        رنگ‌آمیزی نمودار یکی از مسائل اصلی بهینه‌سازی است که به طور گسترده در ادبیات مورد مطالعه قرار گرفته است. در این مطالعه، یک الگوریتم تکاملی نوآورانه به نام الگوریتم تکاملی مبتنی بر متقاطع یکپارچه با عملگر متقاطع منحصر به فرد خود و تکنیک جستجوی محلی برای رنگ‌آمیزی نمودارهای دارای وزن راس پیشنهاد می‌شود. عملگر متقاطع یکپارچه به منظور استفاده از اطلاعات خاص حوزه در افراد و تکنیک جستجوی محلی به منظور کاوش در حلقه‌های مجاور با استفاده از عملیات‌های تعویض وزن هدفمند می‌باشد. عملکرد کار پیشنهادی بر روی مجموعه داده‌های مصنوعی و نمونه‌های DIMACS با مقایسه آن با الگوریتم‌های تکاملی پیشرو از ادبیات ارزیابی می‌شود. مطالعه تجربی نشان می‌دهد که الگوریتم ما در ۷۱٪ موارد آزمایشی بهتر از کار مرتبط عمل کرده است و در ۱۷٪ موارد آزمایشی مصنوعی به نتیجه مشابهی می‌رسد. آزمایش‌های انجام شده بر روی نمونه‌های DIMACS نشان می‌دهد که الگوریتم ما تعداد بهترین رنگ‌ها را در 70 از 73 نمودار پیدا می‌کند، بنابراین کار پیشنهادی در زمان معقول موفق به رنگ‌آمیزی نمودارهای دارای وزن راس می‌شود.
    
    \end{abstract}

    \section{مقدمه}

        نظریه رنگ‌آمیزی نمودار به تقسیم یک مجموعه از راس‌ها به کلاس‌های رنگی جداگانه تحت شرطی می‌پردازد که هیچ راسی که یال مشترکی دارد نمی‌تواند به یک کلاس تخصیص یابد. هدف مسئله کلاسیک رنگ‌آمیزی نمودار، تعیین کوچکترین مقدار رنگ k برای به‌دست آوردن یک راه‌حل قانونی است. مسئله k-رنگ‌آمیزی نمودار (k-GCP) سعی دارد یک رنگ‌آمیزی ممکن برای یک نمودار با یک مقدار داده شده k پیدا کند. اگر راه‌حل بدون تضاد باشد، آنگاه یک k-رنگ‌آمیزی قانونی به‌دست می‌آید. مسئله رنگ‌آمیزی نمودار می‌تواند با حل یک سری مسائل k-رنگ‌آمیزی حل شود. با شروع با یک مقدار k کافی بزرگ، مقدار k می‌تواند هر بار که یک رنگ‌آمیزی قانونی پیدا شود، کاهش یابد. این فرایند تکرار می‌شود تا زمانی که یک راه‌حل غیرقانونی به‌دست آید. هدف اصلی k-GCP کمینه‌سازی تعداد یال‌های تضادی برای یک مقدار ثابت k است.

        مسائل رنگ‌آمیزی نمودار با اهداف مختلفی وجود دارند. مسئله رنگ‌آمیزی متقارن [1] شامل یک رنگ‌آمیزی قانونی با اختصاص رئوس به k کلاس رنگی مستقل است که تعداد رئوس در این کلاس‌ها حداکثر می‌تواند یکی اختلاف داشته باشد، در حالی که هدف مسئله کمینه‌سازی جمع رنگ‌های اختصاص یافته به رئوس است. همچنین، وزن‌ها می‌توانند به رئوس در مسائل رنگ‌آمیزی نمودار اضافه شوند [2]. هدف مسئله رنگ‌آمیزی رأسی وزن‌دار، به‌دست آوردن یک رنگ‌آمیزی قانونی k-رنگی با هدف کمینه‌سازی جمع هزینه‌های کلاس‌های رنگ آن است. هزینه یک کلاس رنگی توسط رأسی که وزن بیشتری در کلاس دارد، تعیین می‌شود.

        مسئله رنگ‌آمیزی نمودار به طور معمول برای مدل‌سازی مسائل واقعی مانند برنامه‌ریزی زمان، تخصیص منابع و تخصیص ثبت‌ها [3]–[5] استفاده می‌شود. اکثر این مسائل دارای تعداد محدودی منابع هستند. از آنجا که k-رنگ‌آمیزی مقدار رنگ ثابت k را در نظر می‌گیرد، این مقدار می‌تواند به تعداد منابع موجود در سیستم اشاره کند، بنابراین k-رنگ‌آمیزی می‌تواند برای حل این مسائل استفاده شود. در اکثر موارد، تعداد رنگ‌ها، k، ممکن است کافی نباشد تا یک رنگ‌آمیزی قانونی به‌دست آید، بنابراین برخی از رئوس بدون رنگ خواهند ماند. اهمیت رئوس ممکن است برابر نباشد، بنابراین از یک مقدار وزن برای نشان دادن اهمیت آن‌ها استفاده می‌شود. کار پیشنهادی ما مسئله k-رنگ‌آمیزی (k-GCP) را با استفاده از یک گراف دارای وزن رأسی با هدف کمینه‌سازی جمع وزن کل رئوس بدون رنگ برای یک مقدار داده شده k در نظر می‌گیرد. مسئله k-رنگ‌آمیزی رأسی با وزن از یک گراف بدون جهت و دارای وزن G=(V،E،w) استفاده می‌کند، که در آن V مجموعه رئوس را نشان می‌دهد، E مجموعه یال‌ها را نشان می‌دهد و w مجموعه مقادیر وزن رئوس در V است که تاکیدی بر اهمیت آن‌ها دارند. هدف k-GCP انجام رنگ‌آمیزی رئوس در V با استفاده از یک تعداد پیش‌تعریف شده از رنگ‌ها است. اگر تعداد داده شده از رنگ‌ها نتواند همه رئوس را رنگ‌آمیزی کند، برخی از رئوس بدون رنگ خواهند ماند. رئوس بدون رنگ به عنوان رئوس تضادی تعریف می‌شوند. تابع تناسب f (k) برابر با جمع وزن کل رئوس بدون رنگ هنگام استفاده از یک تعداد پیش‌تعریف شده از k رنگ است. هدف k-GCP کمینه‌سازی مقدار تابع تناسب f (k) است.

        مسئله رنگ‌آمیزی نمودار به عنوان یک مسئله NPComplete اثبات شده است [6] و بسیاری از روش‌های هیوریستیکی برای مسئله رنگ‌آمیزی نمودار [7]–[11]، مسئله رنگ‌آمیزی متقارن [12]، [13]، مسئله k-رنگ‌آمیزی [14]، [15]، مسئله کمینه‌سازی جمع رنگ‌ها [16] و مسئله رنگ‌آمیزی رأسی وزن‌دار [17] در ادبیات پیشنهاد شده‌اند. در این مطالعه، یک الگوریتم تکاملی ترکیبی [18] برای مسئله k-رنگ‌آمیزی رأسی وزن‌دار پیشنهاد می‌شود. الگوریتم ما به نام الگوریتم تکاملی مبتنی بر متقاطع یکپارچه (InCEA) با عملگر متقاطع جدید خود و تکنیک جستجوی محلی است. عملگر متقاطع یکپارچه گروه بیشینه‌ای از رئوس بدون تضاد را به کلاس‌های رنگی فرزندان با استفاده موفقیت‌آمیز از اطلاعات خاص مسئله در والدین تقسیم می‌کند. دو کلاس رنگی انتخاب شده به‌طور تصادفی از والدین به‌طور تدریجی ترکیب می‌شوند تا هر کلاس رنگی از فرزندان شکل گیرد. رئوس تضادی به استخر انداخته می‌شوند و هر بار که یک کلاس رنگی جدید از فرزندان ایجاد می‌شود، رئوس در استخر امتحان خود را برای پیدا کردن یک کلاس رنگی بدون تضاد می‌دهند. اگر در پایان عملگر متقاطع، رئوسی در استخر موجود باشند، تکنیک جستجوی محلی سعی می‌کند این رئوس را به یکی از کلاس‌های رنگی فرزندان با استفاده از یک عملیات تعویض وزن‌دار قرار دهد تا مقدار تابع تناسب کمینه شود.

        در مطالعه تجربی ما، الگوریتم InCEA را با کارهای مرتبط از ادبیات مقایسه کردیم. نتایج به دست آمده از مجموعه داده‌های مصنوعی تولید شده و نمونه‌های DIMACS نشان می‌دهد که InCEA در بیشتر موارد آزمایشی از نظر مقادیر تناسب و زمان محاسبات عملکرد کارهای مرتبط را برتری می‌بخشد. عملکرد الگوریتم‌ها بر روی ۷۳ نمونه DIMACS که گراف‌های چالش برای مسئله رنگ‌آمیزی نمودار هستند، نیز در این مطالعه ارائه شده است. از آنجا که تعداد کمینه رنگ‌های مورد نیاز برای رنگ‌آمیزی گراف‌ها برای این بنچمارک‌ها در ادبیات پیدا نشده است، این مقاله همچنین کمینه تعداد رنگ‌های استفاده شده برای رنگ‌آمیزی این نمونه‌ها را گزارش می‌دهد. همانطور که در مطالعه تجربی اشاره شده، الگوریتم تکاملی ما می‌تواند در بیشتر موارد آزمایشی ارائه شده به سرعت بسیار خوبی نتایج بهتری را نسبت به الگوریتم‌های موجود در ادبیات به دست آورد.

        مهم‌ترین مشارکت‌های این کار می‌تواند به شرح زیر باشد:

        \begin{itemize}
            
            \item عملگر متقاطع یکپارچه به منظور استفاده از اطلاعات خاص مسئله در افراد با کمک یک استخر و یک عملیات جستجوی معکوس هدفمند استفاده می‌شود.

            \item عملیات تعویض وزن‌دار در تکنیک جستجوی محلی به منظور کاوش در حلقه‌های مجاور و افزایش فرصت برای رسیدن به بیشینه جهانی هدفمند است.

            \item هیچ محاسبات اضافی برای تکنیک جستجوی محلی یا محاسبه تناسب لازم نیست زیرا استخر از قبل رأس(های) بدون رنگ را نگه می‌دارد، بنابراین الگوریتم از بار جستجوهای جامع خلاص می‌شود.

        \end{itemize}

        در بقیه مقاله، ابتدا مطالعات مرتبط و تعریف مسئله مسئله k-رنگ‌آمیزی به ترتیب در بخش‌های 2 و 3 ارائه می‌شوند. در بخش 4، جزئیات اجزای کار پیشنهادی ما را توضیح می‌دهیم. سپس، عملکرد الگوریتم ما را با الگوریتم‌های موجود در ادبیات بر روی نمونه‌های آزمایشی مختلف در بخش 5 مقایسه و بررسی می‌کنیم. در نهایت، خلاصه کار پیشنهادی خود را ارائه می‌دهیم و پیشنهاداتی برای جهت‌های آینده ممکن را در بخش 6 ارائه می‌دهیم.

    \section{کار مرتبط}

        بسیاری از مسائل واقعی مانند برنامه‌ریزی زمان، تخصیص ثبت‌ها و تخصیص منابع می‌توانند با مسئله رنگ‌آمیزی نمودار نمایش داده شوند، جایی که رئوس معرفی‌کننده اشیاء و یال‌ها محدودیت‌ها را نشان می‌دهند. در تمام این مسائل، تعداد محدودی از منابع (رنگ‌ها) وجود دارد، بنابراین برخی از رئوس بدون رنگ خواهند ماند. وزن‌ها می‌توانند به رئوس اضافه شوند تا اهمیت آن‌ها را نشان دهند، و یک گراف دارای وزن رأسی می‌تواند برای مدل‌سازی این مسائل استفاده شود.

        در مسئله تخصیص ثبت‌ها [19]، هدف این است که حداکثر تعداد متغیرها را به حداقل تعداد ثبت‌ها اختصاص داد تا متغیرها به سرعت بالایی توسط واحد پردازش مرکزی (CPU) دسترسی داشته باشند. مسئله می‌تواند با استفاده از یک گراف بدون جهت نمایش داده شود، جایی که متغیرها توسط رئوس نشان داده می‌شوند و مداخلات بین متغیرها توسط یک یال نشان داده می‌شود. هدف این مسئله کمینه کردن تعداد ثبت‌های استفاده شده (رنگ‌ها) و انتخاب متغیرهایی است که کمتر دسترسی دارند، که می‌تواند به طور مستقیم برای انتخاب رئوس بدون رنگ و محاسبه تابع تناسب در k-GCP [20] اعمال شود.

        مسئله تخصیص منابع نیز می‌تواند با استفاده از k-GCP حل شود. یکی از مهم‌ترین مسائل تخصیص منابع در شبکه‌های کامپیوتری [21]–[25]، تخصیص پهنای باند در شبکه‌های بی‌سیم است. در این مسئله، شبکه با سنسورهای خود می‌تواند به عنوان یک گراف بدون جهت و رئوس نمایش داده شود. اگر فاصله بین دو سنسور بیشتر از یک حد پیش‌تعیین شده باشد، این سنسورها نمی‌توانند از همان باند استفاده کنند، بنابراین رئوس نماینده آنها توسط یک یال متصل می‌شوند. هدف این مسئله کمینه کردن تعداد باندها (رنگ‌ها) برای تمامی سنسورها است. همچنین وزن رئوس کیفیت سنسورها را نشان می‌دهد.

        برنامه‌ریزی یک مسئله چالش‌برانگیز برای آموزش [26]، محاسبات [27]، تولید [28] و ارتباطات [29]، [30] است. به طور کلی، مسائل بر روی یک عنصر مانند یک دانش‌آموز، یک تولید، یک کارمند یا یک کار با ویژگی‌های مختلف خود تمرکز دارند. عناصر به عنوان رئوس نمایش داده می‌شوند و تضادها بین این عناصر با یال‌ها در گراف نمایش داده می‌شوند. اهمیت عناصر با وزن رئوس نشان داده می‌شود. زمان، ماشین‌ها، منابع، وسایل نقلیه به عنوان رنگ‌ها استفاده می‌شوند و تابع هدف می‌تواند بیشینه‌سازی کارایی زمانی، بهره‌وری منابع و غیره باشد. بنابراین، مسئله رنگ‌آمیزی نمودار یک انتخاب مناسب برای مسائل برنامه‌ریزی است [31].

        هنگامی که مسائل واقعی با استفاده از مسئله رنگ‌آمیزی نمودار مدل‌سازی می‌شوند، بسیاری از روش‌های دقیق [32]–[34] و هیوریستیک [35]–[38] به دلیل پیچیدگی NP-Hard آن [6], [39], [40] قابل استفاده هستند. اگرچه الگوریتم‌های دقیق پیشنهادی برای حل مسئله رنگ‌آمیزی نمودار [41] می‌توانند بهترین راه‌حل‌ها را برای نمونه‌های کوچک پیدا کنند، اما برای نمونه‌های بزرگ هزینه زیادی از نظر مصرف حافظه و زمان دارند [42].

        الگوریتم تکاملی یکی از روش‌های هیوریستیک برای مسئله رنگ‌آمیزی نمودار است [2]. الگوریتم‌های تکاملی فرایند طبیعی را با عملگرهای متقاطع و جهش شبیه‌سازی می‌کنند. این الگوریتم‌ها با یک جمعیت سر و کار دارند که تعداد پیش‌تعریف شده‌ای از راه‌حل‌های کاندید که به عنوان افراد برای یک گراف دارای وزن رأسی داده شده، نگه داشته می‌شوند [43].

        در ادبیات، الگوریتم‌های تکاملی وجود دارند که مسائل واقعی را که به k-GCP رمزگذاری شده‌اند را حل می‌کنند، مانند الگوریتم تکاملی ترکیبی (HEA) [44] و الگوریتم ممتاز به منظور هزینه (COMA) [45].

        HEA و COMA جمعیت اولیه خود را با استفاده از سه معیاری که در [44] پیشنهاد شده‌اند، ایجاد می‌کنند. جزئیات این معیارها در بخش IV-A ذکر شده است. هنگامی که جمعیت‌های اولیه ایجاد شدند، هر دو HEA و COMA اپراتورهای متقاطع خود را برای بهبود افراد خود اعمال می‌کنند. الگوریتم‌ها دو والدین را از جمعیت خود انتخاب کرده و کلاس‌های رنگی این والدین را ترکیب می‌کنند تا فرزندی ایجاد کنند. انتخاب کلاس‌های رنگ در این الگوریتم‌ها متفاوت است. HEA از عملگر تقسیم بدون تضاد (CFPX) استفاده می‌کند که کلاس‌های رنگ را با بیشترین زیرمجموعه رأس بدون تضاد انتخاب می‌کند. از سوی دیگر، COMA کلاس رنگ را با بیشترین وزن زیرمجموعه بدون تضاد انتخاب می‌کند و عملگر متقاطع آن به نام عملگر تقاضای گرایی هزینه (COPX) است.

        فرزندانی که توسط CFPX یا COPX ایجاد می‌شوند، تضمین نمی‌کنند که یک راه‌حل بدون تضاد تولید کنند. HEA و COMA اپراتور جستجوی محلی را که در [44] پیشنهاد شده است، اعمال می‌کنند. آنها یکی از رأس‌های تضادی را از یک کلاس رنگی از فرزند با استفاده از مقادیر درجه و وزن رأس‌ها انتخاب می‌کنند. هر دو الگوریتم تمامی کلاس‌های رنگ را برای قرار دادن این رأس تضادی بازدید می‌کنند و آن را به کلاس رنگی اختصاص می‌دهند که باعث بیشینه کاهش در مقدار تابع تناسب شود. تعداد ایتریشن‌ها در این مرحله جستجو به ۱۰٪ از کل تعداد تضادها در HEA محدود شده است، در حالی که COMA چنین محدودیتی ندارد و برای هر رأس تضادی در فرزند تمامی کلاس‌های رنگ را بازدید می‌کند. تکنیک‌های مبتنی بر تضاد و مبتنی بر هزینه در HEA معرفی شده‌اند، و کمترین مقدار تابع تناسب به دست آمده از این دو رویکرد به عنوان مقدار تابع تناسب فرزند انتخاب می‌شود. در حالی که COMA از تکنیک‌های مبتنی بر تضاد، مبتنی بر هزینه و مبتنی بر معیار استفاده می‌کند تا مقدار تابع تناسب فرزند را محاسبه کند.

    \section{بیان مسئله}

        گراف وزن‌دار بدون جهت $G(V,E,w)$ را در نظر بگیرید، جایی که $V$ و $E$ مجموعه‌های رئوس و یال‌ها هستند، و $w$ مجموعه‌ی مقادیر وزن رئوس شامل $V$ است. اگر $V$ دارای $n$ راس باشد، اندازه مجموعه وزن، $|w|$، همچنین $n$ است و $E$ حداکثر می‌تواند 
        $\frac{n \times (n-1)}{2}$
        یال داشته باشد، که
        $0 \leq |E| \leq \frac{n \times (n-1)}{2}$

        هر رأس $v \in V$ به یک کلاس رنگ $C_i$ تعلق دارد، که یکی از مجموعه‌های مستقل جدایی شده از 
        $C = \{C_1, C_2, C_3, \dots, C_k\} $
        است، به طوری که
        $1 \leq k \leq n$.

        اگر $u \in V$ مجاور $v \in V$ باشد، در این صورت یک یال $\{u, v\} \in E$ وجود دارد و $u$ و $v$ نمی‌توانند در یک رنگ باشند (1). این به عنوان رنگ‌آمیزی قانونی یا k-ممکن شناخته می‌شود.

        \begin{equation} 
            \forall v, \hspace{1em} u \in C_i, \{u, v\} \notin E, \hspace{1em} i = 1, 2, 3, \dots, k
        \end{equation}

        برای یک مقدار داده شده از تعداد کلاس‌های رنگ $k$، هدف مسئله k-رنگ‌آمیزی ارائه یک کلاس رنگ برای هر رأس $v$ به گونه‌ای است که شرط (1) را رعایت کرده و یک رنگ‌آمیزی ممکن ارائه دهد. اگر $v$ نتواند به یک کلاس رنگ اختصاص یابد، آنگاه $v$ به عنوان یک رأس تضادی تعریف شده و غیررنگ‌آمیزی می‌شود. در این صورت، راه‌حل غیرقابل اجرا است و هدف مسئله k-رنگ‌آمیزی کمینه کردن مجموع وزن‌های رأس‌های تضادی با استفاده از $f(k)$ است (2).

        \begin{equation}
            minimize \hspace{0.5em} f(k) = \sum w(v), \hspace{1em} v \notin C
        \end{equation}

    \section{الگوریتم تکاملی مبتنی بر متقاطع یکپارچه}

        سلام



\end{document}
