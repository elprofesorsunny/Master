\documentclass[a4paper,10pt]{article}

\usepackage{geometry}
\geometry{
    left=1cm,
    right=1cm,
    bottom=2cm,
    top=2cm
}

\usepackage{dblfnote}
\usepackage{lipsum}

\usepackage{xepersian}
\settextfont{Vazirmatn-Regular.ttf}


\title{سوالات احتمالی میان‌ترم 1 درس الگوریتم‌های گراف (با پاسخ)}
\author{استاد مربوطه:\\سرکار خانم دکتر معصومه دامرودی \and به نوشته:\\محمد خورشیدی روزبهانی}
\date{}

\linespread{1.5}

\begin{document}

    \maketitle

    \paragraph{سوال:} تعاریف، قضایا، نتایج و کاربردهای درون اسلایدها را شرح دهید.

    \paragraph{پاسخ:} ابتدا به تعاریف موجود در اسلایدها پرداخته شده و موارد به تفکیک توضیح داده می‌شود.

    \begin{itemize}
        
        \item راس\footnote{\hspace{2pt}Node - Vertex}: راس در گراف نقطه‌ای است که به عنوان یک نقطه یا یک گره شناخته می‌شود و می‌تواند با مقادیر مختلفی مانند عدد صحیح یا رشته مشخص شود.
        
        \item یال\footnote{\hspace{2pt}Edge}: یال در یک گراف، اتصال بین دو راس یا گره است و نشان‌دهنده رابطه بین آن دو راس می‌باشد.
        
        \item در تماس\footnote{\hspace{2pt}Incident}: در مفهوم گراف، هنگامی که یال به یک راس متصل است، می‌گوییم که یال با آن راس «در تماس» است.

        \item گراف ساده\footnote{\hspace{2pt}Graph Simple}: گراف ساده یک گراف است که هیچ یال تکراری یا حلقه‌ای (یالی که شروع و پایانش به یک راس یکسان است) ندارد. به عبارت دیگر، در گراف ساده، هیچ دو راس متصل نیز دوبار در یک یال قرار نمی‌گیرند.
        
        \item درجه\footnote{\hspace{2pt}Degree}: درجه یک راس در گراف، تعداد یال‌های متصل به آن راس است. به عبارت دیگر، درجه یک راس نشان‌دهنده تعداد یال‌هایی است که به آن راس متصل هستند.
        
        \item راس منفرد\footnote{\hspace{2pt}Vertex Isolated}: راس منفرد یک راس در گراف است که هیچ یالی به آن متصل نیست، به عبارت دیگر درجه این راس صفر است.
        
        \item متمم\footnote{\hspace{2pt}Complement}: در مفهوم گراف، متمم یک گراف، گرافی است که همه یال‌های موجود در گراف اصلی حذف شده و همه یال‌هایی که بین رئوس موجود نیستند اضافه شده‌اند. به عبارت دیگر، این گراف حاصل از دو گراف اصلی که هیچ یال مشترکی ندارند می‌باشد.

        \item گراف خالی\footnote{\hspace{2pt}Graph Empty}: گراف  خالی یک گراف است که هیچ راس و هیچ یالی ندارد. به عبارت دیگر، یک گراف با تعداد رئوس و یال‌های صفر است.

        \item زیرگراف\footnote{\hspace{2pt}Subgraph}: زیرگراف یک گراف است که تمام رئوس و یال‌های آن در گراف اصلی وجود داشته باشند. به عبارت دیگر، اگر یک گراف با رئوس و یال‌های خاصی را در نظر بگیرید، هر گرافی که شامل زیرمجموعه‌ای ار آن رئوس و یال‌ها باشد، زیرگرافی از آن گراف است.
        
        \item گراف صفر\footnote{\hspace{2pt}Graph Null}: گراف صفر یک گراف است که تنها یک راس دارد و هیچ یالی ندارد.

        \item گراف کامل\footnote{\hspace{2pt}Graph Complete}: یک گراف است که همه رئوس آن به همه رئوس دیگر با یک یال متصل هستند. به عبارت دیگر، در یک گراف کامل هیچ دو راسی وجود ندارد که بهم هم متصصل نباشند.

        \item گراف :k-منتظم\footnote{\hspace{2pt}Graph k-Regular} گراف k-منتظم یک گراف است که درجه همه رئوس آن برابر با k باشد. به عبارت دیگر، هر راس در این گراف، با k یال به راس‌های دیگر متصل است.

        \item چرخه\footnote{\hspace{2pt}Graph Circuit - Graph Cycle}: در مفهوم گراف، چرخه یک مسیر بسته است که از یک راس شروع شده، از یال‌های مختلف گذر کرده و در نهایت به همان راس اول باز می‌گردد. به عبارت دیگر، یک چرخه گرافی که شامل حداقل بک راس و حداقل یک یال است و اولین و آخرین راس‌ها یکسان هستند.
        
        \item گراف چرخ\footnote{\hspace{2pt}Graph Wheel}: در مفهوم گراف یک گراف چرخ یک نوع خاص از گراف است که از راس مرکزی و چندین راس دیگر تشکیل شده است که همگی به راس مرکزی متصل هستند و هیچ یالی بین رئوس غیرمرکزیِ غیرمجاور وجود ندارد. به عبارت دیگر، یک گراف چرخ همانند گراف چرخه است با این تفاوت که یک راس به عنوان مرکز در نظر گرفته می‌شود و همه رئوس دیگر به آن متصل می‌شوند.
        
        \item گراف مکعب n بُعدی\footnote{\hspace{2pt}Graph n-Cube}: در مفهوم گراف، مکعب n بُعدی یک گراف با ساختار مکعبی است که دارای $n^2$ راس و $n \times (n-1)^2$ یال است. این گراف معمولاً با استفاده از ارقام دودویی به عنوان برچسب رئوس تعریف می‌شود، به طوری که هر راس با یک دنباله n بیتی نمایش داده می‌شود و هر یال به دو راس متصل است که دنباله‌های باینری متفاوت در یک بیت داشته باشند.
        
        \item گراف مسیر\footnote{\hspace{2pt}Graph Path}: گراف مسیر یک گراف است که رئوس آن به صورت متوالی به هم متصل هستند و هیچ یال تکراری یا حلقه‌ای وجود ندارد. به عبارت دیگر، این گراف مانند زنجیره است که رئوس آن به ترتیب به یکدیگر متصل شده‌اند.
        
        \item گراف بازه\footnote{\hspace{2pt}Graph Interval}: گراف بازه یک نوع خاص از گراف است که رئوس آن بازه‌های اعداد حقیقی را نمایش می‌دهند و دو راس متصل هستند اگر و تنها اگر بازه‌های متناظر با آن دو راس تلاقی داشته باشند. به عبارت دیگر، گراف بازه می‌تواند به عنوان نمایشی از یک مجموعهٔ بازه‌های اعداد حقیقی دیده شود، که هر گره نمایانگر یک بازه است و یال بین دو گره وجود دارد اگر و تنها اگر بازه‌های متناظر با آن دو گره تلاقی داشته باشند.
        
        \item گراف قوس دایره‌ای\footnote{\hspace{2pt}Graph Circular-arc}: گراف قوس دایره‌ای، یک نوع خاص از گراف است که رئوس آن بازه‌های یک دایره را نمایش می‌دهند و دو راس متصل هستند اگر و تنها اگر بازه‌های متناظر با آن دو راس اشتراک غیرخالی داشته باشند. به عبارت دیگر، گراف قوس دایره‌ای می‌تواند به عنوان نمایشی از بازه‌های یک دایره دیده شود، که هر راس نمایانگر یک بازه است و دو راس متصل هستند اگر و تنها اگر بازه‌های متناظر با آن دو راس اشتراک غیرخالی داشته باشند.
        
        \item گراف دوبخشی\footnote{\hspace{2pt}Graph Bipartite}: گراف دوبخشی یک گراف است که مجموعه رئوس آن را می‌توان به دو زیرمجموعه جدا از هم تقسیم کرد، به طوری که هیچ راس درون هر زیرمجموعه با راسی در همان زیرمجموعه دیگر متصل نباشد. به عبارت دیگر، این گراف متشکل از دو مجموعه راس است که هر یال تنها بین یک راس از یک مجموعه و یک راس از مجموعه دیگر وجود دارد، نه دو راس از همان مجموعه.

        \item گراف دوبخشی کامل\footnote{\hspace{2pt}Graph Bipartite Complete}: گراف دوبخشی کامل یک گراف دوبخشی است که همه رئوس یک زیرمجموعه با همه رئوس زیرمجموعه دیگر به صورت کامل متصل هستند. به عبارت دیگر، هر راس در یک زیرمجموعه با همه راس‌های زیرمجموعه دیگر متصل است.
        
        \item پیاده‌روی\footnote{\hspace{2pt}Walk}: در مفهوم گراف، پیاده‌روی یک دنباله از رئوس و یال‌ها است که از یک راس شروع شده و در آن رئوس و یال‌ها به ترتیب دنبال می‌شوند، به طوری که هر راس در مسیر با یک یال به راس بعدی متصل باشد. به عبارت دیگر، پیاده‌روی می‌تواند شامل تکرار یال‌ها و رئوس باشد.
        
        \item مسیر\footnote{\hspace{2pt}Trail}: در مفهوم گراف، مسیر یک دنباله از یال‌ها و رئوس است که هر یال در آن فقط یک بار ظاهر شده ولی رئوس ممکن است چندین بار ظاهر شوند. به عبارت دیگر، یک مسیر یک پیاده‌روی است که هیچ یال تکراری ندارد.

        \item مسیر\footnote{\hspace{2pt}Path}: در مفهوم گراف، مسیر یک دنباله از رئوس است که هر راس در آن با ییکک یال به راس بعدی متصل است. به عبارت دیگر، این یک زنجیره از رئوس است که هیچ راس تکراری ندارد.

    \end{itemize}

\end{document}